% !TeX spellcheck = de_DE
\documentclass[12pt,parskip=full]{scrartcl}
 
\usepackage[utf8]{inputenc}
\usepackage[T1]{fontenc}
\usepackage{lmodern}
\usepackage[ngerman]{babel}
\usepackage{color}
\usepackage{amsmath,amssymb,amstext,mathtools,amsthm}
\usepackage{subcaption}
\usepackage{float}
\usepackage{stmaryrd}
\usepackage[hidelinks]{hyperref}
\hypersetup{bookmarksnumbered}
\usepackage{mathtools}

\usepackage{tikz}
\usetikzlibrary{positioning}

\usepackage{enumitem}
\setenumerate{label=\arabic*)}

\newcommand{\setN}{\mathbb{N}}
\newcommand{\setZ}{\mathbb{Z}}
\newcommand{\setQ}{\mathbb{Q}}
\newcommand{\setR}{\mathbb{R}}
\newcommand{\setC}{\mathbb{C}}
\newcommand{\setH}{\mathbb{H}}
\newcommand{\setk}{\Bbbk}
\newcommand{\ldot}{\,.\,}
\newcommand{\Forall}{~\forall}
\newcommand{\Exists}{~\exists}

\newcommand{\corners}[1]{{\ulcorner #1 \urcorner }}
\newcommand{\dotminus}{\buildrel\textstyle.\over{\hbox{\vrule height3pt depth0pt width0pt}{\smash-}}}
\newcommand{\abs}[1]{{\left| #1 \right|}}
\newcommand{\dabs}[1]{{\left\lVert #1 \right\rVert}}
\newcommand{\heading}{\underline}
 
\theoremstyle{definition}
\newtheorem{theorem}{Satz}[section]
\newtheorem{corollary}[theorem]{Folgerung}
\newtheorem{proposition}[theorem]{Proposition}
\newtheorem{lemma}[theorem]{Lemma}
\newtheorem{definition}[theorem]{Definition}
\newtheorem{example}[theorem]{Beispiel}
\newtheorem{axiom}[theorem]{Axiom}
\newtheorem{remark}[theorem]{Bemerkung}

\hfuzz=5pt
 
\title{Skript Modelltheorie}
\author{Lukas Metzger}
\date{\today}
 
\begin{document}
	\maketitle
	
	\setcounter{section}{-1}
	\section{Motivation}
	
	\heading{Aus der Linearen Algebra}
	\begin{itemize}
		\item $K$-Vektorräume, Untervektorräume, Homomorphismen
		\item Gruppen, Untergruppen, Homomorphismen
		\item Ringe, Unterringe, Homomorphismen
		\item Körper, Teilkörper, Homomorphismen
	\end{itemize}

	\heading{Entwicklungsschritte}
	\begin{itemize}
		\item Suche nach allgemeiner Theorie $\Rightarrow$ universelle Algebra.
		\item Modelltheorie (universelle Algebra + Logik)
		\item Kategorientheorie
	\end{itemize}

	\heading{Beispiel von Ax}
	
	Sei $K$ ein Körper, und $P(X) \in K[X]$. $P$ definiert eine Abbildung $\tilde{P}: K \to K$.
	
	$P$ hat die Hopf-Eigenschaft, wenn gilt:
	\begin{center}
		Wenn $\tilde{P}$ injektiv ist, dann ist $\tilde{P}$ surjektiv.
	\end{center}

	Jedes Polynom hat über einem endlichen Körper die Hopf-Eigenschaft.
	
	\heading{Formalisierung der Hopf-Eigenschaft}
	\begin{equation*}
		\forall y \forall z (P(y = P(z) \rightarrow y = z)
	\end{equation*}
	\begin{equation*}
		\forall w \exists v P(v) = w
	\end{equation*}
	
	Für jedes $n$
	\begin{equation*}
		\forall x_0, \dots, x_n \left(\forall y \forall z \left(\sum_{i=0}^{n} x_i y^i = \sum_{i=0}^n x_i z^i \rightarrow y = z\right) \rightarrow \forall w \exists v \sum_{i=0}^n x_i v^i = w\right)
	\end{equation*}
	Logik
	\begin{equation*}
		\underset{\text{log. äquivalent}}{\sim} \forall x_0, \dots \forall x_n \forall w \exists v \exists y \exists z \left( \sum_{i=0}^{n} x_i y^i = \sum_{i=0}^n x_i z^i \right) \rightarrow \sum_{i=0}^n x_i v^i = w
	\end{equation*}
	
	
	\begin{example}
		\begin{equation*}
			\mathbb{F}_{p^n} \underset{\text{erfüllt}}{\models} HE(n) \underset{\forall \exists- \text{Präservation}}{\Rightarrow} \underbrace{\bigcup_{n \in \setN} \mathbb{F}_{p^n}}_{\tilde{\mathbb{F}}_p = \text{ der algebraische Abschluss von } \mathbf{F_p}} \models HE(n)
		\end{equation*}
	\end{example}

	\begin{example}
		Aus dem Kompaktheitssatz folgt: $\setC = \lim\limits_{p \to \infty} \tilde{\mathbb{F}}_p$
	\end{example}

	\section{Grundbegriffe}
	
	\subsection{\texorpdfstring{$\mathcal{L}$-Strukturen}{L-Strukturen}}
	
	\begin{example}
		Der angeordnete Körper der reellen Zahlen $(\setR, \underbrace{+, \cdot}_\text{zweistellig}, \underbrace{-}_\text{einstellig}, \underbrace{0, 1}_\text{konstanten}, \underbrace{<}_\text{zweistellige Relation})$
	\end{example}
	
	\begin{definition}[$\mathcal{L}$-Struktur]
		Sei $\mathcal{L}$ eine Menge von
		\begin{itemize}
			\item Funktionszeichen $f_i \quad (i \in I)$
			\item Relationszeichen $R_j \quad (j \in J)$
		\end{itemize}
		
		Jedes Zeichen hat ein festes $n \in \setN$ als Stelligkeit (arity).
		
		$\mathcal{L}$ heißt Sprache / Signatur / similarity type.
		
		Eine $\mathcal{L}$-Struktur $\mathfrak{A}$ besteht aus
		\begin{itemize}
			\item einer nicht-leeren Menge $A$ (Universum, Träger, Grundmenge)
			\item einer $n$-stellige Funktion $f^\mathfrak{A}: A^n \to A$ für jedes $n$-stellige Funktionszeichen $f \in \mathcal{L}$
			\item einer $n$-stellige Relation $R^\mathfrak{A} \subseteq A^n$ für jedes $n$-stellige Relationszeichen $R \in \mathcal{L}$
		\end{itemize}
	
		\heading{$n = 0$}
		
		$A^0 = \{ \emptyset \}$
		
		$0$-stellige Funktion in $\mathfrak{A}$: $f^\mathfrak{A}: \{ \emptyset \} \to A$ ist eindeutig bestimmt durch $f(\emptyset) \in A$. Daher entsprechen $0$-stellige Funktionen den Konstanten.
		
		$0$-stellige Relationen in $\mathfrak{A}$:
		\begin{equation*}
			R^\mathfrak{A} \subseteq \{ \emptyset \} \begin{cases}
				\text{entweder} & R = \{ \emptyset \} \,\hat{=}\, \text{wahr}\\
				\text{oder} & R = \emptyset \,\hat{=}\, \text{falsch}
			\end{cases}
		\end{equation*}
		Daher entsprechen $0$-stellige Relationszeichen den Aussagenvariablen
	\end{definition}

	\begin{example}
		\begin{enumerate}[label=\alph*)]
			\item Zu jeder Menge $A \neq \emptyset$ und jeder Sprache $\mathcal{L}$ kann ich eine $\mathcal{L}$-Struktur mit Träger $A$ finden!
			\item $\mathcal{L} = \{ R \}$, $R$ 2-stelliges Relationssymbol
			\begin{align*}
				\mathfrak{Q}_1 = (\setQ, <), &\qquad\text{d.h.}\quad R^{\mathfrak{Q}_1} = \{ (q_1, q_2) \in \setQ^2 \mid q_1 < q_2 \} \\
				\mathfrak{Q}_2 = (\setQ, <), &\qquad\text{d.h.}\quad R^{\mathfrak{Q}_2} = \{ (q_1, q_2) \in \setQ^2 \mid q_1 < q_2 \}
			\end{align*}
			sind zwei verschiedene $\mathcal{L}$-Strukturen auf $\setQ$.
			\item $\mathcal{L}_{HGr} = \{ \circ \}$ und $\mathcal{L}_Gr = \{ \circ, {}^{-1}, e \}$
			
			Gruppen sind $\mathcal{L}_{Gr}$-Strukturen $\mathfrak{G}$ mit:
			\begin{itemize}
				\item $\circ^\mathfrak{G}$ ist assoziativ
				\item $e^\mathfrak{G} \circ^\mathfrak{G} g = g \circ^\mathfrak{G} e^\mathfrak{G} = g$ für alle $g \in G$
				\item $g \circ ^\mathfrak{G} g^{-1^\mathfrak{G}} = g^{-1^\mathfrak{G}} = e^\mathfrak{G}$
			\end{itemize}
		
			Alternativ sind Gruppen $\mathcal{L}_{HGr}$-Strukturen $\mathfrak{G}$ mit
			\begin{itemize}
				\item $\circ^\mathfrak{G}$ ist assoziativ
				\item es gibt ein neutrales Element
				\item es gibt inverse Elemente
			\end{itemize}
		\end{enumerate}
	\end{example}

	\begin{definition}
		Seien $\mathfrak{A}$ und $\mathfrak{B}$ $\mathcal{L}$-Strukturen. $h: A \to B$ heißt
		\begin{enumerate}[label=\alph*)]
			\item $\mathcal{L}$-Homomorphismus, falls
			\begin{equation*}
				h(f^\mathfrak{A}(a_1, \dots, a_n)) = f^\mathfrak{B}(h(a_1), \dots, h(a_n))
			\end{equation*}
			für alle $n$ und $a_1, \dots, a_n \in A$, und $n$-stellige $f \in \mathcal{L}$ und
			\begin{equation*}
				(a_1, \dots, a_n) \in R^\mathfrak{A} \Rightarrow (h(a_1), \dots, h(a_n)) \in R^\mathfrak{B}
			\end{equation*}
			für alle $n$ und $a_1, \dots, a_n \in A$, und $n$-stellige $R \in \mathcal{L}$.
			\item Starker Homomorphismus, falls zusätzlich $\Leftrightarrow$ im zweiten Teil gilt.
			\item $\mathcal{L}$-Einbettung falls $h$ injektiver starker $\mathcal{L}$-Homomorphismus ist.
			\item $\mathcal{L}$-Isomorphismus falls $h$ bijektiver starker $\mathcal{L}$-Homomorphismus ist und $h^{-1}$ ebenfalls.
			\item $\mathfrak{A}$ und $\mathfrak{B}$ heißen $\mathcal{L}$-Isomorph falls es ein $\mathcal{L}$-Isomorphismus $h: \mathfrak{A} \to \mathfrak{B}$ gibt.
			\item Ein $\mathcal{L}$-Isomorphismus $h: \mathfrak{A} \to \mathfrak{A}$ heißt $\mathcal{L}$-Automorphismus.
			\item Falls $A \subseteq B$, dann heißt $\mathfrak{A}$ $\mathcal{L}$-Unterstruktur von $\mathfrak{B}$ beziehungsweise $\mathfrak{B}$ $\mathcal{L}$-Oberstruktur von $\mathfrak{A}$, falls die Identität $id_A: A \to B$ eine $\mathcal{L}$-Einbettung ist.
		\end{enumerate}
	\end{definition}

	\begin{remark}
		Falls $\mathcal{L}' \subseteq \mathcal{L}$, dann wird jede $\mathcal{L}$-Struktur $\mathfrak{A}$ durch vergessen zu einer $\mathcal{L}'$-Struktur $\mathfrak{A}_{\upharpoonright \mathcal{L}'}$ (Redukt von $\mathfrak{A}$).
	\end{remark}

	\begin{remark}
		Jeder Halbgruppenhomomorphismus zwischen Gruppen ist ein Gruppenhomomorphismus.
		
		Falls $\mathfrak{G}_1, \mathfrak{G}_2$ $\mathcal{L}_{Gr}$-Strukturen sind und $h: G_1 \to G_2$ $L_{HGr}$ Homomorphismus (genau genommen ${G_1}_{\upharpoonright \mathcal{L}_{HGr}}$ und ${G_2}_{\upharpoonright \mathcal{L}_{HGr}}$) dann ist $h$ automatisch ein $\mathcal{L}_{Gr}$-Homomorphismus.
		
		Dies stimmt nicht für Monoide statt Gruppen.
	\end{remark}

	\begin{remark}""
		\begin{enumerate}
			\item Wenn $h: \mathfrak{A} \to \mathfrak{B}$ ein injektiver Homomorphismus ist (d.h. es existiert Sprache $\mathcal{L}$, die im Hintergrund fest ist, $\mathfrak{A},\mathfrak{B}$ sind $\mathcal{L}$-Strukturen, $h$ ist $\mathcal{L}$-Homomorphismus) dann existiert auf $h(A)$ eine $\mathcal{L}$-Struktur $h(\mathfrak{A})$, so dass $h: \mathfrak{A} \xrightarrow{\sim} h(\mathfrak{A})$, aber $h(\mathfrak{A})$ ist nicht notwendigerweise Unterstruktur von $\mathfrak{B}$.
			\item Der Schnitt von $\mathcal{L}$-Unterstrukturen ist wieder eine $\mathcal{L}$-Unterstruktur.
		\end{enumerate}
	\end{remark}
	
	\begin{corollary}
		Wenn $\mathfrak{A}$ eine $\mathcal{L}$-Struktur und $C \subset A$ ist, dann existiert die von $C$ erzeugte $\mathcal{L}$-Unterstruktur $\langle C \rangle_\mathcal{L} = \langle C \rangle$ das heißt die kleinste Unterstruktur von $\mathfrak{A}$, deren Trägermenge $C$ enthält.
		
		Die Trägermenge von $\langle C \rangle$ erhält man dadurch, dass man $C$ unter den Funktionen $f^\mathfrak{A}$ abschließt.
		
		$R^{\langle C \rangle}$ ist dann $R^\mathfrak{A} \cap \langle C \rangle \times \dots \times \langle C \rangle$
	\end{corollary}

	\subsection{\texorpdfstring{$\mathcal{L}$-Formeln}{L-Formeln}}

	\heading{Verwendete Symbole:}
	\begin{itemize}
		\item Funktions- und Relationszeichen aus $\mathcal{L}$:
		\begin{equation*}
			f_i, R_j, \dots, +, \circ, \leq
		\end{equation*}
		\item Gleichheitszeichen: $\doteq$ (Zieglersche Konvention)
		\item Klammern: $()$
		\item Quantoren: $\forall \quad \exists$
		\item aussagenlogische Junktoren: $\underset{\text{Negation}}{\lnot}, \underset{\text{und}}{\land}, \underset{\text{oder}}{\lor}, \underset{\text{Implikation}}{\rightarrow}, \underset{\text{Äquivalent}}{\leftrightarrow}, \underset{\text{Falsum}}{\bot}, \underset{\text{Verum}}{\top}$
		\item Individuenvariablen: $v_0, v_1, \dots$
	\end{itemize}

	\begin{definition}[$\mathcal{L}$-Terme]
		$\mathcal{L}$-Terme sind:
		\begin{itemize}
			\item Individuenvariablen
			\item Wenn $f$ ein $n$-stelliges Funktionszeichen in $\mathcal{L}$ ist und $\tau_1, \dots, \tau_n$ sind $\mathcal{L}$-Terme dann ist $f \tau_1 \dots \tau_n$ ein $\mathcal{L}$-Term.
		\end{itemize}
	\end{definition}

	\begin{remark}""
		\begin{itemize}
			\item Es gilt die eindeutige Lesbarkeit der Terme
			\item Bei Zeichen wie $+, \cdot$ schreibt man traditionell $v_1 + v_2$ statt $+ v_1 v_2$ muss aber bei Verschachtelungen klammern.
		\end{itemize}
	\end{remark}

	\begin{definition}[Auswertung von Termen in Strukturen]
		Eine Belegung der Individuenvariablen mit Elementen einer Struktur für eine $\mathcal{L}$-Struktur $\mathfrak{A}$ ist eine Abbildung $\beta: \{ v_0, v_1, \dots \} \to A$.
		
		Die Auswertung von einem Term in einer Struktur bezüglich einer Belegung $\tau^\mathfrak{A}[\beta]$ ist induktiv definiert durch:
		\begin{align*}
			v_i^\mathfrak{A}[\beta] &\coloneqq \beta(v_i) \\
			f \tau_1 \dots \tau_n {}^\mathfrak{A} [\beta] &\coloneqq f^\mathfrak{A}(\tau_1^\mathfrak{A}[\beta], \dots, \tau_n^\mathfrak{A}[\beta])
		\end{align*}
	\end{definition}

	\begin{definition}[$\mathcal{L}$-Formeln]
		$\mathcal{L}$-Formeln sind
		\begin{itemize}
			\item $\bot \quad \top$
			\item $\tau_1 \doteq \tau_2$ für $\mathcal{L}$-Terme $\tau_1, \tau_2$
			\item $R \tau_1 \dots \tau_n$ für $\mathcal{L}$-Terme $\tau_1, \dots, \tau_n$ und $n$-stelliges $R \in \mathcal{L}$
		\end{itemize}
	\end{definition}

	\begin{definition}[Auswertung von $\mathcal{L}$-Formeln in Strukturen]""\\
		$\mathfrak{A}$ ist Modell von $\varphi$ unter $\beta$ oder formal $\mathfrak{A} \models \varphi[\beta]$
		\begin{itemize}
			\item stets gilt $\mathfrak{A} \models \top[\beta]$
			\item nie gilt $\mathfrak{A} \models \bot[\beta]$
			\item $\mathfrak{A} \models \corners{\tau_1 \doteq \tau_2}[\beta] \Leftrightarrow \tau_1^\mathfrak{A}[\beta] = \tau_2^\mathfrak{A}[\beta]$
			\item $\mathfrak{A} \models R \tau_1 \dots \tau_n [\beta] \Leftrightarrow (\tau_1^\mathfrak{A}[\beta], \dots, \tau_n^\mathfrak{A}[\beta]) \in R^\mathfrak{A}$
			\item Wenn $\varphi, \varphi_1, \varphi_2$ $\mathcal{L}$-Formeln sind, dann auch
			\begin{align*}
				&\lnot \varphi && \mathfrak{A} \models \lnot \varphi[\beta] \Leftrightarrow \mathfrak{A} \not\models \varphi[\beta] \\
				&(\varphi_1 \land \varphi_2) && \mathfrak{A} \models (\varphi_1 \land \varphi_2) [\beta] \Leftrightarrow \mathfrak{A} \models \varphi_1[\beta] \text{ und } \mathfrak{A} \models \varphi_2[\beta] \\
				&(\varphi_1 \lor \varphi_2) && \mathfrak{A} \models (\varphi_1 \lor \varphi_2) [\beta] \Leftrightarrow \mathfrak{A} \models \varphi_1[\beta] \text{ oder } \mathfrak{A} \models \varphi_2[\beta] \\
				&(\varphi_1 \rightarrow \varphi_2)  && \mathfrak{A} \models (\varphi_1 \rightarrow \varphi_2) [\beta] \Leftrightarrow \text{Wenn } \mathfrak{A} \models \varphi_1[\beta] \text{ dann } \mathfrak{A} \models \varphi_2[\beta] \\
				&(\varphi_1 \leftrightarrow \varphi_2)  && \mathfrak{A} \models (\varphi_1 \leftrightarrow \varphi_2) [\beta] \Leftrightarrow (\mathfrak{A} \models \varphi_1[\beta] \Leftrightarrow \mathfrak{A} \models \varphi_2[\beta]) \\
				&\exists v_i \varphi && \text{Es gibt ein $a \in A$ so dass $\mathfrak{A} \models \varphi\left[\beta \frac{a}{v_i}\right]$}\\
				&\forall v_i \varphi && \text{Für alle $a \in A$ gilt dass $\mathfrak{A} \models \varphi\left[\beta \frac{a}{v_i}\right]$}
			\end{align*}
		\end{itemize}
	\end{definition}
	
	\begin{example}
		$\forall v_0 \underbrace{((\forall v_1 \underbrace{R v_0 v_1}_\text{Wirkungsbereich $\forall v_1$}) \lor R v_1 v_0)}_\text{Wirkungsbereich $\forall v_0$}$
		
		Variablen im Wirkungsbereich eines Quantors heißen gebundene Variablen, alle anderen heißen freie Variablen.
	\end{example}

	\begin{remark}
		$\tau^\mathfrak{A}[\beta]$ beziehungsweise $\mathfrak{A} \models \varphi[\beta]$ hängt nur insofern von $\beta$ ab, als man wissen muss, was $\beta$ mit den freien Variablen macht.
	\end{remark}

	\begin{definition}[$\mathcal{L}$-Aussage]
		Eine $\mathcal{L}$-Aussage ($\mathcal{L}$-Satz, geschlossene Formel) ist eine $\mathcal{L}$-Formel ohne freie Variablen.
	\end{definition}
	
	\begin{theorem}
		Für $\mathcal{L}$-Aussagen $\varphi$ ist $\mathfrak{A} \models \varphi[\beta]$ unabhängig von $\beta$.
		
		Man schreibt:
		\begin{align*}
			\mathfrak{A} & \models \varphi \\
			\mathfrak{A} & \not\models \varphi
		\end{align*}
	\end{theorem}

	\begin{definition}""
		\begin{enumerate}
			\item Eine $\mathcal{L}$-Formel $\varphi$ ist allgemeingültig ($\models \varphi, \vdash \varphi$), falls $\mathfrak{A} \models \varphi[\beta]$ für alle $\mathfrak{A}$ und $\beta$.
			\item $\mathcal{L}$-Formeln $\varphi$ und $\psi$ sind logisch äquivalent ($\varphi \sim \psi$), falls
			\begin{equation*}
				\mathfrak{A} \models \varphi[\beta] \Leftrightarrow \mathfrak{A} \models \psi[\beta]
			\end{equation*}
			für alle $\mathfrak{A}$ und $\beta$.
			\item $\psi$ folgt aus $\phi = \{ \varphi_i \mid i \in I \}$, falls:
			\begin{equation*}
				\mathfrak{A} \models \varphi_i[\beta] \text{ für alle $i \in I$} \quad\Longrightarrow\quad \mathfrak{A} \models \psi[\beta] \text{ für alle $\mathfrak{A}$ und $\beta$}
			\end{equation*}
		\end{enumerate}
	\end{definition}

	\begin{remark}
		$\varphi \sim \psi \quad\Leftrightarrow\quad \vdash (\varphi \leftrightarrow \psi)$
	\end{remark}

	\begin{remark}
		Für $\mathcal{L} \subseteq \mathcal{L'}$ und eine $\mathcal{L}$-Formel $\varphi$ gilt: $\vdash_\mathcal{L} \varphi \Rightarrow \vdash_\mathcal{L'} \varphi$
	\end{remark}

	\begin{theorem}
		Jede $\mathcal{L}$-Formel $\varphi$ ist äquivalent zu einer $\mathcal{L}$-Formel in der folgenden Form:
		\begin{equation*}
			\underbrace{Q_1 v_{i_1} \dots Q_n v_{i_n}}_\text{pränexe Normalform} \underbrace{\bigvee_{j \in J} \bigwedge_{k \in K_j} (\lnot) \varphi_1{i,j}}_\text{disjunktive Normalform}
		\end{equation*}
		mit $Q_i \in \{ \exists, \forall \}$.
	\end{theorem}

	\subsection{Theorien}
	
	\begin{definition}
		\begin{enumerate}
			\item Eine $\mathcal{L}$-Theorie $T$ ist eine Menge von $\mathcal{L}$-Aussagen.
			\item Eine Struktur $\mathfrak{A}$ ist Modell einer Theorie $T$, $\mathfrak{A} \models T$, falls $\mathfrak{A} \models \varphi$ für jedes $\varphi \in T$..
			\item $\operatorname{Mod}(T) = \{ \mathfrak{A} \text{ $\mathcal{L}$-Struktur} \mid \mathfrak{A} \models T \}$ heißt Modellklasse von $T$. \\
			\heading{Achtung:} $\operatorname{Mod}(T)$ ist im Allgemeinen keine Menge!
			\item $T$ ist konsistent (bzw. Widerspruchsfrei) falls $T$ mindestens ein Modell hat (d.h. $\operatorname{Mod}(T) \neq \emptyset$).
			\item Eine Klasse $\mathcal{K}$ von $\mathcal{L}$-Strukturen heißt elementar, falls es eine Theorie $T$ gibt mit $\operatorname{Mod}(T) = \mathcal{K}$.
			\item Sei $\mathfrak{A}$ $\mathcal{L}$-Struktur. Dann ist
			\begin{equation*}
				\operatorname{Th}(\mathfrak{A}) \coloneqq \{ \text{$\varphi$ $\mathcal{L}$-Aussage} \mid \mathfrak{A} \models \varphi \}
			\end{equation*}
			die vollständige Theorie von $\mathfrak{A}$.
			\item Zwei $\mathcal{L}$-Strukturen $\mathfrak{A}, \mathfrak{B}$ heißen elementar äquivalent, $\mathfrak{A} \equiv \mathfrak{B}$, falls $\operatorname{Th}(\mathfrak{A}) = \operatorname{Th}(\mathfrak{B})$.
		\end{enumerate}
	\end{definition}

	\begin{example}""
		\begin{enumerate}
			\item Wenn $\mathfrak{A}$ endlich ist und $\mathfrak{B} \equiv \mathfrak{A}$, dann ist $\mathfrak{B}$ bereits isomorph zu $\mathfrak{A}$.
			\item $(\setQ, +, -, \cdot, 0, 1) \not\equiv (\setR, +, - , \cdot, 0, 1)$, da
			\begin{align*}
				(\setQ, +, -, \cdot, 0, 1) &\not\models \exists v_0 (v_0 \cdot v_0 = 1 + 1) \\(\setR, +, -, \cdot, 0, 1) &\models \exists v_0 (v_0 \cdot v_0 = 1 + 1)
			\end{align*}
			\item $(\overline{\setQ} \cap \setR, +, -, \cdot, 0, 1) \equiv (\setR, +, -, \cdot, 0, 1)$ mit $\overline{\setQ} = \{ c \in \setC \mid \text{ es gibt ein $P \in \setQ[X]$ so dass $P(c) = 0$} \}$ (algebraischer Abschluss von $\setQ$) (Beweis dazu ist nicht trivial)
		\end{enumerate}
	\end{example}

	\begin{definition}
		Seien $T, T'$ $\mathcal{L}$-Theorien, $\varphi$ $\mathcal{L}$-Aussage
		\begin{enumerate}
			\item $T \vdash \varphi$, falls gilt
			\begin{equation*}
				\mathfrak{A} \models T \quad\Longrightarrow\quad \mathfrak{A} \models \varphi
			\end{equation*}
			für alle $\mathfrak{A}$.
			\item $T^\vdash \coloneqq \{ \varphi \text{ $\mathcal{L}$-Aussage}n \mid T \vdash \varphi \}$ heißt der deduktive Abschluss von $T$.
			\item $T$ ist deduktiv abgeschlossen $:\Leftrightarrow T = T^\vdash$.
			\item $T$ und $T'$ heißen äquivalent $T \equiv T'$ falls $T^\vdash = T'^\vdash$.
		\end{enumerate}
	\end{definition}

	\begin{remark}""
		\begin{itemize}
			\item $T \subseteq T^\vdash = {T^\vdash}^\vdash$
			\item $\mathfrak{A} \models T \Rightarrow \mathfrak{A} \models T^\vdash$ beziehungsweise $\operatorname{Mod}(T) = \operatorname{Mod}(T^\vdash)$
			\item $T^\vdash$ ist die maximale Theorie $T' \supseteq T$ mit der Eigenschaft $\operatorname{Mod}(T) = \operatorname{Mod}(T^\vdash)$
		\end{itemize}
	\end{remark}

	\begin{remark}
		Wenn $\mathfrak{A} \models \varphi$ und $\varphi' \sim \varphi$, dann gilt $\mathfrak{A} \models \varphi'$.
		
		Daher unterscheidet man ab sofort logisch äquivalente Formeln nicht mehr.
		
		Formal: definiere $\mathfrak{A} \models \varphi/\sim$ für Äquivalenzklassen $[\varphi] = \varphi/\sim = \{ \varphi' \mid \varphi \sim \varphi' \}$
	\end{remark}

	\begin{theorem}[Tarski-Lindenbaum-Algebren]
		Die $\mathcal{L}$-Formeln bis auf logische Äquivalenz bilden eine boolesche Algebra $\mathcal{F}_\infty(\mathcal{L})$. Die Formeln deren freie Variablen in $\{v_0, \dots, v_{n-1}\}$ enthalten sind bilden eine boolesche Algebra $\mathcal{F}_n(\mathcal{L})$ das bedeutet:
		
		$\mathcal{F}_i(\mathcal{L})$ ist eine partielle Ordnung $[\varphi] \leq [\psi]$ falls $\vdash (\varphi \rightarrow \psi)$ mit
		\begin{itemize}
			\item einem maximalen Element $[\top]$
			\item einem minimalen Element $[\bot]$
			\item je zwei Elemente $[\varphi], [\psi]$ haben
			\begin{itemize}
				\item ein Supremum $[(\varphi \lor \psi)]$
				\item ein Infimum $[(\varphi \land \psi)]$
			\end{itemize}
			\item jedes Element $[\varphi]$ hat ein Komplement $\lnot \varphi$ das heißt
			\begin{itemize}
				\item $[(\varphi \land \lnot \varphi)] = [\bot]$ und
				\item $[(\varphi \lor \lnot \varphi)] = [\top]$
			\end{itemize}
		\end{itemize}
	
		Die Boolesche Algebra ist dann die Struktur $(\mathcal{F}_i(\mathcal{L}), \land, \lor, \lnot, \top, \bot)$ wobei $[\varphi] \land [\psi] = [(\varphi \land \psi)]$ etc.
	\end{theorem}

	\begin{definition}
		Wenn $\mathfrak{B} = (B, \cap, \cup {}^C, 0, 1)$ beziehungsweise $(B, \subseteq)$ eine Boolesche Algebra ist, dann ist
		\begin{equation*}
			\mathfrak{B}^* = (B, \cup, \cap, {}^C, 1, 0) \text{ beziehungsweise } (B, \supseteq)
		\end{equation*}
		ebenfalls eine Boolesche Algebra, die duale Algebra und
		\begin{equation*}
			\mathfrak{B} \to \mathfrak{B}^*, b \mapsto b^C
		\end{equation*}
		ist Isomorphismus Boolescher Algebren. Insbesondere gilt
		\begin{align*}
			(a \cup b)^C &= a^C \cap b^C \\
			(a \cap b)^C &= a^C \cup b^C
		\end{align*}
	\end{definition}

	\begin{theorem}[Stonescher Repräsentationssatz]
		Jede Boolesche Algebra ist Unteralgebra einer Potenzmengenalgebra.
	\end{theorem}

	\begin{remark}
		$\varphi \vdash \psi$ ist partielle Ordnung auf den Äquivalenzklassen $[\varphi]$.
		\begin{itemize}
			\item reflexiv: $\varphi \vdash \varphi$
			\item transitiv: $\varphi \vdash \psi, \psi \vdash \chi \Rightarrow \varphi \vdash \chi$
			\item antisymmetrisch: $\varphi \vdash \psi, \psi \vdash \varphi \Rightarrow \varphi \sim \psi$
		\end{itemize}
	\end{remark}

	\begin{definition}[Filter]
		Ein Filter in einer Booleschen Algebra $\mathfrak{B}$ ist eine Teilmenge $F \subseteq B$ mit
		\begin{itemize}
			\item $1 \in F, 0 \notin F$
			\item Wenn $b \in F, b \subseteq b'$ dann $b' \in F$
			\item Wenn $b_1, b_2 \in F$, dann auch $b_1 \cap b_2 \in F$
		\end{itemize}
	\end{definition}

	\begin{remark}
		Das duale Konzept heißt Ideal.
	\end{remark}

	\begin{example}""
		\begin{itemize}
			\item Wenn $0 \neq b \in B$, dann ist
			\begin{equation*}
				\langle b \rangle \coloneqq \{ b^i \in B \mid b \subseteq b' \}
			\end{equation*}
			ein Filter, der von $b$ erzeugt Hauptfilter.
			\item $\mathfrak{P}(\setN) = \operatorname{Pot}(\setN)$ der Frechet-Filter ist
			\begin{equation*}
				\{ X \subseteq \setN \mid \setN \setminus X \text{ endlich} \}
			\end{equation*}
			\item Sei $T$ eine konsistente $\mathcal{L}$-Theorie, dann ist $T^\vdash$ ein Filter in $\mathcal{F}_0(\mathcal{L})$ der von $T$ erzeugte Filter.
		\end{itemize}
	\end{example}

	\begin{remark}
		\begin{align*}
			\text{$T$ ist inkonsistent} &\Longleftrightarrow \bot \in T^\vdash \\
			&\Longleftrightarrow \text{alle $\varphi \in \mathcal{F}_0(\mathcal{L})$ liegen in $T^\vdash$} \\
			&\Longleftrightarrow \text{es gibt ein $\varphi \in \mathcal{F}_0(\mathcal{L})$ mit $T \vdash \varphi$ und $T \vdash \lnot \varphi$}
		\end{align*}
	\end{remark}

	\begin{definition}
		\begin{enumerate}
			\item Eine $\mathcal{L}$-Theorie $T$ heißt vollständig, falls für jede $\varphi \in \mathcal{F}_0(\mathcal{L})$ entweder $T \vdash \varphi$ oder $T \vdash \lnot \varphi$ (insbesondere sind vollständige Theorien konsistent)
			\item Ein Filter in einer Booleschen Algebra $\mathfrak{B}$ heißt Ultrafilter, falls $F$ Filter ist und für alle $b \in B$ gilt entweder $b \in F$ oder $b^C \in F$.
		\end{enumerate}
	\end{definition}

	\begin{remark}
		\begin{enumerate}
			\item $T$ ist vollständig $\Leftrightarrow$ $T^\vdash$ ist Ultrafilter in $\mathcal{F}_0(\mathcal{L})$
			\item $\mathfrak{A}$ ist $\mathcal{L}$-Struktur, dann ist $\operatorname{Th}(\mathfrak{A}) = \{ \varphi \in \mathcal{F}_0(\mathcal{L}) \mid \mathfrak{A} \models \varphi \}$ vollständig. Man schreibt auch $\operatorname{Th}(\mathfrak{A}) = \operatorname{Th}(\mathfrak{A})^\vdash$.
		\end{enumerate}
	\end{remark}

	\begin{definition}
		$\mathfrak{A}$ sei eine $\mathcal{L}$-Struktur.
		\begin{enumerate}
			\item Definiere
			\begin{equation*}
				\mathcal{L}_A \coloneqq \mathcal{L} \;\dot{\cup}\; \{ c_a \mid a \in A \}
			\end{equation*}
			$\mathfrak{A}$ wird kanonisch zu einer $\mathcal{L}_A$-Struktur $\mathfrak{A}_A$ expandiert durch
			\begin{equation*}
				c_a^{\mathfrak{A}_A} = a
			\end{equation*}
			\item Das atomare Diagramm von $\mathfrak{A}, \operatorname{Diag}(\mathfrak{A})$ besteht aus allen atomaren und negiert-atomaren $\mathcal{L}_A$-Aussagen, die in $\mathfrak{A}$ gelten
			\begin{equation*}
				\operatorname{Diag}(\mathfrak{A}) = \{ \text{$\varphi$ atomar oder $\varphi = \lnot \psi, \psi$ atomare $\mathcal{L}_A$-Aussage} \mid \mathfrak{A} \models \varphi \}
			\end{equation*}
			Das positive atomare Diagramm ist
			\begin{equation*}
				\operatorname{Diag}^+(\mathfrak{A}) = \{ \text{$\varphi$ atomare $\mathcal{L}_A$-Aussage} \mid \mathfrak{A} \models \varphi \}
			\end{equation*}
		\end{enumerate}
	\end{definition}

	\begin{theorem}
		$h: A \to B$ ist $\mathcal{L}$-Einbettung $\mathfrak{A} \hookrightarrow \mathfrak{B}$ genau dann, wenn $\mathfrak{B}_h \models \operatorname{Diag(\mathfrak{A})}$ wobei $\mathfrak{B}_h = \left(\mathfrak{B}, (h(a))_{a \in A} \right)$.
	\end{theorem}

	\begin{proof}
		$h$ injektiv 
		
		$\Leftrightarrow$ für alle $a \neq a'$ gilt $h(a) \neq h(a')$ 
		
		$\Leftrightarrow$ für alle $a \neq a'$ gilt $\mathfrak{B}_h \models \underbrace{\lnot c_a = c_a'}_{\in \operatorname{Diag}(\mathfrak{A})}$
		
		$h$ starker Homomorphismus
		
		$\Leftrightarrow$ für alle $n$ und $a_1, \dots, a_n$
		\begin{align*}
			&\begin{cases}
				\text{falls $f^\mathfrak{A}(a_1, \dots, a_n) \overset{(\neq)}{=} a$, dann $f^\mathfrak{B}(h(a_1), \dots, h(a_n) \overset{(\neq)}{=} h(a)$} \\
				\text{falls (nicht) $R^\mathfrak{A}(a_1, \dots, a_n)$, dann (nicht) $R^\mathfrak{B}(h(a_1), \dots, h(a_n))$}
			\end{cases} \\
			\Leftrightarrow&\begin{cases}
				\mathfrak{B}_h \models (\lnot) f(c_{a_1}, \dots, c_{a_n}) = c_a \\
				\mathfrak{B}_h \models (\lnot) R(c_{a_1}, \dots, c_{a_n})
			\end{cases}
		\end{align*}
	\end{proof}

	\begin{theorem}
		$h: A \to B$ ist $\mathcal{L}$-Homomorphismus $\mathfrak{A} \to \mathfrak{B}$ $\Leftrightarrow$ $\mathfrak{B}_h \models \operatorname{Diag}^+(\mathfrak{A})$
	\end{theorem}

	\begin{proof}
		Wie eben.
	\end{proof}

	\section{Elementar Unterstrukturen und Kompaktheit}
	
	\subsection{Elementare Unterstrukturen}
	
	\begin{definition}
		Seien $\mathfrak{A}, \mathfrak{B}$ $\mathcal{L}$-Strukturen. \begin{enumerate}
			\item $h: A \to B$ heißt elementare Abbildung, wenn für alle $\mathcal{L}$-Formeln $\varphi = \varphi(v_0, \dots, v_{n-1})$ und $a_0, \dots, a_{n-1} \in A$ gilt:
			
			Wenn $\mathfrak{A} \models \varphi(a_0, \dots, a_{n-1})$, dann $\mathfrak{B} \models \varphi(h(a_0), \dots, h(a_{n-1})$. Durch Betrachten von $\lnot \varphi$ folgt
			\begin{equation*}
				\mathfrak{A} \models \varphi(a_0, \dots, a_{n-1}) \Leftrightarrow \mathfrak{B} \models \varphi(h(a_0), \dots, h(a_{n-1}))
			\end{equation*}
			\item $\mathfrak{A}$ heißt elementare Unterstruktur von $\mathfrak{B}$, $\mathfrak{A} \preccurlyeq \mathfrak{B}$, falls $A \subseteq B$ und $id_A: A \to B$ elementare Abbildung.
		\end{enumerate}
	\end{definition}

	\begin{remark}
		$h: A \to B$ elementar $\Leftrightarrow$ $\mathfrak{B}_h \models \operatorname{Th}(\mathfrak{A}_A) \supseteq \operatorname{Th}(\mathfrak{A}) \cup \operatorname{Diag}(\mathfrak{A})$
		
		Also: Wenn $\mathfrak{A} \preccurlyeq \mathfrak{B}$ dann $\mathfrak{A} \equiv \mathfrak{B}$ und $\mathfrak{A} \subseteq \mathfrak{B}$.
		
		Die Umkehrung gilt nicht!
		
		Aber
		\begin{equation*}
			\mathfrak{A} \preccurlyeq \mathfrak{B} \Leftrightarrow (\mathfrak{A} \subseteq \mathfrak{B} \text{ und } \mathfrak{A} \equiv \mathfrak{B})
		\end{equation*}
	\end{remark}

	\begin{example}
		$(\setN, <) \supseteq (\setN \setminus \{ 0 \}, <)$
		
		$(\setN, <) \cong (\setN \setminus \{ 0 \}, <)$ also $(\setN, <) \equiv (\setN \setminus \{ 0 \}, <)$
		
		Variante 1: Sauber beweisen per Induktion über den Aufbau der Formeln
		
		Variante 2: Ist klar
		
		$(\setN \setminus \{ 0 \}, <) \npreceq (\setN, <)$ da $(\setN \setminus \{ 0 \}, <) \models \lnot \exists x \; x < 1$ aber $(\setN, <) \not\models \exists x \; x < 1$.
	\end{example}

	\begin{example}
		$\mathcal{L} = \{ E \}$ $E$ zweistelliges Relationssymbol, $T = \text{$E$ ist Äquivalenzrelation}$
		
		Falls $\mathfrak{A} \models T$ und $\mathfrak{B} \supseteq \mathfrak{A}$ beliebige Oberstruktur. Dann bleibt Äquivalenz aus $\mathfrak{A}$ in $\mathfrak{B}$ erhalten und umgekehrt, aber es können Äquivalenzklassen in der Oberstruktur dazu kommen und größer werden.
		
		\begin{enumerate}
			\item Wenn eine endliche Zahl von Äquivalenzklassen existieren, dann bleibt die Anzahl in der elementaren Oberstruktur erhalten.
			\item Wenn eine endliche Äquivalenzklasse existiert, dann bleibt deren Größe in der elementaren Oberstruktur erhalten.
			\item Wenn jede Äquivalenzklasse $n$ Elemente hat, dann hat auch in jeder Oberstruktur jede Äquivalenzklasse $n$ Elemente.
			\item Für jedes $n \in N \setminus \{ 0 \}$ gibt es genau eine Äquivalenzklasse mit $n$ Elementen und keine unendliche Klasse. In einer Elementaren Oberstruktur kommen nur unendliche große Äquivalenzklassen dazu.
		\end{enumerate}
	\end{example}

	\begin{theorem}[Tarskis Test]
		Sei $\mathcal{L}$ eine Sprache, und $\mathfrak{B}$ eine $\mathcal{L}$-Struktur, und $A \subseteq B$. Dann ist $A$ genau dann Träger einer elementaren Unterstruktur von $\mathfrak{B}$, wenn für alle $\mathcal{L}_A$-Formeln $\varphi(v_0) \in \mathcal{F}_0(\mathcal{L}_A)$, die in $\mathfrak{B}$ erfüllt sind, gilt dass sie mit einem $a \in A$ erfüllt sind.
		
		Das heißt wenn $\mathfrak{B} \models \exists v_0 \varphi(v_0)$, dann existiert $x \in A$ mit $\mathfrak{B} \models \varphi(a)$.
	\end{theorem}

	\begin{proof}
		\heading{$\Rightarrow$} Angenommen $\mathfrak{A} \preccurlyeq \mathfrak{B} \models \exists v_0 \varphi(v_0)$ (wegen $\preccurlyeq$).
		
		Also existiert $a \in A$ mit $\mathfrak{A} \models \varphi(a)$, somit $\mathfrak{B} \models \varphi(a)$ (wegen $\preccurlyeq$)
		
		\heading{$\Leftarrow$} 
		\begin{enumerate}
			\item $\mathfrak{B} \models \exists v_0 v_0 \doteq v_0$
		
			Also gibt es $a \in A$ mit $\mathfrak{B} \models a \doteq a$ insbesondere $A \neq \emptyset$.
			
			\item Seien $f \in \mathcal{L}$ $n$-stellig, $a_1, \dots, a_n \in A$
			\begin{equation*}
				\mathfrak{B} \models \exists v_0 f a_1 \dots a_n \doteq v_0
			\end{equation*}
			\textit{Bedingung:} es existiert $a \in A$ mit $\mathfrak{B} \models f a_1 \dots a_n \doteq a$.
			
			Also $f^\mathfrak{B}(a_1, \dots, a_n) \in A$, das heißt $A$ ist Träger einer Unterstruktur.
			\item Zeige per Induktion übe den Aufbau der $\mathcal{L}_A$-Formeln
			\begin{equation*}
				\mathfrak{A} \models \varphi \Leftrightarrow \mathfrak{B} \models \varphi
			\end{equation*}
			\begin{itemize}
				\item Induktionsanfang: $\varphi$ Atomar
				\begin{align*}
					\mathfrak{A} \subseteq \mathfrak{B} &\Leftrightarrow id_A: A \to B \text{$\mathcal{L}_A$-Einbettung} \\
					&\Leftrightarrow \mathcal{L}_h \models \operatorname{Diag}(\mathfrak{A}_A) = \operatorname{Diag(\mathfrak{A})} \\
					&\Leftrightarrow \text{für alle atomaren Formeln $\varphi \in \mathcal{F}_0(\mathcal{L}_A)$ gilt: } (\mathfrak{A} \models \varphi \Leftrightarrow \mathfrak{B} \models \varphi)
				\end{align*}
				\item Induktionsschritte
				\begin{equation*}
					\mathfrak{A} \models \lnot \varphi \Leftrightarrow \mathfrak{A} \not\models \varphi \underset{\text{IV}}{\Leftrightarrow} \mathfrak{B} \not\models \varphi \Leftrightarrow \mathfrak{B} \models \lnot \varphi
				\end{equation*}
				\begin{equation*}
					\mathfrak{A} \models (\varphi_1 \land \varphi_2) \Leftrightarrow \begin{cases}
						\mathfrak{A} \models \varphi_1 \\
						\text{und} \\
						\mathfrak{A} \models \varphi_2
					\end{cases} \quad\underset{\text{IV}}{\Longleftrightarrow}\quad \begin{drcases}
						\mathfrak{B} \models \varphi_1 \\
						\text{und} \\
						\mathfrak{B} \models \varphi_2
					\end{drcases} \Leftrightarrow \mathfrak{B} \models (\varphi_1 \land \varphi_2)
				\end{equation*}
				\begin{align*}
					\mathfrak{A} \models \exists v_0 \varphi(v_0) &\Leftrightarrow \text{ex. $a \in A$ mit } \mathfrak{A} \models \varphi(a) \\
					&\underset{\text{IV}}{\Leftrightarrow} \text{ex. $a \in A$ mit } \mathfrak{B} \models \varphi(a) \\&\Rightarrow \text{ex. $a \in B$ mit } \mathfrak{B} \models \varphi(a) \Leftrightarrow \mathfrak{B} \models \exists v_0 \varphi(v_0)
				\end{align*}
			\end{itemize}
			Da $\{ \lnot, \land, \exists \}$ ein vollständiges Junktoren-Quantoren-System bilden ist die Aussage damit gezeigt.
		\end{enumerate}
	\end{proof}

	\begin{corollary}
		\label{cor:elUnterstruktExistiert}
		Sei $\mathfrak{B}$ $\mathcal{L}$-Struktur, $S \subseteq B$. Dann existiert eine elementare Unterstruktur $\mathfrak{A} \preccurlyeq \mathfrak{B}$ mit $S \subseteq A$ und $\abs{A} \leq \max \{ \abs{S}, \abs{\mathcal{L}}, \aleph_0 \}$.
	\end{corollary}

	\begin{proof}
		Definiere induktiv $S_i$ für $i \in \setN$.
		\begin{align*}
			S_0 &\coloneqq S \\
			S_{i+1} &\coloneqq S_i \cup \{ a_\varphi \mid \varphi(x) \text{ $\mathcal{L}_{S_i}$-Formel} \mathfrak{B} \models \exists \varphi(x) \text{ und $a_\varphi$ ist ein Element mit $\mathfrak{B} \models \varphi(a_\varphi)$} \} \\
			S_\omega &\coloneqq \bigcup_{i \in \omega} S_i
		\end{align*}
		Nach Konstruktion ist $S_\omega$ Träger einer elementaren Unterstruktur $\mathfrak{A} \preccurlyeq \mathfrak{B}$.
		
		Denn: Wenn $\mathfrak{B} \models \exists x \varphi(x), \varphi \in \mathcal{F}_1(\mathcal{L}_{S_\omega})$.
		
		Also existiert $n$ mit $\varphi \in \mathcal{F}_1(\mathcal{L}_{S_n})$, dann existiert $a_\varphi \in S_{n+1} \subseteq S_\omega$ mit $\mathfrak{B} \models \varphi(a_\varphi)$. Das heißt Tarskis Test gilt.
		
		\textit{Behauptung:} $\abs{S_\omega} \leq \max \{ \abs{S}, \abs{\mathcal{L}}, \aleph_0 \}$
		
		Per Induktion $\abs{S_i} \leq \max \{ \abs{S}, \abs{\mathcal{L}}, \aleph_0 \}$
		
		\heading{$i=0$}
		\begin{equation*}
			\abs{S_0} = \abs{S} \leq \max \{ \abs{S}, \abs{\mathcal{L}}, \aleph_0 \}
		\end{equation*}
		
		\heading{$i \to i+1$}
		\begin{align*}
			\abs{S_{i+1}} &\leq \abs{S_i} + \underbrace{\abs{\mathcal{F}_1(\mathcal{L}_{S_i})}}_\text{endliche Folgen mit Zeichen aus $Z(S_i)$} \\
			&\leq \abs{S_i} + \abs{Z(S_i)^{< \omega}} \\
			&= \abs{S_i} + \abs{Z(S_i)} \\
			&= \abs{S_i} + \abs{\mathcal{L}} + \aleph_0 + \abs{S_i} \\
			&= \abs{\mathcal{L}} + \abs{S_i} + \aleph_0 \\
			&\overset{\text{IV}}{\leq} \abs{\mathcal{L}} + \max \{ \abs{\mathcal{L}}, \abs{S}, \aleph_0 \} + \aleph_0 \\
			&= \max \{ \abs{L}, \abs{S}, \aleph_0 \}
		\end{align*}
		wobei
		\begin{equation*}
			Z(S_i) = \mathcal{L} \cup \{ v_0, v_1, \dots \} \cup \{ \lnot, \lor, \land, \exists, \forall \} \cup S_i
		\end{equation*}
	\end{proof}

	\begin{remark}
		Für $\abs{\mathcal{L}} = \abs{S} = \aleph_0$ heißt die Folgerung auch Satz von Löwenheim.
	\end{remark}

	Sei $\mathfrak{A}_0 \subseteq \mathfrak{A}_1 \subseteq \mathfrak{A}_2 \subseteq \dots$ eine gerichtete Vereinigung.
	
	Es gibt eine eindeutig bestimmte $\mathcal{L}$-Struktur $\mathfrak{A}_\omega$ auf $\bigcup_{i \in \omega} A_i$, so dass $\mathfrak{A}_i \subseteq \mathfrak{A}_\omega$ für alle $i$.
	
	\begin{theorem}
		Falls $\mathfrak{A}_0 \preccurlyeq \mathfrak{A}_1 \preccurlyeq \mathfrak{A}_2 \preccurlyeq \dots$ dann gilt $\mathfrak{A}_i \preccurlyeq \mathfrak{A}_\omega$ für alle $i$.
	\end{theorem}

	\begin{proof}
		Induktion über den Aufbau der Formeln: $\mathfrak{A}_i \models \varphi \Leftrightarrow \mathfrak{A}_\omega \models \varphi$ für $\varphi \in \mathcal{F}_0(\mathcal{L}_{A_i})$
		
		\heading{Atomar:} da $\mathfrak{A}_i \subseteq \mathfrak{A}_\omega$
		
		\heading{Negation und Konjunktion:} wie letztes Mal
		
		\heading{Existenzquantor:} $\mathfrak{A}_i \models \exists x \varphi(x)$ dann $\mathfrak{A}_i \models \varphi(a)$ für ein $a \in A_i$.
		
		$\overset{\text{IV}}{\Rightarrow} \mathfrak{A}_\omega \models \varphi(a)$ also $\mathfrak{A}_\omega \models \exists x \varphi(x)$.
		
		$\mathfrak{A}_\omega \models \exists x \varphi(x)$, dann $\mathfrak{A}_\omega \models \varphi(a)$ für ein $a \in A_\omega$. Das heißt ex existiert $n \geq i$ mit $a \in A_n$.
		
		Also gilt $\mathfrak{A}_n \models \varphi(a)$ und somit
		\begin{equation*}
			\mathfrak{A}_i \preccurlyeq \mathfrak{A}_n \models \exists x \varphi(x) \Rightarrow \mathfrak{A}_i \models \exists x \varphi()x
		\end{equation*}
	\end{proof}

	\subsection{Kompaktheitssatz und Ultraprodukte}
	
	\begin{theorem}[Kompaktheitssatz]
		Sei $\mathcal{L}$ eine Sprache und $T$ eine $\mathcal{L}$-Theorie.
		
		$T$ hat genau dann ein Modell, wenn jede endliche Teiltheorie $T_0 \subseteq T$ ein Modell hat.
	\end{theorem}

	\begin{corollary}[Satz von Löwenheim-Skolem-Tarski aufwärts]
		Sei $\mathcal{L}$ eine Sprache und $\mathfrak{A}$ eine unendliche $\mathcal{L}$-Struktur.
		Dann existiert zu jeder Kardinalzahl $\kappa \geq \max \{ \abs{A}, \abs{\mathcal{L}} \}$ ein $\mathfrak{B} \succcurlyeq \mathfrak{A}$ mit $\abs{B} = \kappa$.
	\end{corollary}

	\begin{proof}
		Betrachte $\mathcal{L}^c \coloneqq \mathcal{L}_A \;\dot{\cup}\; \{ c_i \mid i < \kappa \}$
		
		und die $\mathcal{L}^C$-Theorie $T^c \coloneqq \operatorname{Th}(\mathfrak{A}_A) \;\cup\; \{ \lnot c_i \doteq c_j \mid i \neq j \}$
		
		Zeige mit dem Kompaktheitssatz: $T^c$ ist konsistent.
		
		Sei $T_0 \subseteq_\text{endl} T^c$.
		
		Dann $T_0 \subseteq \operatorname{Th}(\mathfrak{A}) \cup \{ \lnot c_i \doteq c_j \mid i,j \in \text{ endlicher Menge} \}$.
		
		$\mathfrak{A}$ wird Modell von $T_0$, indem man die endlich vielen Konstanten in $T_0$ durch beliebige, paarweise verschiedene Elemente von $A$ interpretiert.
		
		Sei $\mathcal{L}' \models T^c$.
		
		Dann ist $\underbrace{\mathcal{L}' \upharpoonright_\mathcal{L}}_\text{Redukt auf $\mathcal{L}$} \succcurlyeq \mathfrak{A}$ und $\abs{B'} \geq \kappa$.
		
		Wähle Teilmenge $S \subseteq B$, die $A$ enthält und so, dass $\abs{S} = \kappa$. Wende Folgerung \ref{cor:elUnterstruktExistiert} auf $\mathfrak{B}'_A$ an.
		
		Dann erhält man $\mathfrak{B} \preccurlyeq \mathfrak{B}'_A$ in $\mathcal{L}_A$ mit $\abs{B} \geq \abs{S} = \kappa$ und $\abs{B} \leq \max \{ \abs{\mathcal{L}_A}, \abs{S}, \aleph_0 \} = \kappa$
		
		Und
		\begin{equation*}
			\begin{drcases}
				\mathfrak{A} \preccurlyeq \mathfrak{B}' \text{ in $\mathcal{L}_A$} \\
				\mathfrak{B} \preccurlyeq \mathfrak{B}' \text{ in $\mathcal{L}_A$} \\
				A \subseteq B'
			\end{drcases} \Rightarrow \mathfrak{A} \preccurlyeq \mathfrak{B}
		\end{equation*}
	\end{proof}

	\heading{Ultraprodukte}

	Seien $\mathfrak{A}_i$ $\mathcal{L}$-Strukturen $(i \in I)$ und sei
	\begin{equation*}
		\prod_{i \in I} A_i = \{ p: I \to \bigcup_{i \in I} A_i \mid p(i) \in A_i \}
	\end{equation*}
	Mit dem Auswahlaxiom gilt:
	\begin{equation*}
		A_i \neq \emptyset \text{ für alle $i \in I$} \Rightarrow \prod_{i \in I} A_i \neq \emptyset
	\end{equation*}
	
	Definiere $\mathcal{L}$-Struktur $\prod_{i \in I} \mathfrak{A}_i$ auf $\prod_{i \in I} A_i$.
	
	\begin{align*}
		f^\mathfrak{A}(p_1, \dots, p_n) = p \quad&\Leftrightarrow\quad \text{für alle $i \in I$ } p(i) =  f^{\mathfrak{A}_i}(p_1(i), \dots, p_n(i) \\
		(p_1, \dots, p_n) \in R^\mathfrak{A} \quad&\Leftrightarrow\quad \text{für alle $i \in I$ } (p_1(i), \dots, p_n(i)) \in R^{\mathfrak{A}_i}
	\end{align*}
	
	Betrachte Ultrafilter $\mathcal{U}$ in $\operatorname{Pot}(I)$ also
	\begin{itemize}
		\item $\mathcal{U} \subsetneq \operatorname{Pot}(I), \emptyset \notin \mathcal{U}$
		\item Wenn $X \in \mathcal{U}, X \subseteq Y$, dann $Y \in \mathcal{U}$
		\item Wenn $X,Y \in \mathcal{U}$, dann $X \cap Y \in \mathcal{U}$
		\item Wenn $X \subseteq I$, dann entweder $X \in \mathcal{U}$ oder $I \setminus X \in \mathcal{U}$.
	\end{itemize}

	Ultrafilter $\mathcal{U}$ definiert eine Art Maß auf $\operatorname{Pot}(I)$
	\begin{equation*}
		\mu_\mathcal{U} = \chi_\mathcal{U}: X \mapsto \begin{cases}
			1 & \text{wenn $X \in \mathcal{U}$} \\
			0 & \text{wenn $X \notin \mathcal{U}$}
		\end{cases}
	\end{equation*}
	$X$ mit $X \in \mathcal{U}$ heißt auch $\mathcal{U}$-groß.
	
	\begin{lemma}
		Ein Ultrafilter $\mathcal{U}$ definiert eine Äquivalenzrelation $\sim_\mathcal{U}$ auf $\prod_{i \in I} A_i$ durch
		\begin{equation*}
			p \sim_\mathcal{U} p' :\Leftrightarrow \{ i \in I \mid p(i) = p'(i) \} \in \mathcal{U}
		\end{equation*}
	\end{lemma}

	\begin{proof}
		\begin{itemize}
			\item Reflexiv: klar, da $I \in \mathcal{U}$
			\item Symmetrie: klar per Definition
			\item Transitivität: $p \sim_\mathcal{U} p' \sim_\mathcal{U} p''$
			\begin{equation*}
				\{ i \mid p(i) = p''(i) \} \supseteq \{ i \mid p(i) = p'(i) \} \cap \{ i \mid p'(i) = p''(i) \} \in \mathcal{U} \cap \mathcal{U} = \mathcal{U}.
			\end{equation*}
		\end{itemize}
	\end{proof}

	\begin{definition}
		Seien $\mathfrak{A}_i (i \in I)$ $\mathcal{L}$-Strukturen, $\mathcal{U}$ ein Ultrafilter auf $I$.
		
		Das Ultraprodukt der $\mathfrak{A}_i$ bezüglich $\mathcal{U}$ ist die $\mathcal{L}$-Struktur
		\begin{equation*}
			\prod_{i \in I} \mathfrak{A}_i / \sim_\mathcal{U}
		\end{equation*}
		mit Träger $\prod_{i \in I} A_i / \sim_\mathcal{U}$ und
		\begin{align*}
			(p_1/\sim_\mathcal{U}, \dots, p_m/\sim_\mathcal{U}) \in R^{\prod_{i \in I} \mathfrak{A}_i / \sim_\mathcal{U}} \quad&:\Leftrightarrow\quad \{ i \mid (p_1(i), \dots, p_n(i)) \in R^\mathfrak{A}_i \} \in \mathcal{U} \\f^{\prod_{i \in I} \mathfrak{A}_i / \sim_\mathcal{U}}(p_1/\sim_\mathcal{U}, \dots, p_m/\sim_\mathcal{U}) = p/\sim_\mathcal{U} \quad&:\Leftrightarrow\quad \{ i \mid f^{\mathfrak{A}_i}(p_1(i), \dots, p_n(i) = p(i) \} \in \mathcal{U}
		\end{align*}
	\end{definition}
	\begin{proof}
		\heading{Wohldefiniertheit}
		
		Seien $p_1 \sim_\mathcal{U} p_1', \dots, p_n \sim_\mathcal{U} p_n'$ zu zeigen ist
		\begin{equation*}
			X \coloneqq \{ i \mid (p_1(i), \dots, p_n(i)) \in R^{\mathfrak{A}_i} \} \in \mathcal{U} \Leftrightarrow \{ i \mid (p_1'(i), \dots, p_n'(i)) \in R^{\mathfrak{A}_i} \} \in \mathcal{U}
		\end{equation*}
		
		Sei $X_j = \{ i \mid p_j(i) = p_j'(i) \} \in \mathcal{U}$.
		
		Falls $X \in \mathcal{U}$ auf $X \cap X_1 \cap \dots \cap X_n \in \mathcal{U}$ gilt
		\begin{equation*}
			\begin{drcases}
				(p_1(i), \dots, p_n(i)) \in R^{\mathfrak{A}_i} \\
				p_1(i) = p_1'(i) \\
				\vdots \\
				p_n(i) = p_n'(i)
			\end{drcases} \Rightarrow (p_1'(i), \dots, p_n'(i)) \in R^{\mathfrak{A}_i}
		\end{equation*}
		
		Analog für Funktionszeichen.
		
		Warum existiert überhaupt solch ein $p_\mathcal{U}$?
		
		Man sieht, dass $f^{\prod_{i \in I} \mathfrak{A}_i} (p_1, \dots, p_n) / \mathcal{U}$ es tut.
		
		Falls $\mathfrak{A}_i = \mathfrak{A}$ für alle $i \in I$ dann heißt $\prod_{i \in I} \mathfrak{A}/\mathcal{U} = \mathfrak{A}^I/\mathcal{U}$ auch Ultrapotenz von $\mathfrak{A}$.
	\end{proof}

	\begin{theorem}[Satz von \L os]
		Sei $\varphi$ eine $\mathcal{L}$-Aussage dann gilt
		\begin{equation*}
			\prod_{i \in I} \mathfrak{A}_i / \mathcal{U} \models \varphi \quad\Leftrightarrow\quad \{ i \mid \mathfrak{A}_i \models \varphi \} \in \mathcal{U}
		\end{equation*}
		Insbesondere 
		\begin{itemize}
			\item falls $\mathfrak{A}_i \models T$ für alle $i$, dann $\prod_{i \in I} \mathfrak{A}_i / \mathcal{U} \models T$
			\item falls $\mathfrak{A}_i \equiv \mathfrak{A}_j$ für alle $i \in I$, dann $\prod \mathfrak{A}_i / \mathcal{U} \equiv \mathfrak{A}_i$
		\end{itemize}
	\end{theorem}

	\begin{corollary}
		\begin{equation*}
		 \delta: \mathfrak{A} \to \mathfrak{A}^I / \mathcal{U}, \quad a \mapsto (a,a, \dots, a,a)/ \mathcal{U}
		\end{equation*}
		ist elementare Einbettung, das heißt
		\begin{equation*}
			\mathfrak{A} \preccurlyeq \mathfrak{A}^I/\mathcal{U}
		\end{equation*}
	\end{corollary}

	\begin{proof} zum Satz von \L os (Skizze)
		
		Induktion über den Aufbau der Formeln
		
		\begin{itemize}
			\item $\varphi$ atomar: Entweder Induktion über den Aufbau der Terme oder betrachte termreduzierte Formeln. Dazu sei $f$ einstellig und $c$ Konstante eine atomare Formel ist auch $f f c \doteq c$, diese ist aber äquivalent zu $\exists x (f c \doteq x  \land f x \doteq c)$. Das heißt ohne Einschränkung kann man nur atomare Formeln der Formen $R \tau_1 \dots \tau_n$ oder $\tau_1 \doteq \tau_2$ oder $f \tau_1 \dots \tau_n \doteq \tau$ betrachten, wobei $\tau_i, \tau$ Konstanten oder Individuenvariablen sind.
			\item Satz von \L os für termreduzierte atomare Formeln ist im Wesentlichen die Definition der $\mathcal{L}$-Struktur auf $\prod A_i / \sim_\mathcal{U}$.
			\item Induktion:
			
			Für und
			\begin{align*}
				&\prod \mathfrak{A}_i / \mathcal{U} \models (\varphi \land \psi) \\
				\Leftrightarrow& \prod \mathfrak{A}_i / \mathcal{U} \models \phi \text{ und } \prod \mathfrak{A}_i / \mathcal{U} \models \psi \\
				\Leftrightarrow& I_\varphi = \{ i \mid \mathfrak{A}_i \models \varphi \} \in \mathcal{U} \text{ und } I_\psi = \{ i \mid \mathfrak{A}_i \models \psi\} \in \mathcal{U} \\
				\Leftrightarrow& \{ i \mid \mathfrak{A}_i \models \varphi \land \psi \} = I_\varphi \cap I_\psi \in \mathcal{U}
			\end{align*}
			
			Für nicht
			\begin{align*}
				&\prod_{i \in I} \mathfrak{A}_i / \mathcal{U} \models \lnot \varphi \\
				\Leftrightarrow& \prod_{i \in I} \mathfrak{A}_i / \mathcal{U} \not\models \varphi \\
				\Leftrightarrow& I_\varphi = \{ i \mid \mathfrak{A}_i \models \varphi \} \notin \mathcal{U} \\
				\overset{\text{Ultra}}{\Leftrightarrow}&  I \setminus I_\varphi \{ i \mid \mathfrak{A}_i \models \lnot \varphi \} \in \mathcal{U}
			\end{align*}
			
			Für Existenz
			\begin{align*}
				&\prod_{i \in I} \mathfrak{A}_i / \mathcal{U} \models \exists x \varphi \\
				\Leftrightarrow& \text{ex existiert $p$ mit } \prod \mathfrak{A}_i/\mathcal{U} \models \varphi(p / \mathcal{U}) \\
				\overset{\text{Ind.}}{\Leftrightarrow}& \text{es existiert $p$ mit } \{ i \mid \mathfrak{A}_i \models \varphi(p(i)) \} \in \mathcal{U} \\
				\Leftrightarrow& \{ i \mid \text{ex $p(i) \in A_i$ mit } \mathfrak{A}_i \models \varphi(p(i)) \} \in \mathcal{U} \\
				\Leftrightarrow& \{ i \mid \mathfrak{A}_i \models \exists x \varphi \} \in \mathcal{U}
			\end{align*}
		\end{itemize}
	\end{proof}

	\begin{remark}
		\begin{itemize}
			\item $\langle i \rangle = \{ X \subseteq I \mid i \in X \}$ Ultrafilter, der von $i$ erzeugte Haupt-Ultrafilter
			\begin{equation*}
				\prod_{i \in I} \mathfrak{A}_i/\langle i \rangle \cong \mathfrak{A}_i
			\end{equation*}
			\item Mit Lemma von Zorn (bzw. AC): Jeder eigentliche Filter kann zu einem Ultrafilter erweitert werden.
		\end{itemize}
	\end{remark}

	\begin{definition}
		Sei $I$ eine unendliche Menge, betrachte Filter der $\omega$-endlichen Mengen
		\begin{equation*}
			\mathcal{F} = \{ X \mid I \setminus X \text{ endlich} \}
		\end{equation*}
		$\mathcal{F}$ kann zu Ultrafilter $\mathcal{U}$ erweitert werden. Solche Ultrafilter heißen freie Ultrafilter. Dies sind die nicht-Haupt-Ultrafilter.
	\end{definition}

	\begin{remark}
		Wenn $\mathfrak{A}$ endlich ist, dann ist $\mathfrak{A}^I/\mathcal{U} \cong \mathfrak{A}$.
		
		Wenn $\mathfrak{A}$ unendlich ist und $\mathcal{U}$ frei ist, dann ist häufig $\mathfrak{A} \precnsim \prod \mathfrak{A}_i / \mathcal{U}$.
		
		Wenn $\abs{A_i} < \abs{A_{i+1}}$ endlich ist und $\mathcal{U}$ frei, dann ist
		\begin{equation*}
			\abs{\prod_{i \in I} \mathfrak{A}_i / \mathcal{U}} = 2^{\aleph_0}
		\end{equation*}
		
		Wenn $\abs{A_i} = \aleph_0$ für alle $i$ und $\mathcal{U}$ frei,
		\begin{equation*}
			\abs{\prod_{i \in I} \mathfrak{A}_i / \mathcal{U}} = 2^{\aleph_0}
		\end{equation*}
	\end{remark}

	\begin{theorem}
		Seien $\mathfrak{A}_i (i \in \setN)$ endliche $\mathcal{L}$-Strukturen. Für jedes $n \in \setN$ sei nur endlich oft $\abs{A_i} \leq n$. Sei $\mathcal{U}$ freier Ultrafilter auf $\setN$. Dann ist
		\begin{equation*}
			\abs{\prod_{i \in I} \mathfrak{A}_i / \mathcal{U}} = 2^{\aleph_0}
		\end{equation*}
	\end{theorem}

	\begin{proof}
		\begin{equation*}
			\abs{\prod_{i \in \setN} A_i} \leq \sup \{ \abs{A_i} \mid i \in \setN \}^{\aleph_0} = \aleph_0^{\aleph_0} = 2^{\aleph_0}
		\end{equation*}
		Damit
		\begin{equation*}
			\abs{\prod A_i / \mathcal{U}} \leq 2^{\aleph_0}
		\end{equation*}
		
		\heading{Für $\geq$:} Ohne Einschränkung sei $\abs{A_i} \leq \abs{A_{i+1}}$ und $\abs{A_i} = \{ 0, \dots, n_i \}$ mit $n_i = \abs{A_i} - 1$.
		
		Für $r,s \in \setR \cap [0,1)$ konstruiere $p_r \in \prod_{i \in I} A_i$ mit $r \neq s$, dann stimmen $p_r$ und $p_s$ nur auf endlich vielen Indizes überein.
		
		\begin{equation*}
			p_r(i) \coloneqq j \Leftrightarrow r \in \left[ \frac{j}{A_i}, \frac{j+1}{A_i} \right)
		\end{equation*}
		
		\begin{equation*}
			\Rightarrow p_r \nsim_\mathcal{U} p_s
		\end{equation*}
	\end{proof}

	\begin{proof} zum Kompaktheitssatz
		
		Sei $T$ eine endlich erfüllbare $\mathcal{L}$-Theorie. Zu zeigen ist $T$ ist konsistent.
		
		Sei $I = \operatorname{Pot}_{<\aleph_0}(T) = \{ T_0 \mid T_0 \subseteq_\text{endl} T \}$.
		
		Für $T_0 \subseteq_\text{endl} T$ d.h. $T_0 \in I$ sei $\langle T_0 \rangle = \{ T_1 \in I \mid T_0 \subseteq T_1 \}$.
		
		Sei weiter $\mathcal{F} = \{ \mathcal{X} \subseteq I \mid \text{ ex. $T_0 \in I$ mit } \langle T_0 \rangle \subseteq \mathcal{X} \}$.
		
		$\mathcal{F}$ ist Filter auf $I$:
		\begin{itemize}
			\item $\emptyset \in \mathcal{F}$
			\item Monotonie: per Definition
			\item $\mathcal{X}_1, \mathcal{X}_2 \in \mathcal{F}$, dann existiert $T_i \subseteq_\text{endl} T$ mit $\langle T_i \rangle \subseteq \mathcal{X}_i$. Dann gilt
			\begin{equation*}
				\langle T_1 \cup T_2 \rangle = \langle T_1 \rangle \cap \langle T_2 \rangle \subseteq \mathcal{X}_1 \cap \mathcal{X}_2
			\end{equation*}
		\end{itemize}
	
		Sei $\mathcal{U}$ ein Ultrafilter, der $\mathcal{F}$ erweitert. Wähle für jedes $T_0 \in I$ ein Modell $\mathfrak{M}_{T_0} \models T_0$ und setze $\mathfrak{M} \coloneq \prod_{T_0 \in I} \mathfrak{M}_{T_0} / \mathcal{U}$.
		
		Mit Satz von \L os: prüfe, dass $\varphi \in T \Rightarrow \mathfrak{M} \models \varphi$.
		
		\begin{equation*}
			\{ T_1 \in I \mid \mathfrak{M}_{T_1} \models \varphi \} \supseteq \{ T1 \in I \mid \varphi \in T_1 \} = \langle \{ \varphi \} \rangle \in \mathcal{F} \subseteq \mathcal{U}
		\end{equation*}
	\end{proof}

	\begin{definition}
		$(X, \mathcal{O})$ heißt topologischer Raum und $\mathcal{O}$ heißt Topologie auf $X$), falls
		\begin{itemize}
			\item $\mathcal{O} \subseteq \operatorname{Pot(X)}$
			\item $\mathcal{O}$ ist abgeschlossen bezüglich endlicher Schnitte und beliebiger Vereinigungen
			\item Insbesondere $\emptyset, X \in \mathcal{O}$
		\end{itemize}
	
		$U \in \mathcal{O}$ heißt offen beziehungsweise offene Menge $A \subseteq X$ mit $X \setminus A \in \mathcal{O}$ heißt offen bzw. offene Menge. $A \subseteq X$ mit $X \setminus A \in \mathcal{O}$ heißt abgeschlossen bzw. abgeschlossene Menge.
	\end{definition}

	\begin{definition}
		$Q \subseteq \operatorname{Pot}(X)$ heißt Basis einer Topologie $\mathcal{O}$, falls $Q$ abgeschlossen ist bezüglich endlicher Schnitte.
		
		Dann ist $\mathcal{O} = \{ \bigcup Q_i \mid Q_i \in Q \} \cup \{ \emptyset, X \}$ eine Topologie, und zwar die kleinste, in der alle Mengen aus $Q$ offen sind.
	\end{definition}

	\begin{definition}
		Eine Abbildung heißt stetig, falls Urbilder offener Mengen wieder offen sind.
	\end{definition}

	Sei $\mathfrak{B}$ eine Boolesche Algebra und $\mathcal{U}_\mathfrak{B}$ die Menge der Ultrafilter in $\mathfrak{B}$. Damit ist $\mathcal{U} \subseteq \operatorname{Pot}(B)$ also $\mathcal{U}_\mathfrak{B} \subseteq \operatorname{Pot}(\operatorname{Pot}(B))$.
	
	Für $a \in B$, definiere
	\begin{equation*}
		[[a]] \coloneqq \{ U \in \mathcal{U}_\mathfrak{B} \mid a \in U \} \subseteq \mathcal{U}_\mathfrak{B}
	\end{equation*}
	
	\begin{theorem}""
		\begin{enumerate}
			\item $[[\cdot]]: \mathfrak{B} \hookrightarrow \operatorname{Pot}(\mathcal{U}_\mathfrak{B})$ ist Einbettung Boolescher Algebren (Teil des Stoneschen Repräsentationssatzes)
			\item $\{ [[a]] \mid a \in B \}$ ist Basis einer Topologie auf $\mathcal{U}_\mathfrak{B}$.
		\end{enumerate}
	\end{theorem}

	\begin{definition}
		$\mathcal{U}_\mathfrak{B}$ heißt auch Stone-Raum $S(\mathfrak{B})$ von $\mathfrak{B}$.
	\end{definition}

	\begin{proof}""
		\begin{enumerate}
			\item
			\begin{itemize}
				\item $[[0]] = \emptyset$, da $0 \notin U$ per Definition
				\item $[[1]] = U_\mathfrak{B}$, da $1 \in U$ für jedes $U$
				\item $[[a \cap b]] = [[a]] \cap [[b]]$ folgt aus den Filtereigenschaften
				\item $[[a \cup b]] = [[a]] \cup [[b]]$ folgt aus de Morgan und dem nächsten Schritt
				\item $[[a^c]] = \{ U \mid a^c \in U \} \overset{\text{ultra}}{=} \{ U \mid a \notin U \} = [[a]]^c$
			\end{itemize}
			Das heißt $[[\cdot]]$ ist Homomorphismus der Booleschen Algebra.
		
			Fehlt noch injektivität: Seien $a \neq b$: Zu zeigen ist, es existiert ein Ultrafilter $U$ der $a$ und $b$ trennt, das heißt $a \in U \Leftrightarrow b \notin U$.
		
			Es gilt $a \nsubseteq b$ oder $b \nsubseteq a$ das heißt $a \cap b^c \neq \emptyset$ oder $a^c \cap b \neq \emptyset$.
		
			Es existiert also ultrafilter $U$ mit $a \cap b^c \in U$ oder $a^c \cap b \in U$.
			
			Falls z.B. $a \cap b^c \in U$, dann ist $a \in U, b^c \in U \Rightarrow b \notin U$.
			\item Wegen $[[a]] \cap [[b]] = [[a \cap b]]$
		\end{enumerate}
	\end{proof}

	\begin{remark}
		Die Basis-offenen Mengen $[[a]]$ sind auch abgeschlossen, da $[[a]]^c = [[a^c]]$.
		
		Mengen die offen und abgeschlossen sind heißen clopen.
		
		Topologische Räume mit einer Basis aus clopen Mengen sind total unzusammenhängend.
	\end{remark}

	\begin{definition}
		Ein topologischer Raum $(X, \mathcal{O})$ heißt kompakt, falls die endliche Überdeckungseigenschaft gilt:
		
		Falls $X = \bigcup_{i \in I} \{ U_i \mid U_i \text{ offen} \}$ dann existiert $I_0 \subseteq_\text{endl} I$ mit $X = \bigcup \{ U_i \mid i \in I_0 \}$
		
		Oder in äquivalenter Formulierung: $\bigcap \{ A_i \mid A_i \text{ abgeschlossen}, i \in I \} = \emptyset$ dann existiert $I_0 \subseteq_\text{endl} I$ mit $\bigcap \{ A_i \mid i \in I_0 \} = \emptyset$.
	\end{definition}

	\begin{theorem}
		Der Stone-Raum ist kompakt.
	\end{theorem}

	\begin{remark}
		Der Kompaktheitssatz ist äquivalent zur Kompaktheit von $S(\mathcal{F}_0(\mathcal{L}))$.
		
		Ohne Einschränkung sei $\bot \notin T$
		\begin{equation*}
			T \text{ inkonsistent} \qquad\Leftrightarrow\qquad \bigcap_{\varphi \in T} [[\varphi]] = \emptyset
		\end{equation*}
		und nach Kompaktheitssatz sagt es gibt endliches $T_0 \subseteq T$ so dass $T_0$ inkonsistent ist.
		
		Und mit der Kompaktheit von $S(\mathcal{F}_0(\mathcal{L}))$ existieren $\varphi_0, \dots, \varphi_n \in T$ mit $[\varphi_0[]] \cap \dots \cap [[\varphi_n]] = \emptyset$.
		
		Wir können im ersten Fall $T_0 = \{ \varphi_0, \dots, \varphi_n \}$ mit $\varphi_i$ aus dem zweiten Teil wählen.
	\end{remark}

\end{document}


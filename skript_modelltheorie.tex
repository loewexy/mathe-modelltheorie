% !TeX spellcheck = de_DE
\documentclass[12pt,parskip=full]{scrartcl}
 
\usepackage[utf8]{inputenc}
\usepackage[T1]{fontenc}
\usepackage{lmodern}
\usepackage[ngerman]{babel}
\usepackage{color}
\usepackage{amsmath,amssymb,amstext,mathtools,amsthm}
\usepackage{subcaption}
\usepackage{float}
\usepackage{stmaryrd}
\usepackage[hidelinks]{hyperref}
\hypersetup{bookmarksnumbered}

\usepackage{tikz}
\usetikzlibrary{positioning}

\usepackage{enumitem}
\setenumerate{label=\arabic*)}

\newcommand{\setN}{\mathbb{N}}
\newcommand{\setZ}{\mathbb{Z}}
\newcommand{\setQ}{\mathbb{Q}}
\newcommand{\setR}{\mathbb{R}}
\newcommand{\setC}{\mathbb{C}}
\newcommand{\setH}{\mathbb{H}}
\newcommand{\setk}{\Bbbk}
\newcommand{\ldot}{\,.\,}
\newcommand{\Forall}{~\forall}
\newcommand{\Exists}{~\exists}

\newcommand{\corners}[1]{{\ulcorner #1 \urcorner }}
\newcommand{\dotminus}{\buildrel\textstyle.\over{\hbox{\vrule height3pt depth0pt width0pt}{\smash-}}}
\newcommand{\abs}[1]{{\left| #1 \right|}}
\newcommand{\dabs}[1]{{\left\lVert #1 \right\rVert}}
\newcommand{\heading}{\underline}
 
\theoremstyle{definition}
\newtheorem{theorem}{Satz}[section]
\newtheorem{corollary}[theorem]{Folgerung}
\newtheorem{proposition}[theorem]{Proposition}
\newtheorem{lemma}[theorem]{Lemma}
\newtheorem{definition}[theorem]{Definition}
\newtheorem{example}[theorem]{Beispiel}
\newtheorem*{axiom}{Axiom}

\theoremstyle{remark}
\newtheorem*{remark}{Bemerkung}

\hfuzz=5pt
 
\title{Skript Modelltheorie}
\author{Lukas Metzger}
\date{\today}
 
\begin{document}
	\maketitle
	
	\setcounter{section}{-1}
	\section{Motivation}
	
	\heading{Aus der Linearen Algebra}
	\begin{itemize}
		\item $K$-Vektorräume, Untervektorräume, Homomorphismen
		\item Gruppen, Untergruppen, Homomorphismen
		\item Ringe, Unterringe, Homomorphismen
		\item Körper, Teilkörper, Homomorphismen
	\end{itemize}

	\heading{Entwicklungsschritte}
	\begin{itemize}
		\item Suche nach allgemeiner Theorie $\Rightarrow$ universelle Algebra.
		\item Modelltheorie (universelle Algebra + Logik)
		\item Kategorientheorie
	\end{itemize}

	\heading{Beispiel von Ax}
	
	Sei $K$ ein Körper, und $P(X) \in K[X]$. $P$ definiert eine Abbildung $\tilde{P}: K \to K$.
	
	$P$ hat die Hopf-Eigenschaft, wenn gilt:
	\begin{center}
		Wenn $\tilde{P}$ injektiv ist, dann ist $\tilde{P}$ surjektiv.
	\end{center}

	Jedes Polynom hat über einem endlichen Körper die Hopf-Eigenschaft.
	
	\heading{Formalisierung der Hopf-Eigenschaft}
	\begin{equation*}
		\forall y \forall z (P(y = P(z) \rightarrow y = z)
	\end{equation*}
	\begin{equation*}
		\forall w \exists v P(v) = w
	\end{equation*}
	
	Für jedes $n$
	\begin{equation*}
		\forall x_0, \dots, x_n \left(\forall y \forall z \left(\sum_{i=0}^{n} x_i y^i = \sum_{i=0}^n x_i z^i \rightarrow y = z\right) \rightarrow \forall w \exists v \sum_{i=0}^n x_i v^i = w\right)
	\end{equation*}
	Logik
	\begin{equation*}
		\underset{\text{log. äquivalent}}{\sim} \forall x_0, \dots \forall x_n \forall w \exists v \exists y \exists z \left( \sum_{i=0}^{n} x_i y^i = \sum_{i=0}^n x_i z^i \right) \rightarrow \sum_{i=0}^n x_i v^i = w
	\end{equation*}
	
	
	\begin{example}
		\begin{equation*}
			\mathbb{F}_{p^n} \underset{\text{erfüllt}}{\models} HE(n) \underset{\forall \exists- \text{Präservation}}{\Rightarrow} \underbrace{\bigcup_{n \in \setN} \mathbb{F}_{p^n}}_{\tilde{\mathbb{F}}_p = \text{ der algebraische Abschluss von } \mathbf{F_p}} \models HE(n)
		\end{equation*}
	\end{example}

	\begin{example}
		Aus dem Kompaktheitssatz folgt: $\setC = \lim\limits_{p \to \infty} \tilde{\mathbb{F}}_p$
	\end{example}

	\section{Grundbegriffe}
	
	\subsection{\texorpdfstring{$\mathcal{L}$-Strukturen}{L-Strukturen}}
	
	\begin{example}
		Der angeordnete Körper der reellen Zahlen $(\setR, \underbrace{+, \cdot}_\text{zweistellig}, \underbrace{-}_\text{einstellig}, \underbrace{0, 1}_\text{konstanten}, \underbrace{<}_\text{zweistellige Relation})$
	\end{example}
	
	\begin{definition}[$\mathcal{L}$-Struktur]
		Sei $\mathcal{L}$ eine Menge von
		\begin{itemize}
			\item Funktionszeichen $f_i \quad (i \in I)$
			\item Relationszeichen $R_j \quad (j \in J)$
		\end{itemize}
		
		Jedes Zeichen hat ein festes $n \in \setN$ als Stelligkeit (arity).
		
		$\mathcal{L}$ heißt Sprache / Signatur / similarity type.
		
		Eine $\mathcal{L}$-Struktur $\mathfrak{A}$ besteht aus
		\begin{itemize}
			\item einer nicht-leeren Menge $A$ (Universum, Träger, Grundmenge)
			\item einer $n$-stellige Funktion $f^\mathfrak{A}: A^n \to A$ für jedes $n$-stellige Funktionszeichen $f \in \mathcal{L}$
			\item einer $n$-stellige Relation $R^\mathfrak{A} \subseteq A^n$ für jedes $n$-stellige Relationszeichen $R \in \mathcal{L}$
		\end{itemize}
	
		\heading{$n = 0$}
		
		$A^0 = \{ \emptyset \}$
		
		$0$-stellige Funktion in $\mathfrak{A}$: $f^\mathfrak{A}: \{ \emptyset \} \to A$ ist eindeutig bestimmt durch $f(\emptyset) \in A$. Daher entsprechen $0$-stellige Funktionen den Konstanten.
		
		$0$-stellige Relationen in $\mathfrak{A}$:
		\begin{equation*}
			R^\mathfrak{A} \subseteq \{ \emptyset \} \begin{cases}
				\text{entweder} & R = \{ \emptyset \} \,\hat{=}\, \text{wahr}\\
				\text{oder} & R = \emptyset \,\hat{=}\, \text{falsch}
			\end{cases}
		\end{equation*}
		Daher entsprechen $0$-stellige Relationszeichen den Aussagenvariablen
	\end{definition}

	\begin{example}
		\begin{enumerate}[label=\alph*)]
			\item Zu jeder Menge $A \neq \emptyset$ und jeder Sprache $\mathcal{L}$ kann ich eine $\mathcal{L}$-Struktur mit Träger $A$ finden!
			\item $\mathcal{L} = \{ R \}$, $R$ 2-stelliges Relationssymbol
			\begin{align*}
				\mathfrak{Q}_1 = (\setQ, <), &\qquad\text{d.h.}\quad R^{\mathfrak{Q}_1} = \{ (q_1, q_2) \in \setQ^2 \mid q_1 < q_2 \} \\
				\mathfrak{Q}_2 = (\setQ, <), &\qquad\text{d.h.}\quad R^{\mathfrak{Q}_2} = \{ (q_1, q_2) \in \setQ^2 \mid q_1 < q_2 \}
			\end{align*}
			sind zwei verschiedene $\mathcal{L}$-Strukturen auf $\setQ$.
			\item $\mathcal{L}_{HGr} = \{ \circ \}$ und $\mathcal{L}_Gr = \{ \circ, {}^{-1}, e \}$
			
			Gruppen sind $\mathcal{L}_{Gr}$-Strukturen $\mathfrak{G}$ mit:
			\begin{itemize}
				\item $\circ^\mathfrak{G}$ ist assoziativ
				\item $e^\mathfrak{G} \circ^\mathfrak{G} g = g \circ^\mathfrak{G} e^\mathfrak{G} = g$ für alle $g \in G$
				\item $g \circ ^\mathfrak{G} g^{-1^\mathfrak{G}} = g^{-1^\mathfrak{G}} = e^\mathfrak{G}$
			\end{itemize}
		
			Alternativ sind Gruppen $\mathcal{L}_{HGr}$-Strukturen $\mathfrak{G}$ mit
			\begin{itemize}
				\item $\circ^\mathfrak{G}$ ist assoziativ
				\item es gibt ein neutrales Element
				\item es gibt inverse Elemente
			\end{itemize}
		\end{enumerate}
	\end{example}

	\begin{definition}
		Seien $\mathfrak{A}$ und $\mathfrak{B}$ $\mathcal{L}$-Strukturen. $h: A \to B$ heißt
		\begin{enumerate}[label=\alph*)]
			\item $\mathcal{L}$-Homomorphismus, falls
			\begin{equation*}
				h(f^\mathfrak{A}(a_1, \dots, a_n)) = f^\mathfrak{B}(h(a_1), \dots, h(a_n))
			\end{equation*}
			für alle $n$ und $a_1, \dots, a_n \in A$, und $n$-stellige $f \in \mathcal{L}$ und
			\begin{equation*}
				(a_1, \dots, a_n) \in R^\mathfrak{A} \Rightarrow (h(a_1), \dots, h(a_n)) \in R^\mathfrak{B}
			\end{equation*}
			für alle $n$ und $a_1, \dots, a_n \in A$, und $n$-stellige $R \in \mathcal{L}$.
			\item Starker Homomorphismus, falls zusätzlich $\Leftrightarrow$ im zweiten Teil gilt.
			\item $\mathcal{L}$-Einbettung falls $h$ injektiver starker $\mathcal{L}$-Homomorphismus ist.
			\item $\mathcal{L}$-Isomorphismus falls $h$ bijektiver starker $\mathcal{L}$-Homomorphismus ist und $h^{-1}$ ebenfalls.
			\item $\mathfrak{A}$ und $\mathfrak{B}$ heißen $\mathcal{L}$-Isomorph falls es ein $\mathcal{L}$-Isomorphismus $h: \mathfrak{A} \to \mathfrak{B}$ gibt.
			\item Ein $\mathcal{L}$-Isomorphismus $h: \mathfrak{A} \to \mathfrak{A}$ heißt $\mathcal{L}$-Automorphismus.
			\item Falls $A \subseteq B$, dann heißt $\mathfrak{A}$ $\mathcal{L}$-Unterstruktur von $\mathfrak{B}$ beziehungsweise $\mathfrak{B}$ $\mathcal{L}$-Oberstruktur von $\mathfrak{A}$, falls die Identität $id_A: A \to B$ eine $\mathcal{L}$-Einbettung ist.
		\end{enumerate}
	\end{definition}

	\begin{remark}
		Falls $\mathcal{L}' \subseteq \mathcal{L}$, dann wird jede $\mathcal{L}$-Struktur $\mathfrak{A}$ durch vergessen zu einer $\mathcal{L}'$-Struktur $\mathfrak{A}_{\upharpoonright \mathcal{L}'}$ (Redukt von $\mathfrak{A}$).
	\end{remark}

	\begin{remark}
		Jeder Halbgruppenhomomorphismus zwischen Gruppen ist ein Gruppenhomomorphismus.
		
		Falls $\mathfrak{G}_1, \mathfrak{G}_2$ $\mathcal{L}_{Gr}$-Strukturen sind und $h: G_1 \to G_2$ $L_{HGr}$ Homomorphismus (genau genommen ${G_1}_{\upharpoonright \mathcal{L}_{HGr}}$ und ${G_2}_{\upharpoonright \mathcal{L}_{HGr}}$) dann ist $h$ automatisch ein $\mathcal{L}_{Gr}$-Homomorphismus.
		
		Dies stimmt nicht für Monoide statt Gruppen.
	\end{remark}

	\begin{remark}""
		\begin{enumerate}
			\item Wenn $h: \mathfrak{A} \to \mathfrak{B}$ ein injektiver Homomorphismus ist (d.h. es existiert Sprache $\mathcal{L}$, die im Hintergrund fest ist, $\mathfrak{A},\mathfrak{B}$ sind $\mathcal{L}$-Strukturen, $h$ ist $\mathcal{L}$-Homomorphismus) dann existiert auf $h(A)$ eine $\mathcal{L}$-Struktur $h(\mathfrak{A})$, so dass $h: \mathfrak{A} \xrightarrow{\sim} h(\mathfrak{A})$, aber $h(\mathfrak{A})$ ist nicht notwendigerweise Unterstruktur von $\mathfrak{B}$.
			\item Der Schnitt von $\mathcal{L}$-Unterstrukturen ist wieder eine $\mathcal{L}$-Unterstruktur.
		\end{enumerate}
	\end{remark}
	
	\begin{corollary}
		Wenn $\mathfrak{A}$ eine $\mathcal{L}$-Struktur und $C \subset A$ ist, dann existiert die von $C$ erzeugte $\mathcal{L}$-Unterstruktur $\langle C \rangle_\mathcal{L} = \langle C \rangle$ das heißt die kleinste Unterstruktur von $\mathfrak{A}$, deren Trägermenge $C$ enthält.
		
		Die Trägermenge von $\langle C \rangle$ erhält man dadurch, dass man $C$ unter den Funktionen $f^\mathfrak{A}$ abschließt.
		
		$R^{\langle C \rangle}$ ist dann $R^\mathfrak{A} \cap \langle C \rangle \times \dots \times \langle C \rangle$
	\end{corollary}

	\subsection{\texorpdfstring{$\mathcal{L}$-Formeln}{L-Formeln}}

	\heading{Verwendete Symbole:}
	\begin{itemize}
		\item Funktions- und Relationszeichen aus $\mathcal{L}$:
		\begin{equation*}
			f_i, R_j, \dots, +, \circ, \leq
		\end{equation*}
		\item Gleichheitszeichen: $\doteq$ (Zieglersche Konvention)
		\item Klammern: $()$
		\item Quantoren: $\forall \quad \exists$
		\item aussagenlogische Junktoren: $\underset{\text{Negation}}{\lnot}, \underset{\text{und}}{\land}, \underset{\text{oder}}{\lor}, \underset{\text{Implikation}}{\rightarrow}, \underset{\text{Äquivalent}}{\leftrightarrow}, \underset{\text{Falsum}}{\bot}, \underset{\text{Verum}}{\top}$
		\item Individuenvariablen: $v_0, v_1, \dots$
	\end{itemize}

	\begin{definition}[$\mathcal{L}$-Terme]
		$\mathcal{L}$-Terme sind:
		\begin{itemize}
			\item Individuenvariablen
			\item Wenn $f$ ein $n$-stelliges Funktionszeichen in $\mathcal{L}$ ist und $\tau_1, \dots, \tau_n$ sind $\mathcal{L}$-Terme dann ist $f \tau_1 \dots \tau_n$ ein $\mathcal{L}$-Term.
		\end{itemize}
	\end{definition}

	\begin{remark}""
		\begin{itemize}
			\item Es gilt die eindeutige Lesbarkeit der Terme
			\item Bei Zeichen wie $+, \cdot$ schreibt man traditionell $v_1 + v_2$ statt $+ v_1 v_2$ muss aber bei Verschachtelungen klammern.
		\end{itemize}
	\end{remark}

	\begin{definition}[Auswertung von Termen in Strukturen]
		Eine Belegung der Individuenvariablen mit Elementen einer Struktur für eine $\mathcal{L}$-Struktur $\mathfrak{A}$ ist eine Abbildung $\beta: \{ v_0, v_1, \dots \} \to A$.
		
		Die Auswertung von einem Term in einer Struktur bezüglich einer Belegung $\tau^\mathfrak{A}[\beta]$ ist induktiv definiert durch:
		\begin{align*}
			v_i^\mathfrak{A}[\beta] &\coloneqq \beta(v_i) \\
			f \tau_1 \dots \tau_n {}^\mathfrak{A} [\beta] &\coloneqq f^\mathfrak{A}(\tau_1^\mathfrak{A}[\beta], \dots, \tau_n^\mathfrak{A}[\beta])
		\end{align*}
	\end{definition}

	\begin{definition}[$\mathcal{L}$-Formeln]
		$\mathcal{L}$-Formeln sind
		\begin{itemize}
			\item $\bot \quad \top$
			\item $\tau_1 \doteq \tau_2$ für $\mathcal{L}$-Terme $\tau_1, \tau_2$
			\item $R \tau_1 \dots \tau_n$ für $\mathcal{L}$-Terme $\tau_1, \dots, \tau_n$ und $n$-stelliges $R \in \mathcal{L}$
		\end{itemize}
	\end{definition}

	\begin{definition}[Auswertung von $\mathcal{L}$-Formeln in Strukturen]""\\
		$\mathfrak{A}$ ist Modell von $\varphi$ unter $\beta$ oder formal $\mathfrak{A} \models \varphi[\beta]$
		\begin{itemize}
			\item stets gilt $\mathfrak{A} \models \top[\beta]$
			\item nie gilt $\mathfrak{A} \models \bot[\beta]$
			\item $\mathfrak{A} \models \corners{\tau_1 \doteq \tau_2}[\beta] \Leftrightarrow \tau_1^\mathfrak{A}[\beta] = \tau_2^\mathfrak{A}[\beta]$
			\item $\mathfrak{A} \models R \tau_1 \dots \tau_n [\beta] \Leftrightarrow (\tau_1^\mathfrak{A}[\beta], \dots, \tau_n^\mathfrak{A}[\beta]) \in R^\mathfrak{A}$
			\item Wenn $\varphi, \varphi_1, \varphi_2$ $\mathcal{L}$-Formeln sind, dann auch
			\begin{align*}
				&\lnot \varphi && \mathfrak{A} \models \lnot \varphi[\beta] \Leftrightarrow \mathfrak{A} \not\models \varphi[\beta] \\
				&(\varphi_1 \land \varphi_2) && \mathfrak{A} \models (\varphi_1 \land \varphi_2) [\beta] \Leftrightarrow \mathfrak{A} \models \varphi_1[\beta] \text{ und } \mathfrak{A} \models \varphi_2[\beta] \\
				&(\varphi_1 \lor \varphi_2) && \mathfrak{A} \models (\varphi_1 \lor \varphi_2) [\beta] \Leftrightarrow \mathfrak{A} \models \varphi_1[\beta] \text{ oder } \mathfrak{A} \models \varphi_2[\beta] \\
				&(\varphi_1 \rightarrow \varphi_2)  && \mathfrak{A} \models (\varphi_1 \rightarrow \varphi_2) [\beta] \Leftrightarrow \text{Wenn } \mathfrak{A} \models \varphi_1[\beta] \text{ dann } \mathfrak{A} \models \varphi_2[\beta] \\
				&(\varphi_1 \leftrightarrow \varphi_2)  && \mathfrak{A} \models (\varphi_1 \leftrightarrow \varphi_2) [\beta] \Leftrightarrow (\mathfrak{A} \models \varphi_1[\beta] \Leftrightarrow \mathfrak{A} \models \varphi_2[\beta]) \\
				&\exists v_i \varphi && \text{Es gibt ein $a \in A$ so dass $\mathfrak{A} \models \varphi\left[\beta \frac{a}{v_i}\right]$}\\
				&\forall v_i \varphi && \text{Für alle $a \in A$ gilt dass $\mathfrak{A} \models \varphi\left[\beta \frac{a}{v_i}\right]$}
			\end{align*}
		\end{itemize}
	\end{definition}
	
	\begin{example}
		$\forall v_0 \underbrace{((\forall v_1 \underbrace{R v_0 v_1}_\text{Wirkungsbereich $\forall v_1$}) \lor R v_1 v_0)}_\text{Wirkungsbereich $\forall v_0$}$
		
		Variablen im Wirkungsbereich eines Quantors heißen gebundene Variablen, alle anderen heißen freie Variablen.
	\end{example}

	\begin{remark}
		$\tau^\mathfrak{A}[\beta]$ beziehungsweise $\mathfrak{A} \models \varphi[\beta]$ hängt nur insofern von $\beta$ ab, als man wissen muss, was $\beta$ mit den freien Variablen macht.
	\end{remark}

	\begin{definition}[$\mathcal{L}$-Aussage]
		Eine $\mathcal{L}$-Aussage ($\mathcal{L}$-Satz, geschlossene Formel) ist eine $\mathcal{L}$-Formel ohne freie Variablen.
	\end{definition}
	
	\begin{theorem}
		Für $\mathcal{L}$-Aussagen $\varphi$ ist $\mathfrak{A} \models \varphi[\beta]$ unabhängig von $\beta$.
		
		Man schreibt:
		\begin{align*}
			\mathfrak{A} & \models \varphi \\
			\mathfrak{A} & \not\models \varphi
		\end{align*}
	\end{theorem}

	\begin{definition}""
		\begin{enumerate}
			\item Eine $\mathcal{L}$-Formel $\varphi$ ist allgemeingültig ($\models \varphi, \vdash \varphi$), falls $\mathfrak{A} \models \varphi[\beta]$ für alle $\mathfrak{A}$ und $\beta$.
			\item $\mathcal{L}$-Formeln $\varphi$ und $\psi$ sind logisch äquivalent ($\varphi \sim \psi$), falls
			\begin{equation*}
				\mathfrak{A} \models \varphi[\beta] \Leftrightarrow \mathfrak{A} \models \psi[\beta]
			\end{equation*}
			für alle $\mathfrak{A}$ und $\beta$.
			\item $\psi$ folgt aus $\phi = \{ \varphi_i \mid i \in I \}$, falls:
			\begin{equation*}
				\mathfrak{A} \models \varphi_i[\beta] \text{ für alle $i \in I$} \quad\Longrightarrow\quad \mathfrak{A} \models \psi[\beta] \text{ für alle $\mathfrak{A}$ und $\beta$}
			\end{equation*}
		\end{enumerate}
	\end{definition}

	\begin{remark}
		$\varphi \sim \psi \quad\Leftrightarrow\quad \vdash (\varphi \leftrightarrow \psi)$
	\end{remark}

	\begin{remark}
		Für $\mathcal{L} \subseteq \mathcal{L'}$ und eine $\mathcal{L}$-Formel $\varphi$ gilt: $\vdash_\mathcal{L} \varphi \Rightarrow \vdash_\mathcal{L'} \varphi$
	\end{remark}

	\begin{theorem}
		Jede $\mathcal{L}$-Formel $\varphi$ ist äquivalent zu einer $\mathcal{L}$-Formel in der folgenden Form:
		\begin{equation*}
			\underbrace{Q_1 v_{i_1} \dots Q_n v_{i_n}}_\text{pränexe Normalform} \underbrace{\bigvee_{j \in J} \bigwedge_{k \in K_j} (\lnot) \varphi_1{i,j}}_\text{disjunktive Normalform}
		\end{equation*}
		mit $Q_i \in \{ \exists, \forall \}$.
	\end{theorem}

	\subsection{Theorien}
	
	\begin{definition}
		\begin{enumerate}
			\item Eine $\mathcal{L}$-Theorie $T$ ist eine Menge von $\mathcal{L}$-Aussagen.
			\item Eine Struktur $\mathfrak{A}$ ist Modell einer Theorie $T$, $\mathfrak{A} \models T$, falls $\mathfrak{A} \models \varphi$ für jedes $\varphi \in T$..
			\item $\operatorname{Mod}(T) = \{ \mathfrak{A} \text{ $\mathcal{L}$-Struktur} \mid \mathfrak{A} \models T \}$ heißt Modellklasse von $T$. \\
			\heading{Achtung:} $\operatorname{Mod}(T)$ ist im Allgemeinen keine Menge!
			\item $T$ ist konsistent (bzw. Widerspruchsfrei) falls $T$ mindestens ein Modell hat (d.h. $\operatorname{Mod}(T) \neq \emptyset$).
			\item Eine Klasse $\mathcal{K}$ von $\mathcal{L}$-Strukturen heißt elementar, falls es eine Theorie $T$ gibt mit $\operatorname{Mod}(T) = \mathcal{K}$.
			\item Sei $\mathfrak{A}$ $\mathcal{L}$-Struktur. Dann ist
			\begin{equation*}
				\operatorname{Th}(\mathfrak{A}) \coloneqq \{ \text{$\varphi$ $\mathcal{L}$-Aussage} \mid \mathfrak{A} \models \varphi \}
			\end{equation*}
			die vollständige Theorie von $\mathfrak{A}$.
			\item Zwei $\mathcal{L}$-Strukturen $\mathfrak{A}, \mathfrak{B}$ heißen elementar äquivalent, $\mathfrak{A} \equiv \mathfrak{B}$, falls $\operatorname{Th}(\mathfrak{A}) = \operatorname{Th}(\mathfrak{B})$.
		\end{enumerate}
	\end{definition}

	\begin{example}""
		\begin{enumerate}
			\item Wenn $\mathfrak{A}$ endlich ist und $\mathfrak{B} \equiv \mathfrak{A}$, dann ist $\mathfrak{B}$ bereits isomorph zu $\mathfrak{A}$.
			\item $(\setQ, +, -, \cdot, 0, 1) \not\equiv (\setR, +, - , \cdot, 0, 1)$, da
			\begin{align*}
				(\setQ, +, -, \cdot, 0, 1) &\not\models \exists v_0 (v_0 \cdot v_0 = 1 + 1) \\(\setR, +, -, \cdot, 0, 1) &\models \exists v_0 (v_0 \cdot v_0 = 1 + 1)
			\end{align*}
			\item $(\overline{\setQ} \cap \setR, +, -, \cdot, 0, 1) \equiv (\setR, +, -, \cdot, 0, 1)$ mit $\overline{\setQ} = \{ c \in \setC \mid \text{ es gibt ein $P \in \setQ[X]$ so dass $P(c) = 0$} \}$ (algebraischer Abschluss von $\setQ$) (Beweis dazu ist nicht trivial)
		\end{enumerate}
	\end{example}

	\begin{definition}
		Seien $T, T'$ $\mathcal{L}$-Theorien, $\varphi$ $\mathcal{L}$-Aussage
		\begin{enumerate}
			\item $T \vdash \varphi$, falls gilt
			\begin{equation*}
				\mathfrak{A} \models T \quad\Longrightarrow\quad \mathfrak{A} \models \varphi
			\end{equation*}
			für alle $\mathfrak{A}$.
			\item $T^\vdash \coloneqq \{ \varphi \text{ $\mathcal{L}$-Aussage}n \mid T \vdash \varphi \}$ heißt der deduktive Abschluss von $T$.
			\item $T$ ist deduktiv abgeschlossen $:\Leftrightarrow T = T^\vdash$.
			\item $T$ und $T'$ heißen äquivalent $T \equiv T'$ falls $T^\vdash = T'^\vdash$.
		\end{enumerate}
	\end{definition}

	\begin{remark}""
		\begin{itemize}
			\item $T \subseteq T^\vdash = {T^\vdash}^\vdash$
			\item $\mathfrak{A} \models T \Rightarrow \mathfrak{A} \models T^\vdash$ beziehungsweise $\operatorname{Mod}(T) = \operatorname{Mod}(T^\vdash)$
			\item $T^\vdash$ ist die maximale Theorie $T' \supseteq T$ mit der Eigenschaft $\operatorname{Mod}(T) = \operatorname{Mod}(T^\vdash)$
		\end{itemize}
	\end{remark}

	\begin{remark}
		Wenn $\mathfrak{A} \models \varphi$ und $\varphi' \sim \varphi$, dann gilt $\mathfrak{A} \models \varphi'$.
		
		Daher unterscheidet man ab sofort logisch äquivalente Formeln nicht mehr.
		
		Formal: definiere $\mathfrak{A} \models \varphi/\sim$ für Äquivalenzklassen $[\varphi] = \varphi/\sim = \{ \varphi' \mid \varphi \sim \varphi' \}$
	\end{remark}

	\begin{theorem}[Tarski-Lindenbaum-Algebren]
		Die $\mathcal{L}$-Formeln bis auf logische Äquivalenz bilden eine boolesche Algebra $\mathcal{F}_\infty(\mathcal{L})$. Die Formeln deren freie Variablen in $\{v_0, \dots, v_{n-1}\}$ enthalten sind bilden eine boolesche Algebra $\mathcal{F}_n(\mathcal{L})$ das bedeutet:
		
		$\mathcal{F}_i(\mathcal{L})$ ist eine partielle Ordnung $[\varphi] \leq [\psi]$ falls $\vdash (\varphi \rightarrow \psi)$ mit
		\begin{itemize}
			\item einem maximalen Element $[\top]$
			\item einem minimalen Element $[\bot]$
			\item je zwei Elemente $[\varphi], [\psi]$ haben
			\begin{itemize}
				\item ein Supremum $[(\varphi \lor \psi)]$
				\item ein Infimum $[(\varphi \land \psi)]$
			\end{itemize}
			\item jedes Element $[\varphi]$ hat ein Komplement $\lnot \varphi$ das heißt
			\begin{itemize}
				\item $[(\varphi \land \lnot \varphi)] = [\bot]$ und
				\item $[(\varphi \lor \lnot \varphi)] = [\top]$
			\end{itemize}
		\end{itemize}
	
		Die Boolesche Algebra ist dann die Struktur $(\mathcal{F}_i(\mathcal{L}), \land, \lor, \lnot, \top, \bot)$ wobei $[\varphi] \land [\psi] = [(\varphi \land \psi)]$ etc.
	\end{theorem}

	\begin{definition}
		Wenn $\mathfrak{B} = (B, \cap, \cup {}^C, 0, 1)$ beziehungsweise $(B, \subseteq)$ eine Boolesche Algebra ist, dann ist
		\begin{equation*}
			\mathfrak{B}^* = (B, \cup, \cap, {}^C, 1, 0) \text{ beziehungsweise } (B, \supseteq)
		\end{equation*}
		ebenfalls eine Boolesche Algebra, die duale Algebra und
		\begin{equation*}
			\mathfrak{B} \to \mathfrak{B}^*, b \mapsto b^C
		\end{equation*}
		ist Isomorphismus Boolescher Algebren. Insbesondere gilt
		\begin{align*}
			(a \cup b)^C &= a^C \cap b^C \\
			(a \cap b)^C &= a^C \cup b^C
		\end{align*}
	\end{definition}

	\begin{theorem}[Stonescher Repräsentationssatz]
		Jede Boolesche Algebra ist Unteralgebra einer Potenzmengenalgebra.
	\end{theorem}

	\begin{remark}
		$\varphi \vdash \psi$ ist partielle Ordnung auf den Äquivalenzklassen $[\varphi]$.
		\begin{itemize}
			\item reflexiv: $\varphi \vdash \varphi$
			\item transitiv: $\varphi \vdash \psi, \psi \vdash \chi \Rightarrow \varphi \vdash \chi$
			\item antisymmetrisch: $\varphi \vdash \psi, \psi \vdash \varphi \Rightarrow \varphi \sim \psi$
		\end{itemize}
	\end{remark}

	\begin{definition}[Filter]
		Ein Filter in einer Booleschen Algebra $\mathfrak{B}$ ist eine Teilmenge $F \subseteq B$ mit
		\begin{itemize}
			\item $1 \in F, 0 \notin F$
			\item Wenn $b \in F, b \subseteq b'$ dann $b' \in F$
			\item Wenn $b_1, b_2 \in F$, dann auch $b_1 \cap b_2 \in F$
		\end{itemize}
	\end{definition}

	\begin{remark}
		Das duale Konzept heißt Ideal.
	\end{remark}

	\begin{example}""
		\begin{itemize}
			\item Wenn $0 \neq b \in B$, dann ist
			\begin{equation*}
				\langle b \rangle \coloneqq \{ b^i \in B \mid b \subseteq b' \}
			\end{equation*}
			ein Filter, der von $b$ erzeugt Hauptfilter.
			\item $\mathfrak{P}(\setN) = \operatorname{Pot}(\setN)$ der Frechet-Filter ist
			\begin{equation*}
				\{ X \subseteq \setN \mid \setN \setminus X \text{ endlich} \}
			\end{equation*}
			\item Sei $T$ eine konsistente $\mathcal{L}$-Theorie, dann ist $T^\vdash$ ein Filter in $\mathcal{F}_0(\mathcal{L})$ der von $T$ erzeugte Filter.
		\end{itemize}
	\end{example}

	\begin{remark}
		\begin{align*}
			\text{$T$ ist inkonsistent} &\Longleftrightarrow \bot \in T^\vdash \\
			&\Longleftrightarrow \text{alle $\varphi \in \mathcal{F}_0(\mathcal{L})$ liegen in $T^\vdash$} \\
			&\Longleftrightarrow \text{es gibt ein $\varphi \in \mathcal{F}_0(\mathcal{L})$ mit $T \vdash \varphi$ und $T \vdash \lnot \varphi$}
		\end{align*}
	\end{remark}

	\begin{definition}
		\begin{enumerate}
			\item Eine $\mathcal{L}$-Theorie $T$ heißt vollständig, falls für jede $\varphi \in \mathcal{F}_0(\mathcal{L})$ entweder $T \vdash \varphi$ oder $T \vdash \lnot \varphi$ (insbesondere sind vollständige Theorien konsistent)
			\item Ein Filter in einer Booleschen Algebra $\mathfrak{B}$ heißt Ultrafilter, falls $F$ Filter ist und für alle $b \in B$ gilt entweder $b \in F$ oder $b^C \in F$.
		\end{enumerate}
	\end{definition}

	\begin{remark}
		\begin{enumerate}
			\item $T$ ist vollständig $\Leftrightarrow$ $T^\vdash$ ist Ultrafilter in $\mathcal{F}_0(\mathcal{L})$
			\item $\mathfrak{A}$ ist $\mathcal{L}$-Struktur, dann ist $\operatorname{Th}(\mathfrak{A}) = \{ \varphi \in \mathcal{F}_0(\mathcal{L}) \mid \mathfrak{A} \models \varphi \}$ vollständig. Man schreibt auch $\operatorname{Th}(\mathfrak{A}) = \operatorname{Th}(\mathfrak{A})^\vdash$.
		\end{enumerate}
	\end{remark}

	\begin{definition}
		$\mathfrak{A}$ sei eine $\mathcal{L}$-Struktur.
		\begin{enumerate}
			\item Definiere
			\begin{equation*}
				\mathcal{L}_A \coloneqq \mathcal{L} \;\dot{\cup}\; \{ c_a \mid a \in A \}
			\end{equation*}
			$\mathfrak{A}$ wird kanonisch zu einer $\mathcal{L}_A$-Struktur $\mathfrak{A}_A$ expandiert durch
			\begin{equation*}
				c_a^{\mathfrak{A}_A} = a
			\end{equation*}
			\item Das atomare Diagramm von $\mathfrak{A}, \operatorname{Diag}(\mathfrak{A})$ besteht aus allen atomaren und negiert-atomaren $\mathcal{L}_A$-Aussagen, die in $\mathfrak{A}$ gelten
			\begin{equation*}
				\operatorname{Diag}(\mathfrak{A}) = \{ \text{$\varphi$ atomar oder $\varphi = \lnot \psi, \psi$ atomare $\mathcal{L}_A$-Aussage} \mid \mathfrak{A} \models \varphi \}
			\end{equation*}
			Das positive atomare Diagramm ist
			\begin{equation*}
				\operatorname{Diag}^+(\mathfrak{A}) = \{ \text{$\varphi$ atomare $\mathcal{L}_A$-Aussage} \mid \mathfrak{A} \models \varphi \}
			\end{equation*}
		\end{enumerate}
	\end{definition}

	\begin{theorem}
		$h: A \to B$ ist $\mathcal{L}$-Einbettung $\mathfrak{A} \hookrightarrow \mathfrak{B}$ genau dann, wenn $\mathfrak{B}_h \models \operatorname{Diag(\mathfrak{A})}$ wobei $\mathfrak{B}_h = \left(\mathfrak{B}, (h(a))_{a \in A} \right)$.
	\end{theorem}

	\begin{proof}
		$h$ injektiv 
		
		$\Leftrightarrow$ für alle $a \neq a'$ gilt $h(a) \neq h(a')$ 
		
		$\Leftrightarrow$ für alle $a \neq a'$ gilt $\mathfrak{B}_h \models \underbrace{\lnot c_a = c_a'}_{\in \operatorname{Diag}(\mathfrak{A})}$
		
		$h$ starker Homomorphismus
		
		$\Leftrightarrow$ für alle $n$ und $a_1, \dots, a_n$
		\begin{align*}
			&\begin{cases}
				\text{falls $f^\mathfrak{A}(a_1, \dots, a_n) \overset{(\neq)}{=} a$, dann $f^\mathfrak{B}(h(a_1), \dots, h(a_n) \overset{(\neq)}{=} h(a)$} \\
				\text{falls (nicht) $R^\mathfrak{A}(a_1, \dots, a_n)$, dann (nicht) $R^\mathfrak{B}(h(a_1), \dots, h(a_n))$}
			\end{cases} \\
			\Leftrightarrow&\begin{cases}
				\mathfrak{B}_h \models (\lnot) f(c_{a_1}, \dots, c_{a_n}) = c_a \\
				\mathfrak{B}_h \models (\lnot) R(c_{a_0}, \dots, c_{a_n})
			\end{cases}
		\end{align*}
	\end{proof}

	\begin{theorem}
		$h: A \to B$ ist $\mathcal{L}$-Homomorphismus $\mathfrak{A} \to \mathfrak{B}$ $\Leftrightarrow$ $\mathfrak{B}_h \models \operatorname{Diag}^+(\mathfrak{A})$
	\end{theorem}

	\begin{proof}
		Wie eben.
	\end{proof}

	\section{Elementar Unterstrukturen und Kompaktheit}
	
	\subsection{Elementare Unterstrukturen}
	
	\begin{definition}
		Seien $\mathfrak{A}, \mathfrak{B}$ $\mathcal{L}$-Strukturen. \begin{enumerate}
			\item $h: A \to B$ heißt elementare Abbildung, wenn für alle $\mathcal{L}$-Formeln $\varphi = \varphi(v_0, \dots, v_{n-1})$ und $a_1, \dots, a_n \in A$ gilt:
			
			Wenn $\mathfrak{A} \models \varphi(a_0, \dots, a_{n-1})$, dann $\mathfrak{B} \models \varphi(h(a_0), \dots, h(a_{n-1})$ durch betrachten von $\lnot \varphi$ folgt
			\begin{equation*}
				\mathfrak{A} \models \varphi(a_0, \dots, a_{n-1}) \Leftrightarrow \mathfrak{B} \models \varphi(h(a_0), \dots, h(a_{n-1}))
			\end{equation*}
			\item $\mathfrak{A}$ heißt elementare Unterstruktur von $\mathfrak{B}$, $\mathfrak{A} \preceq \mathfrak{B}$, falls $A \subseteq B$ und $id_A: A \to B$ elementare  Abbildung.
		\end{enumerate}
	\end{definition}

\end{document}


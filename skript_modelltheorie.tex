% !TeX spellcheck = de_DE
\documentclass[12pt,parskip=full]{scrartcl}
 
\usepackage[utf8]{inputenc}
\usepackage[T1]{fontenc}
\usepackage{lmodern}
\usepackage[ngerman]{babel}
\usepackage{color}
\usepackage{amsmath,amssymb,amstext,mathtools,amsthm}
\usepackage{subcaption}
\usepackage{float}
\usepackage{stmaryrd}
\usepackage[hidelinks]{hyperref}
\hypersetup{bookmarksnumbered}
\usepackage{mathtools}

\usepackage{tikz}
\usetikzlibrary{positioning}
\usetikzlibrary{arrows}

\usepackage{enumitem}
\setenumerate{label=\arabic*)}

\newcommand{\setN}{\mathbb{N}}
\newcommand{\setZ}{\mathbb{Z}}
\newcommand{\setQ}{\mathbb{Q}}
\newcommand{\setR}{\mathbb{R}}
\newcommand{\setC}{\mathbb{C}}
\newcommand{\setH}{\mathbb{H}}
\newcommand{\setk}{\Bbbk}
\newcommand{\ldot}{\,.\,}
\newcommand{\Forall}{~\forall}
\newcommand{\Exists}{~\exists}

\newcommand{\corners}[1]{{\ulcorner #1 \urcorner }}
\newcommand{\dotminus}{\buildrel\textstyle.\over{\hbox{\vrule height3pt depth0pt width0pt}{\smash-}}}
\newcommand{\abs}[1]{{\left| #1 \right|}}
\newcommand{\dabs}[1]{{\left\lVert #1 \right\rVert}}
\newcommand{\heading}{\underline}
 
\theoremstyle{definition}
\newtheorem{theorem}{Satz}[section]
\newtheorem{corollary}[theorem]{Folgerung}
\newtheorem{proposition}[theorem]{Proposition}
\newtheorem{lemma}[theorem]{Lemma}
\newtheorem{definition}[theorem]{Definition}
\newtheorem{example}[theorem]{Beispiel}
\newtheorem{axiom}[theorem]{Axiom}
\newtheorem{remark}[theorem]{Bemerkung}

\hfuzz=5pt
 
\title{Skript Modelltheorie}
\author{Lukas Metzger}
\date{\today}
 
\begin{document}
	\maketitle
	
	\setcounter{section}{-1}
	\section{Motivation}
	
	\heading{Aus der Linearen Algebra}
	\begin{itemize}
		\item $K$-Vektorräume, Untervektorräume, Homomorphismen
		\item Gruppen, Untergruppen, Homomorphismen
		\item Ringe, Unterringe, Homomorphismen
		\item Körper, Teilkörper, Homomorphismen
	\end{itemize}

	\heading{Entwicklungsschritte}
	\begin{itemize}
		\item Suche nach allgemeiner Theorie $\Rightarrow$ universelle Algebra.
		\item Modelltheorie (universelle Algebra + Logik)
		\item Kategorientheorie
	\end{itemize}

	\heading{Beispiel von Ax}
	
	Sei $K$ ein Körper, und $P(X) \in K[X]$. $P$ definiert eine Abbildung $\tilde{P}: K \to K$.
	
	$P$ hat die Hopf-Eigenschaft, wenn gilt:
	\begin{center}
		Wenn $\tilde{P}$ injektiv ist, dann ist $\tilde{P}$ surjektiv.
	\end{center}

	Jedes Polynom hat über einem endlichen Körper die Hopf-Eigenschaft.
	
	\heading{Formalisierung der Hopf-Eigenschaft}
	\begin{equation*}
		\forall y \forall z (P(y = P(z) \rightarrow y = z)
	\end{equation*}
	\begin{equation*}
		\forall w \exists v P(v) = w
	\end{equation*}
	
	Für jedes $n$
	\begin{equation*}
		\forall x_0, \dots, x_n \left(\forall y \forall z \left(\sum_{i=0}^{n} x_i y^i = \sum_{i=0}^n x_i z^i \rightarrow y = z\right) \rightarrow \forall w \exists v \sum_{i=0}^n x_i v^i = w\right)
	\end{equation*}
	Logik
	\begin{equation*}
		\underset{\text{log. äquivalent}}{\sim} \forall x_0, \dots \forall x_n \forall w \exists v \exists y \exists z \left( \sum_{i=0}^{n} x_i y^i = \sum_{i=0}^n x_i z^i \right) \rightarrow \sum_{i=0}^n x_i v^i = w
	\end{equation*}
	
	
	\begin{example}
		\begin{equation*}
			\mathbb{F}_{p^n} \underset{\text{erfüllt}}{\models} HE(n) \underset{\forall \exists- \text{Präservation}}{\Rightarrow} \underbrace{\bigcup_{n \in \setN} \mathbb{F}_{p^n}}_{\tilde{\mathbb{F}}_p = \text{ der algebraische Abschluss von } \mathbf{F_p}} \models HE(n)
		\end{equation*}
	\end{example}

	\begin{example}
		Aus dem Kompaktheitssatz folgt: $\setC = \lim\limits_{p \to \infty} \tilde{\mathbb{F}}_p$
	\end{example}

	\section{Grundbegriffe}
	
	\subsection{\texorpdfstring{$\mathcal{L}$-Strukturen}{L-Strukturen}}
	
	\begin{example}
		Der angeordnete Körper der reellen Zahlen $(\setR, \underbrace{+, \cdot}_\text{zweistellig}, \underbrace{-}_\text{einstellig}, \underbrace{0, 1}_\text{konstanten}, \underbrace{<}_\text{zweistellige Relation})$
	\end{example}
	
	\begin{definition}[$\mathcal{L}$-Struktur]
		Sei $\mathcal{L}$ eine Menge von
		\begin{itemize}
			\item Funktionszeichen $f_i \quad (i \in I)$
			\item Relationszeichen $R_j \quad (j \in J)$
		\end{itemize}
		
		Jedes Zeichen hat ein festes $n \in \setN$ als Stelligkeit (arity).
		
		$\mathcal{L}$ heißt Sprache / Signatur / similarity type.
		
		Eine $\mathcal{L}$-Struktur $\mathfrak{A}$ besteht aus
		\begin{itemize}
			\item einer nicht-leeren Menge $A$ (Universum, Träger, Grundmenge)
			\item einer $n$-stellige Funktion $f^\mathfrak{A}: A^n \to A$ für jedes $n$-stellige Funktionszeichen $f \in \mathcal{L}$
			\item einer $n$-stellige Relation $R^\mathfrak{A} \subseteq A^n$ für jedes $n$-stellige Relationszeichen $R \in \mathcal{L}$
		\end{itemize}
	
		\heading{$n = 0$}
		
		$A^0 = \{ \emptyset \}$
		
		$0$-stellige Funktion in $\mathfrak{A}$: $f^\mathfrak{A}: \{ \emptyset \} \to A$ ist eindeutig bestimmt durch $f(\emptyset) \in A$. Daher entsprechen $0$-stellige Funktionen den Konstanten.
		
		$0$-stellige Relationen in $\mathfrak{A}$:
		\begin{equation*}
			R^\mathfrak{A} \subseteq \{ \emptyset \} \begin{cases}
				\text{entweder} & R = \{ \emptyset \} \,\hat{=}\, \text{wahr}\\
				\text{oder} & R = \emptyset \,\hat{=}\, \text{falsch}
			\end{cases}
		\end{equation*}
		Daher entsprechen $0$-stellige Relationszeichen den Aussagenvariablen
	\end{definition}

	\begin{example}
		\begin{enumerate}[label=\alph*)]
			\item Zu jeder Menge $A \neq \emptyset$ und jeder Sprache $\mathcal{L}$ kann ich eine $\mathcal{L}$-Struktur mit Träger $A$ finden!
			\item $\mathcal{L} = \{ R \}$, $R$ 2-stelliges Relationssymbol
			\begin{align*}
				\mathfrak{Q}_1 = (\setQ, <), &\qquad\text{d.h.}\quad R^{\mathfrak{Q}_1} = \{ (q_1, q_2) \in \setQ^2 \mid q_1 < q_2 \} \\
				\mathfrak{Q}_2 = (\setQ, <), &\qquad\text{d.h.}\quad R^{\mathfrak{Q}_2} = \{ (q_1, q_2) \in \setQ^2 \mid q_1 < q_2 \}
			\end{align*}
			sind zwei verschiedene $\mathcal{L}$-Strukturen auf $\setQ$.
			\item $\mathcal{L}_{HGr} = \{ \circ \}$ und $\mathcal{L}_Gr = \{ \circ, {}^{-1}, e \}$
			
			Gruppen sind $\mathcal{L}_{Gr}$-Strukturen $\mathfrak{G}$ mit:
			\begin{itemize}
				\item $\circ^\mathfrak{G}$ ist assoziativ
				\item $e^\mathfrak{G} \circ^\mathfrak{G} g = g \circ^\mathfrak{G} e^\mathfrak{G} = g$ für alle $g \in G$
				\item $g \circ ^\mathfrak{G} g^{-1^\mathfrak{G}} = g^{-1^\mathfrak{G}} = e^\mathfrak{G}$
			\end{itemize}
		
			Alternativ sind Gruppen $\mathcal{L}_{HGr}$-Strukturen $\mathfrak{G}$ mit
			\begin{itemize}
				\item $\circ^\mathfrak{G}$ ist assoziativ
				\item es gibt ein neutrales Element
				\item es gibt inverse Elemente
			\end{itemize}
		\end{enumerate}
	\end{example}

	\begin{definition}
		Seien $\mathfrak{A}$ und $\mathfrak{B}$ $\mathcal{L}$-Strukturen. $h: A \to B$ heißt
		\begin{enumerate}[label=\alph*)]
			\item $\mathcal{L}$-Homomorphismus, falls
			\begin{equation*}
				h(f^\mathfrak{A}(a_1, \dots, a_n)) = f^\mathfrak{B}(h(a_1), \dots, h(a_n))
			\end{equation*}
			für alle $n$ und $a_1, \dots, a_n \in A$, und $n$-stellige $f \in \mathcal{L}$ und
			\begin{equation*}
				(a_1, \dots, a_n) \in R^\mathfrak{A} \Rightarrow (h(a_1), \dots, h(a_n)) \in R^\mathfrak{B}
			\end{equation*}
			für alle $n$ und $a_1, \dots, a_n \in A$, und $n$-stellige $R \in \mathcal{L}$.
			\item Starker Homomorphismus, falls zusätzlich $\Leftrightarrow$ im zweiten Teil gilt.
			\item $\mathcal{L}$-Einbettung falls $h$ injektiver starker $\mathcal{L}$-Homomorphismus ist.
			\item $\mathcal{L}$-Isomorphismus falls $h$ bijektiver starker $\mathcal{L}$-Homomorphismus ist und $h^{-1}$ ebenfalls.
			\item $\mathfrak{A}$ und $\mathfrak{B}$ heißen $\mathcal{L}$-Isomorph falls es ein $\mathcal{L}$-Isomorphismus $h: \mathfrak{A} \to \mathfrak{B}$ gibt.
			\item Ein $\mathcal{L}$-Isomorphismus $h: \mathfrak{A} \to \mathfrak{A}$ heißt $\mathcal{L}$-Automorphismus.
			\item Falls $A \subseteq B$, dann heißt $\mathfrak{A}$ $\mathcal{L}$-Unterstruktur von $\mathfrak{B}$ beziehungsweise $\mathfrak{B}$ $\mathcal{L}$-Oberstruktur von $\mathfrak{A}$, falls die Identität $id_A: A \to B$ eine $\mathcal{L}$-Einbettung ist.
		\end{enumerate}
	\end{definition}

	\begin{remark}
		Falls $\mathcal{L}' \subseteq \mathcal{L}$, dann wird jede $\mathcal{L}$-Struktur $\mathfrak{A}$ durch vergessen zu einer $\mathcal{L}'$-Struktur $\mathfrak{A}_{\upharpoonright \mathcal{L}'}$ (Redukt von $\mathfrak{A}$).
	\end{remark}

	\begin{remark}
		Jeder Halbgruppenhomomorphismus zwischen Gruppen ist ein Gruppenhomomorphismus.
		
		Falls $\mathfrak{G}_1, \mathfrak{G}_2$ $\mathcal{L}_{Gr}$-Strukturen sind und $h: G_1 \to G_2$ $L_{HGr}$ Homomorphismus (genau genommen ${G_1}_{\upharpoonright \mathcal{L}_{HGr}}$ und ${G_2}_{\upharpoonright \mathcal{L}_{HGr}}$) dann ist $h$ automatisch ein $\mathcal{L}_{Gr}$-Homomorphismus.
		
		Dies stimmt nicht für Monoide statt Gruppen.
	\end{remark}

	\begin{remark}""
		\begin{enumerate}
			\item Wenn $h: \mathfrak{A} \to \mathfrak{B}$ ein injektiver Homomorphismus ist (d.h. es existiert Sprache $\mathcal{L}$, die im Hintergrund fest ist, $\mathfrak{A},\mathfrak{B}$ sind $\mathcal{L}$-Strukturen, $h$ ist $\mathcal{L}$-Homomorphismus) dann existiert auf $h(A)$ eine $\mathcal{L}$-Struktur $h(\mathfrak{A})$, so dass $h: \mathfrak{A} \xrightarrow{\sim} h(\mathfrak{A})$, aber $h(\mathfrak{A})$ ist nicht notwendigerweise Unterstruktur von $\mathfrak{B}$.
			\item Der Schnitt von $\mathcal{L}$-Unterstrukturen ist wieder eine $\mathcal{L}$-Unterstruktur.
		\end{enumerate}
	\end{remark}
	
	\begin{corollary}
		Wenn $\mathfrak{A}$ eine $\mathcal{L}$-Struktur und $C \subset A$ ist, dann existiert die von $C$ erzeugte $\mathcal{L}$-Unterstruktur $\langle C \rangle_\mathcal{L} = \langle C \rangle$ das heißt die kleinste Unterstruktur von $\mathfrak{A}$, deren Trägermenge $C$ enthält.
		
		Die Trägermenge von $\langle C \rangle$ erhält man dadurch, dass man $C$ unter den Funktionen $f^\mathfrak{A}$ abschließt.
		
		$R^{\langle C \rangle}$ ist dann $R^\mathfrak{A} \cap \langle C \rangle \times \dots \times \langle C \rangle$
	\end{corollary}

	\subsection{\texorpdfstring{$\mathcal{L}$-Formeln}{L-Formeln}}

	\heading{Verwendete Symbole:}
	\begin{itemize}
		\item Funktions- und Relationszeichen aus $\mathcal{L}$:
		\begin{equation*}
			f_i, R_j, \dots, +, \circ, \leq
		\end{equation*}
		\item Gleichheitszeichen: $\doteq$ (Zieglersche Konvention)
		\item Klammern: $()$
		\item Quantoren: $\forall \quad \exists$
		\item aussagenlogische Junktoren: $\underset{\text{Negation}}{\lnot}, \underset{\text{und}}{\land}, \underset{\text{oder}}{\lor}, \underset{\text{Implikation}}{\rightarrow}, \underset{\text{Äquivalent}}{\leftrightarrow}, \underset{\text{Falsum}}{\bot}, \underset{\text{Verum}}{\top}$
		\item Individuenvariablen: $v_0, v_1, \dots$
	\end{itemize}

	\begin{definition}[$\mathcal{L}$-Terme]
		$\mathcal{L}$-Terme sind:
		\begin{itemize}
			\item Individuenvariablen
			\item Wenn $f$ ein $n$-stelliges Funktionszeichen in $\mathcal{L}$ ist und $\tau_1, \dots, \tau_n$ sind $\mathcal{L}$-Terme dann ist $f \tau_1 \dots \tau_n$ ein $\mathcal{L}$-Term.
		\end{itemize}
	\end{definition}

	\begin{remark}""
		\begin{itemize}
			\item Es gilt die eindeutige Lesbarkeit der Terme
			\item Bei Zeichen wie $+, \cdot$ schreibt man traditionell $v_1 + v_2$ statt $+ v_1 v_2$ muss aber bei Verschachtelungen klammern.
		\end{itemize}
	\end{remark}

	\begin{definition}[Auswertung von Termen in Strukturen]
		Eine Belegung der Individuenvariablen mit Elementen einer Struktur für eine $\mathcal{L}$-Struktur $\mathfrak{A}$ ist eine Abbildung $\beta: \{ v_0, v_1, \dots \} \to A$.
		
		Die Auswertung von einem Term in einer Struktur bezüglich einer Belegung $\tau^\mathfrak{A}[\beta]$ ist induktiv definiert durch:
		\begin{align*}
			v_i^\mathfrak{A}[\beta] &\coloneqq \beta(v_i) \\
			f \tau_1 \dots \tau_n {}^\mathfrak{A} [\beta] &\coloneqq f^\mathfrak{A}(\tau_1^\mathfrak{A}[\beta], \dots, \tau_n^\mathfrak{A}[\beta])
		\end{align*}
	\end{definition}

	\begin{definition}[$\mathcal{L}$-Formeln]
		$\mathcal{L}$-Formeln sind
		\begin{itemize}
			\item $\bot \quad \top$
			\item $\tau_1 \doteq \tau_2$ für $\mathcal{L}$-Terme $\tau_1, \tau_2$
			\item $R \tau_1 \dots \tau_n$ für $\mathcal{L}$-Terme $\tau_1, \dots, \tau_n$ und $n$-stelliges $R \in \mathcal{L}$
		\end{itemize}
	\end{definition}

	\begin{definition}[Auswertung von $\mathcal{L}$-Formeln in Strukturen]""\\
		$\mathfrak{A}$ ist Modell von $\varphi$ unter $\beta$ oder formal $\mathfrak{A} \models \varphi[\beta]$
		\begin{itemize}
			\item stets gilt $\mathfrak{A} \models \top[\beta]$
			\item nie gilt $\mathfrak{A} \models \bot[\beta]$
			\item $\mathfrak{A} \models \corners{\tau_1 \doteq \tau_2}[\beta] \Leftrightarrow \tau_1^\mathfrak{A}[\beta] = \tau_2^\mathfrak{A}[\beta]$
			\item $\mathfrak{A} \models R \tau_1 \dots \tau_n [\beta] \Leftrightarrow (\tau_1^\mathfrak{A}[\beta], \dots, \tau_n^\mathfrak{A}[\beta]) \in R^\mathfrak{A}$
			\item Wenn $\varphi, \varphi_1, \varphi_2$ $\mathcal{L}$-Formeln sind, dann auch
			\begin{align*}
				&\lnot \varphi && \mathfrak{A} \models \lnot \varphi[\beta] \Leftrightarrow \mathfrak{A} \not\models \varphi[\beta] \\
				&(\varphi_1 \land \varphi_2) && \mathfrak{A} \models (\varphi_1 \land \varphi_2) [\beta] \Leftrightarrow \mathfrak{A} \models \varphi_1[\beta] \text{ und } \mathfrak{A} \models \varphi_2[\beta] \\
				&(\varphi_1 \lor \varphi_2) && \mathfrak{A} \models (\varphi_1 \lor \varphi_2) [\beta] \Leftrightarrow \mathfrak{A} \models \varphi_1[\beta] \text{ oder } \mathfrak{A} \models \varphi_2[\beta] \\
				&(\varphi_1 \rightarrow \varphi_2)  && \mathfrak{A} \models (\varphi_1 \rightarrow \varphi_2) [\beta] \Leftrightarrow \text{Wenn } \mathfrak{A} \models \varphi_1[\beta] \text{ dann } \mathfrak{A} \models \varphi_2[\beta] \\
				&(\varphi_1 \leftrightarrow \varphi_2)  && \mathfrak{A} \models (\varphi_1 \leftrightarrow \varphi_2) [\beta] \Leftrightarrow (\mathfrak{A} \models \varphi_1[\beta] \Leftrightarrow \mathfrak{A} \models \varphi_2[\beta]) \\
				&\exists v_i \varphi && \text{Es gibt ein $a \in A$ so dass $\mathfrak{A} \models \varphi\left[\beta \frac{a}{v_i}\right]$}\\
				&\forall v_i \varphi && \text{Für alle $a \in A$ gilt dass $\mathfrak{A} \models \varphi\left[\beta \frac{a}{v_i}\right]$}
			\end{align*}
		\end{itemize}
	\end{definition}
	
	\begin{example}
		$\forall v_0 \underbrace{((\forall v_1 \underbrace{R v_0 v_1}_\text{Wirkungsbereich $\forall v_1$}) \lor R v_1 v_0)}_\text{Wirkungsbereich $\forall v_0$}$
		
		Variablen im Wirkungsbereich eines Quantors heißen gebundene Variablen, alle anderen heißen freie Variablen.
	\end{example}

	\begin{remark}
		$\tau^\mathfrak{A}[\beta]$ beziehungsweise $\mathfrak{A} \models \varphi[\beta]$ hängt nur insofern von $\beta$ ab, als man wissen muss, was $\beta$ mit den freien Variablen macht.
	\end{remark}

	\begin{definition}[$\mathcal{L}$-Aussage]
		Eine $\mathcal{L}$-Aussage ($\mathcal{L}$-Satz, geschlossene Formel) ist eine $\mathcal{L}$-Formel ohne freie Variablen.
	\end{definition}
	
	\begin{theorem}
		Für $\mathcal{L}$-Aussagen $\varphi$ ist $\mathfrak{A} \models \varphi[\beta]$ unabhängig von $\beta$.
		
		Man schreibt:
		\begin{align*}
			\mathfrak{A} & \models \varphi \\
			\mathfrak{A} & \not\models \varphi
		\end{align*}
	\end{theorem}

	\begin{definition}""
		\begin{enumerate}
			\item Eine $\mathcal{L}$-Formel $\varphi$ ist allgemeingültig ($\models \varphi, \vdash \varphi$), falls $\mathfrak{A} \models \varphi[\beta]$ für alle $\mathfrak{A}$ und $\beta$.
			\item $\mathcal{L}$-Formeln $\varphi$ und $\psi$ sind logisch äquivalent ($\varphi \sim \psi$), falls
			\begin{equation*}
				\mathfrak{A} \models \varphi[\beta] \Leftrightarrow \mathfrak{A} \models \psi[\beta]
			\end{equation*}
			für alle $\mathfrak{A}$ und $\beta$.
			\item $\psi$ folgt aus $\phi = \{ \varphi_i \mid i \in I \}$, falls:
			\begin{equation*}
				\mathfrak{A} \models \varphi_i[\beta] \text{ für alle $i \in I$} \quad\Longrightarrow\quad \mathfrak{A} \models \psi[\beta] \text{ für alle $\mathfrak{A}$ und $\beta$}
			\end{equation*}
		\end{enumerate}
	\end{definition}

	\begin{remark}
		$\varphi \sim \psi \quad\Leftrightarrow\quad \vdash (\varphi \leftrightarrow \psi)$
	\end{remark}

	\begin{remark}
		Für $\mathcal{L} \subseteq \mathcal{L'}$ und eine $\mathcal{L}$-Formel $\varphi$ gilt: $\vdash_\mathcal{L} \varphi \Rightarrow \vdash_\mathcal{L'} \varphi$
	\end{remark}

	\begin{theorem}
		Jede $\mathcal{L}$-Formel $\varphi$ ist äquivalent zu einer $\mathcal{L}$-Formel in der folgenden Form:
		\begin{equation*}
			\underbrace{Q_1 v_{i_1} \dots Q_n v_{i_n}}_\text{pränexe Normalform} \underbrace{\bigvee_{j \in J} \bigwedge_{k \in K_j} (\lnot) \varphi_1{i,j}}_\text{disjunktive Normalform}
		\end{equation*}
		mit $Q_i \in \{ \exists, \forall \}$.
	\end{theorem}

	\subsection{Theorien}
	
	\begin{definition}
		\begin{enumerate}
			\item Eine $\mathcal{L}$-Theorie $T$ ist eine Menge von $\mathcal{L}$-Aussagen.
			\item Eine Struktur $\mathfrak{A}$ ist Modell einer Theorie $T$, $\mathfrak{A} \models T$, falls $\mathfrak{A} \models \varphi$ für jedes $\varphi \in T$..
			\item $\operatorname{Mod}(T) = \{ \mathfrak{A} \text{ $\mathcal{L}$-Struktur} \mid \mathfrak{A} \models T \}$ heißt Modellklasse von $T$. \\
			\heading{Achtung:} $\operatorname{Mod}(T)$ ist im Allgemeinen keine Menge!
			\item $T$ ist konsistent (bzw. Widerspruchsfrei) falls $T$ mindestens ein Modell hat (d.h. $\operatorname{Mod}(T) \neq \emptyset$).
			\item Eine Klasse $\mathcal{K}$ von $\mathcal{L}$-Strukturen heißt elementar, falls es eine Theorie $T$ gibt mit $\operatorname{Mod}(T) = \mathcal{K}$.
			\item Sei $\mathfrak{A}$ $\mathcal{L}$-Struktur. Dann ist
			\begin{equation*}
				\operatorname{Th}(\mathfrak{A}) \coloneqq \{ \text{$\varphi$ $\mathcal{L}$-Aussage} \mid \mathfrak{A} \models \varphi \}
			\end{equation*}
			die vollständige Theorie von $\mathfrak{A}$.
			\item Zwei $\mathcal{L}$-Strukturen $\mathfrak{A}, \mathfrak{B}$ heißen elementar äquivalent, $\mathfrak{A} \equiv \mathfrak{B}$, falls $\operatorname{Th}(\mathfrak{A}) = \operatorname{Th}(\mathfrak{B})$.
		\end{enumerate}
	\end{definition}

	\begin{example}""
		\begin{enumerate}
			\item Wenn $\mathfrak{A}$ endlich ist und $\mathfrak{B} \equiv \mathfrak{A}$, dann ist $\mathfrak{B}$ bereits isomorph zu $\mathfrak{A}$.
			\item $(\setQ, +, -, \cdot, 0, 1) \not\equiv (\setR, +, - , \cdot, 0, 1)$, da
			\begin{align*}
				(\setQ, +, -, \cdot, 0, 1) &\not\models \exists v_0 (v_0 \cdot v_0 = 1 + 1) \\(\setR, +, -, \cdot, 0, 1) &\models \exists v_0 (v_0 \cdot v_0 = 1 + 1)
			\end{align*}
			\item $(\overline{\setQ} \cap \setR, +, -, \cdot, 0, 1) \equiv (\setR, +, -, \cdot, 0, 1)$ mit $\overline{\setQ} = \{ c \in \setC \mid \text{ es gibt ein $P \in \setQ[X]$ so dass $P(c) = 0$} \}$ (algebraischer Abschluss von $\setQ$) (Beweis dazu ist nicht trivial)
		\end{enumerate}
	\end{example}

	\begin{definition}
		Seien $T, T'$ $\mathcal{L}$-Theorien, $\varphi$ $\mathcal{L}$-Aussage
		\begin{enumerate}
			\item $T \vdash \varphi$, falls gilt
			\begin{equation*}
				\mathfrak{A} \models T \quad\Longrightarrow\quad \mathfrak{A} \models \varphi
			\end{equation*}
			für alle $\mathfrak{A}$.
			\item $T^\vdash \coloneqq \{ \varphi \text{ $\mathcal{L}$-Aussage}n \mid T \vdash \varphi \}$ heißt der deduktive Abschluss von $T$.
			\item $T$ ist deduktiv abgeschlossen $:\Leftrightarrow T = T^\vdash$.
			\item $T$ und $T'$ heißen äquivalent $T \equiv T'$ falls $T^\vdash = T'^\vdash$.
		\end{enumerate}
	\end{definition}

	\begin{remark}""
		\begin{itemize}
			\item $T \subseteq T^\vdash = {T^\vdash}^\vdash$
			\item $\mathfrak{A} \models T \Rightarrow \mathfrak{A} \models T^\vdash$ beziehungsweise $\operatorname{Mod}(T) = \operatorname{Mod}(T^\vdash)$
			\item $T^\vdash$ ist die maximale Theorie $T' \supseteq T$ mit der Eigenschaft $\operatorname{Mod}(T) = \operatorname{Mod}(T^\vdash)$
		\end{itemize}
	\end{remark}

	\begin{remark}
		Wenn $\mathfrak{A} \models \varphi$ und $\varphi' \sim \varphi$, dann gilt $\mathfrak{A} \models \varphi'$.
		
		Daher unterscheidet man ab sofort logisch äquivalente Formeln nicht mehr.
		
		Formal: definiere $\mathfrak{A} \models \varphi/\sim$ für Äquivalenzklassen $[\varphi] = \varphi/\sim = \{ \varphi' \mid \varphi \sim \varphi' \}$
	\end{remark}

	\begin{theorem}[Tarski-Lindenbaum-Algebren]
		Die $\mathcal{L}$-Formeln bis auf logische Äquivalenz bilden eine boolesche Algebra $\mathcal{F}_\infty(\mathcal{L})$. Die Formeln deren freie Variablen in $\{v_0, \dots, v_{n-1}\}$ enthalten sind bilden eine boolesche Algebra $\mathcal{F}_n(\mathcal{L})$ das bedeutet:
		
		$\mathcal{F}_i(\mathcal{L})$ ist eine partielle Ordnung $[\varphi] \leq [\psi]$ falls $\vdash (\varphi \rightarrow \psi)$ mit
		\begin{itemize}
			\item einem maximalen Element $[\top]$
			\item einem minimalen Element $[\bot]$
			\item je zwei Elemente $[\varphi], [\psi]$ haben
			\begin{itemize}
				\item ein Supremum $[(\varphi \lor \psi)]$
				\item ein Infimum $[(\varphi \land \psi)]$
			\end{itemize}
			\item jedes Element $[\varphi]$ hat ein Komplement $\lnot \varphi$ das heißt
			\begin{itemize}
				\item $[(\varphi \land \lnot \varphi)] = [\bot]$ und
				\item $[(\varphi \lor \lnot \varphi)] = [\top]$
			\end{itemize}
		\end{itemize}
	
		Die Boolesche Algebra ist dann die Struktur $(\mathcal{F}_i(\mathcal{L}), \land, \lor, \lnot, \top, \bot)$ wobei $[\varphi] \land [\psi] = [(\varphi \land \psi)]$ etc.
	\end{theorem}

	\begin{definition}
		Wenn $\mathfrak{B} = (B, \cap, \cup {}^C, 0, 1)$ beziehungsweise $(B, \subseteq)$ eine Boolesche Algebra ist, dann ist
		\begin{equation*}
			\mathfrak{B}^* = (B, \cup, \cap, {}^C, 1, 0) \text{ beziehungsweise } (B, \supseteq)
		\end{equation*}
		ebenfalls eine Boolesche Algebra, die duale Algebra und
		\begin{equation*}
			\mathfrak{B} \to \mathfrak{B}^*, b \mapsto b^C
		\end{equation*}
		ist Isomorphismus Boolescher Algebren. Insbesondere gilt
		\begin{align*}
			(a \cup b)^C &= a^C \cap b^C \\
			(a \cap b)^C &= a^C \cup b^C
		\end{align*}
	\end{definition}

	\begin{theorem}[Stonescher Repräsentationssatz]
		Jede Boolesche Algebra ist Unteralgebra einer Potenzmengenalgebra.
	\end{theorem}

	\begin{remark}
		$\varphi \vdash \psi$ ist partielle Ordnung auf den Äquivalenzklassen $[\varphi]$.
		\begin{itemize}
			\item reflexiv: $\varphi \vdash \varphi$
			\item transitiv: $\varphi \vdash \psi, \psi \vdash \chi \Rightarrow \varphi \vdash \chi$
			\item antisymmetrisch: $\varphi \vdash \psi, \psi \vdash \varphi \Rightarrow \varphi \sim \psi$
		\end{itemize}
	\end{remark}

	\begin{definition}[Filter]
		Ein Filter in einer Booleschen Algebra $\mathfrak{B}$ ist eine Teilmenge $F \subseteq B$ mit
		\begin{itemize}
			\item $1 \in F, 0 \notin F$
			\item Wenn $b \in F, b \subseteq b'$ dann $b' \in F$
			\item Wenn $b_1, b_2 \in F$, dann auch $b_1 \cap b_2 \in F$
		\end{itemize}
	\end{definition}

	\begin{remark}
		Das duale Konzept heißt Ideal.
	\end{remark}

	\begin{example}""
		\begin{itemize}
			\item Wenn $0 \neq b \in B$, dann ist
			\begin{equation*}
				\langle b \rangle \coloneqq \{ b^i \in B \mid b \subseteq b' \}
			\end{equation*}
			ein Filter, der von $b$ erzeugt Hauptfilter.
			\item $\mathfrak{P}(\setN) = \operatorname{Pot}(\setN)$ der Frechet-Filter ist
			\begin{equation*}
				\{ X \subseteq \setN \mid \setN \setminus X \text{ endlich} \}
			\end{equation*}
			\item Sei $T$ eine konsistente $\mathcal{L}$-Theorie, dann ist $T^\vdash$ ein Filter in $\mathcal{F}_0(\mathcal{L})$ der von $T$ erzeugte Filter.
		\end{itemize}
	\end{example}

	\begin{remark}
		\begin{align*}
			\text{$T$ ist inkonsistent} &\Longleftrightarrow \bot \in T^\vdash \\
			&\Longleftrightarrow \text{alle $\varphi \in \mathcal{F}_0(\mathcal{L})$ liegen in $T^\vdash$} \\
			&\Longleftrightarrow \text{es gibt ein $\varphi \in \mathcal{F}_0(\mathcal{L})$ mit $T \vdash \varphi$ und $T \vdash \lnot \varphi$}
		\end{align*}
	\end{remark}

	\begin{definition}
		\begin{enumerate}
			\item Eine $\mathcal{L}$-Theorie $T$ heißt vollständig, falls für jede $\varphi \in \mathcal{F}_0(\mathcal{L})$ entweder $T \vdash \varphi$ oder $T \vdash \lnot \varphi$ (insbesondere sind vollständige Theorien konsistent)
			\item Ein Filter in einer Booleschen Algebra $\mathfrak{B}$ heißt Ultrafilter, falls $F$ Filter ist und für alle $b \in B$ gilt entweder $b \in F$ oder $b^C \in F$.
		\end{enumerate}
	\end{definition}

	\begin{remark}
		\begin{enumerate}
			\item $T$ ist vollständig $\Leftrightarrow$ $T^\vdash$ ist Ultrafilter in $\mathcal{F}_0(\mathcal{L})$
			\item $\mathfrak{A}$ ist $\mathcal{L}$-Struktur, dann ist $\operatorname{Th}(\mathfrak{A}) = \{ \varphi \in \mathcal{F}_0(\mathcal{L}) \mid \mathfrak{A} \models \varphi \}$ vollständig. Man schreibt auch $\operatorname{Th}(\mathfrak{A}) = \operatorname{Th}(\mathfrak{A})^\vdash$.
		\end{enumerate}
	\end{remark}

	\begin{definition}
		$\mathfrak{A}$ sei eine $\mathcal{L}$-Struktur.
		\begin{enumerate}
			\item Definiere
			\begin{equation*}
				\mathcal{L}_A \coloneqq \mathcal{L} \;\dot{\cup}\; \{ c_a \mid a \in A \}
			\end{equation*}
			$\mathfrak{A}$ wird kanonisch zu einer $\mathcal{L}_A$-Struktur $\mathfrak{A}_A$ expandiert durch
			\begin{equation*}
				c_a^{\mathfrak{A}_A} = a
			\end{equation*}
			\item Das atomare Diagramm von $\mathfrak{A}, \operatorname{Diag}(\mathfrak{A})$ besteht aus allen atomaren und negiert-atomaren $\mathcal{L}_A$-Aussagen, die in $\mathfrak{A}$ gelten
			\begin{equation*}
				\operatorname{Diag}(\mathfrak{A}) = \{ \text{$\varphi$ atomar oder $\varphi = \lnot \psi, \psi$ atomare $\mathcal{L}_A$-Aussage} \mid \mathfrak{A} \models \varphi \}
			\end{equation*}
			Das positive atomare Diagramm ist
			\begin{equation*}
				\operatorname{Diag}^+(\mathfrak{A}) = \{ \text{$\varphi$ atomare $\mathcal{L}_A$-Aussage} \mid \mathfrak{A} \models \varphi \}
			\end{equation*}
			\item Das elementare Diagramm von $\mathfrak{A}$ ist
			\begin{equation*}
				\operatorname{Diag}_\mathfrak{A}(a) = \operatorname{Th}(\mathfrak{A}_A) = \{ \varphi \; \mathcal{L}_A\text{-Aussage} \mid \mathfrak{A} \models \varphi \}
			\end{equation*}
		\end{enumerate}
	\end{definition}

	\begin{theorem}
		$h: A \to B$ ist $\mathcal{L}$-Einbettung $\mathfrak{A} \hookrightarrow \mathfrak{B}$ genau dann, wenn $\mathfrak{B}_h \models \operatorname{Diag(\mathfrak{A})}$ wobei $\mathfrak{B}_h = \left(\mathfrak{B}, (h(a))_{a \in A} \right)$.
	\end{theorem}

	\begin{proof}
		$h$ injektiv 
		
		$\Leftrightarrow$ für alle $a \neq a'$ gilt $h(a) \neq h(a')$ 
		
		$\Leftrightarrow$ für alle $a \neq a'$ gilt $\mathfrak{B}_h \models \underbrace{\lnot c_a = c_a'}_{\in \operatorname{Diag}(\mathfrak{A})}$
		
		$h$ starker Homomorphismus
		
		$\Leftrightarrow$ für alle $n$ und $a_1, \dots, a_n$
		\begin{align*}
			&\begin{cases}
				\text{falls $f^\mathfrak{A}(a_1, \dots, a_n) \overset{(\neq)}{=} a$, dann $f^\mathfrak{B}(h(a_1), \dots, h(a_n) \overset{(\neq)}{=} h(a)$} \\
				\text{falls (nicht) $R^\mathfrak{A}(a_1, \dots, a_n)$, dann (nicht) $R^\mathfrak{B}(h(a_1), \dots, h(a_n))$}
			\end{cases} \\
			\Leftrightarrow&\begin{cases}
				\mathfrak{B}_h \models (\lnot) f(c_{a_1}, \dots, c_{a_n}) = c_a \\
				\mathfrak{B}_h \models (\lnot) R(c_{a_1}, \dots, c_{a_n})
			\end{cases}
		\end{align*}
	\end{proof}

	\begin{theorem}
		$h: A \to B$ ist $\mathcal{L}$-Homomorphismus $\mathfrak{A} \to \mathfrak{B}$ $\Leftrightarrow$ $\mathfrak{B}_h \models \operatorname{Diag}^+(\mathfrak{A})$
	\end{theorem}

	\begin{proof}
		Wie eben.
	\end{proof}

	\section{Elementar Unterstrukturen und Kompaktheit}
	
	\subsection{Elementare Unterstrukturen}
	
	\begin{definition}
		Seien $\mathfrak{A}, \mathfrak{B}$ $\mathcal{L}$-Strukturen. \begin{enumerate}
			\item $h: A \to B$ heißt elementare Abbildung, wenn für alle $\mathcal{L}$-Formeln $\varphi = \varphi(v_0, \dots, v_{n-1})$ und $a_0, \dots, a_{n-1} \in A$ gilt:
			
			Wenn $\mathfrak{A} \models \varphi(a_0, \dots, a_{n-1})$, dann $\mathfrak{B} \models \varphi(h(a_0), \dots, h(a_{n-1})$. Durch Betrachten von $\lnot \varphi$ folgt
			\begin{equation*}
				\mathfrak{A} \models \varphi(a_0, \dots, a_{n-1}) \Leftrightarrow \mathfrak{B} \models \varphi(h(a_0), \dots, h(a_{n-1}))
			\end{equation*}
			\item $\mathfrak{A}$ heißt elementare Unterstruktur von $\mathfrak{B}$, $\mathfrak{A} \preccurlyeq \mathfrak{B}$, falls $A \subseteq B$ und $id_A: A \to B$ elementare Abbildung.
		\end{enumerate}
	\end{definition}

	\begin{remark}
		$h: A \to B$ elementar $\Leftrightarrow$ $\mathfrak{B}_h \models \operatorname{Th}(\mathfrak{A}_A) \supseteq \operatorname{Th}(\mathfrak{A}) \cup \operatorname{Diag}(\mathfrak{A})$
		
		Also: Wenn $\mathfrak{A} \preccurlyeq \mathfrak{B}$ dann $\mathfrak{A} \equiv \mathfrak{B}$ und $\mathfrak{A} \subseteq \mathfrak{B}$.
		
		Die Umkehrung gilt nicht!
		
		Aber
		\begin{equation*}
			\mathfrak{A} \preccurlyeq \mathfrak{B} \Leftrightarrow (\mathfrak{A} \subseteq \mathfrak{B} \text{ und } \mathfrak{A} \equiv \mathfrak{B})
		\end{equation*}
	\end{remark}

	\begin{example}
		$(\setN, <) \supseteq (\setN \setminus \{ 0 \}, <)$
		
		$(\setN, <) \cong (\setN \setminus \{ 0 \}, <)$ also $(\setN, <) \equiv (\setN \setminus \{ 0 \}, <)$
		
		Variante 1: Sauber beweisen per Induktion über den Aufbau der Formeln
		
		Variante 2: Ist klar
		
		$(\setN \setminus \{ 0 \}, <) \npreceq (\setN, <)$ da $(\setN \setminus \{ 0 \}, <) \models \lnot \exists x \; x < 1$ aber $(\setN, <) \not\models \exists x \; x < 1$.
	\end{example}

	\begin{example}
		$\mathcal{L} = \{ E \}$ $E$ zweistelliges Relationssymbol, $T = \text{$E$ ist Äquivalenzrelation}$
		
		Falls $\mathfrak{A} \models T$ und $\mathfrak{B} \supseteq \mathfrak{A}$ beliebige Oberstruktur. Dann bleibt Äquivalenz aus $\mathfrak{A}$ in $\mathfrak{B}$ erhalten und umgekehrt, aber es können Äquivalenzklassen in der Oberstruktur dazu kommen und größer werden.
		
		\begin{enumerate}
			\item Wenn eine endliche Zahl von Äquivalenzklassen existieren, dann bleibt die Anzahl in der elementaren Oberstruktur erhalten.
			\item Wenn eine endliche Äquivalenzklasse existiert, dann bleibt deren Größe in der elementaren Oberstruktur erhalten.
			\item Wenn jede Äquivalenzklasse $n$ Elemente hat, dann hat auch in jeder Oberstruktur jede Äquivalenzklasse $n$ Elemente.
			\item Für jedes $n \in N \setminus \{ 0 \}$ gibt es genau eine Äquivalenzklasse mit $n$ Elementen und keine unendliche Klasse. In einer Elementaren Oberstruktur kommen nur unendliche große Äquivalenzklassen dazu.
		\end{enumerate}
	\end{example}

	\begin{theorem}[Tarskis Test]
		Sei $\mathcal{L}$ eine Sprache, und $\mathfrak{B}$ eine $\mathcal{L}$-Struktur, und $A \subseteq B$. Dann ist $A$ genau dann Träger einer elementaren Unterstruktur von $\mathfrak{B}$, wenn für alle $\mathcal{L}_A$-Formeln $\varphi(v_0) \in \mathcal{F}_0(\mathcal{L}_A)$, die in $\mathfrak{B}$ erfüllt sind, gilt dass sie mit einem $a \in A$ erfüllt sind.
		
		Das heißt wenn $\mathfrak{B} \models \exists v_0 \varphi(v_0)$, dann existiert $x \in A$ mit $\mathfrak{B} \models \varphi(a)$.
	\end{theorem}

	\begin{proof}
		\heading{$\Rightarrow$} Angenommen $\mathfrak{A} \preccurlyeq \mathfrak{B} \models \exists v_0 \varphi(v_0)$ (wegen $\preccurlyeq$).
		
		Also existiert $a \in A$ mit $\mathfrak{A} \models \varphi(a)$, somit $\mathfrak{B} \models \varphi(a)$ (wegen $\preccurlyeq$)
		
		\heading{$\Leftarrow$} 
		\begin{enumerate}
			\item $\mathfrak{B} \models \exists v_0 v_0 \doteq v_0$
		
			Also gibt es $a \in A$ mit $\mathfrak{B} \models a \doteq a$ insbesondere $A \neq \emptyset$.
			
			\item Seien $f \in \mathcal{L}$ $n$-stellig, $a_1, \dots, a_n \in A$
			\begin{equation*}
				\mathfrak{B} \models \exists v_0 f a_1 \dots a_n \doteq v_0
			\end{equation*}
			\textit{Bedingung:} es existiert $a \in A$ mit $\mathfrak{B} \models f a_1 \dots a_n \doteq a$.
			
			Also $f^\mathfrak{B}(a_1, \dots, a_n) \in A$, das heißt $A$ ist Träger einer Unterstruktur.
			\item Zeige per Induktion übe den Aufbau der $\mathcal{L}_A$-Formeln
			\begin{equation*}
				\mathfrak{A} \models \varphi \Leftrightarrow \mathfrak{B} \models \varphi
			\end{equation*}
			\begin{itemize}
				\item Induktionsanfang: $\varphi$ Atomar
				\begin{align*}
					\mathfrak{A} \subseteq \mathfrak{B} &\Leftrightarrow id_A: A \to B \text{$\mathcal{L}_A$-Einbettung} \\
					&\Leftrightarrow \mathcal{L}_h \models \operatorname{Diag}(\mathfrak{A}_A) = \operatorname{Diag(\mathfrak{A})} \\
					&\Leftrightarrow \text{für alle atomaren Formeln $\varphi \in \mathcal{F}_0(\mathcal{L}_A)$ gilt: } (\mathfrak{A} \models \varphi \Leftrightarrow \mathfrak{B} \models \varphi)
				\end{align*}
				\item Induktionsschritte
				\begin{equation*}
					\mathfrak{A} \models \lnot \varphi \Leftrightarrow \mathfrak{A} \not\models \varphi \underset{\text{IV}}{\Leftrightarrow} \mathfrak{B} \not\models \varphi \Leftrightarrow \mathfrak{B} \models \lnot \varphi
				\end{equation*}
				\begin{equation*}
					\mathfrak{A} \models (\varphi_1 \land \varphi_2) \Leftrightarrow \begin{cases}
						\mathfrak{A} \models \varphi_1 \\
						\text{und} \\
						\mathfrak{A} \models \varphi_2
					\end{cases} \quad\underset{\text{IV}}{\Longleftrightarrow}\quad \begin{drcases}
						\mathfrak{B} \models \varphi_1 \\
						\text{und} \\
						\mathfrak{B} \models \varphi_2
					\end{drcases} \Leftrightarrow \mathfrak{B} \models (\varphi_1 \land \varphi_2)
				\end{equation*}
				\begin{align*}
					\mathfrak{A} \models \exists v_0 \varphi(v_0) &\Leftrightarrow \text{ex. $a \in A$ mit } \mathfrak{A} \models \varphi(a) \\
					&\underset{\text{IV}}{\Leftrightarrow} \text{ex. $a \in A$ mit } \mathfrak{B} \models \varphi(a) \\&\Rightarrow \text{ex. $a \in B$ mit } \mathfrak{B} \models \varphi(a) \Leftrightarrow \mathfrak{B} \models \exists v_0 \varphi(v_0)
				\end{align*}
			\end{itemize}
			Da $\{ \lnot, \land, \exists \}$ ein vollständiges Junktoren-Quantoren-System bilden ist die Aussage damit gezeigt.
		\end{enumerate}
	\end{proof}

	\begin{corollary}
		\label{cor:elUnterstruktExistiert}
		Sei $\mathfrak{B}$ $\mathcal{L}$-Struktur, $S \subseteq B$. Dann existiert eine elementare Unterstruktur $\mathfrak{A} \preccurlyeq \mathfrak{B}$ mit $S \subseteq A$ und $\abs{A} \leq \max \{ \abs{S}, \abs{\mathcal{L}}, \aleph_0 \}$.
	\end{corollary}

	\begin{proof}
		Definiere induktiv $S_i$ für $i \in \setN$.
		\begin{align*}
			S_0 &\coloneqq S \\
			S_{i+1} &\coloneqq S_i \cup \{ a_\varphi \mid \varphi(x) \text{ $\mathcal{L}_{S_i}$-Formel} \mathfrak{B} \models \exists \varphi(x) \text{ und $a_\varphi$ ist ein Element mit $\mathfrak{B} \models \varphi(a_\varphi)$} \} \\
			S_\omega &\coloneqq \bigcup_{i \in \omega} S_i
		\end{align*}
		Nach Konstruktion ist $S_\omega$ Träger einer elementaren Unterstruktur $\mathfrak{A} \preccurlyeq \mathfrak{B}$.
		
		Denn: Wenn $\mathfrak{B} \models \exists x \varphi(x), \varphi \in \mathcal{F}_1(\mathcal{L}_{S_\omega})$.
		
		Also existiert $n$ mit $\varphi \in \mathcal{F}_1(\mathcal{L}_{S_n})$, dann existiert $a_\varphi \in S_{n+1} \subseteq S_\omega$ mit $\mathfrak{B} \models \varphi(a_\varphi)$. Das heißt Tarskis Test gilt.
		
		\textit{Behauptung:} $\abs{S_\omega} \leq \max \{ \abs{S}, \abs{\mathcal{L}}, \aleph_0 \}$
		
		Per Induktion $\abs{S_i} \leq \max \{ \abs{S}, \abs{\mathcal{L}}, \aleph_0 \}$
		
		\heading{$i=0$}
		\begin{equation*}
			\abs{S_0} = \abs{S} \leq \max \{ \abs{S}, \abs{\mathcal{L}}, \aleph_0 \}
		\end{equation*}
		
		\heading{$i \to i+1$}
		\begin{align*}
			\abs{S_{i+1}} &\leq \abs{S_i} + \underbrace{\abs{\mathcal{F}_1(\mathcal{L}_{S_i})}}_\text{endliche Folgen mit Zeichen aus $Z(S_i)$} \\
			&\leq \abs{S_i} + \abs{Z(S_i)^{< \omega}} \\
			&= \abs{S_i} + \abs{Z(S_i)} \\
			&= \abs{S_i} + \abs{\mathcal{L}} + \aleph_0 + \abs{S_i} \\
			&= \abs{\mathcal{L}} + \abs{S_i} + \aleph_0 \\
			&\overset{\text{IV}}{\leq} \abs{\mathcal{L}} + \max \{ \abs{\mathcal{L}}, \abs{S}, \aleph_0 \} + \aleph_0 \\
			&= \max \{ \abs{L}, \abs{S}, \aleph_0 \}
		\end{align*}
		wobei
		\begin{equation*}
			Z(S_i) = \mathcal{L} \cup \{ v_0, v_1, \dots \} \cup \{ \lnot, \lor, \land, \exists, \forall \} \cup S_i
		\end{equation*}
	\end{proof}

	\begin{remark}
		Für $\abs{\mathcal{L}} = \abs{S} = \aleph_0$ heißt die Folgerung auch Satz von Löwenheim.
	\end{remark}

	Sei $\mathfrak{A}_0 \subseteq \mathfrak{A}_1 \subseteq \mathfrak{A}_2 \subseteq \dots$ eine gerichtete Vereinigung.
	
	Es gibt eine eindeutig bestimmte $\mathcal{L}$-Struktur $\mathfrak{A}_\omega$ auf $\bigcup_{i \in \omega} A_i$, so dass $\mathfrak{A}_i \subseteq \mathfrak{A}_\omega$ für alle $i$.
	
	\begin{theorem}
		Falls $\mathfrak{A}_0 \preccurlyeq \mathfrak{A}_1 \preccurlyeq \mathfrak{A}_2 \preccurlyeq \dots$ dann gilt $\mathfrak{A}_i \preccurlyeq \mathfrak{A}_\omega$ für alle $i$.
	\end{theorem}

	\begin{proof}
		Induktion über den Aufbau der Formeln: $\mathfrak{A}_i \models \varphi \Leftrightarrow \mathfrak{A}_\omega \models \varphi$ für $\varphi \in \mathcal{F}_0(\mathcal{L}_{A_i})$
		
		\heading{Atomar:} da $\mathfrak{A}_i \subseteq \mathfrak{A}_\omega$
		
		\heading{Negation und Konjunktion:} wie letztes Mal
		
		\heading{Existenzquantor:} $\mathfrak{A}_i \models \exists x \varphi(x)$ dann $\mathfrak{A}_i \models \varphi(a)$ für ein $a \in A_i$.
		
		$\overset{\text{IV}}{\Rightarrow} \mathfrak{A}_\omega \models \varphi(a)$ also $\mathfrak{A}_\omega \models \exists x \varphi(x)$.
		
		$\mathfrak{A}_\omega \models \exists x \varphi(x)$, dann $\mathfrak{A}_\omega \models \varphi(a)$ für ein $a \in A_\omega$. Das heißt ex existiert $n \geq i$ mit $a \in A_n$.
		
		Also gilt $\mathfrak{A}_n \models \varphi(a)$ und somit
		\begin{equation*}
			\mathfrak{A}_i \preccurlyeq \mathfrak{A}_n \models \exists x \varphi(x) \Rightarrow \mathfrak{A}_i \models \exists x \varphi()x
		\end{equation*}
	\end{proof}

	\subsection{Kompaktheitssatz und Ultraprodukte}
	
	\begin{theorem}[Kompaktheitssatz]
		Sei $\mathcal{L}$ eine Sprache und $T$ eine $\mathcal{L}$-Theorie.
		
		$T$ hat genau dann ein Modell, wenn jede endliche Teiltheorie $T_0 \subseteq T$ ein Modell hat.
	\end{theorem}

	\begin{corollary}[Satz von Löwenheim-Skolem-Tarski aufwärts]
		Sei $\mathcal{L}$ eine Sprache und $\mathfrak{A}$ eine unendliche $\mathcal{L}$-Struktur.
		Dann existiert zu jeder Kardinalzahl $\kappa \geq \max \{ \abs{A}, \abs{\mathcal{L}} \}$ ein $\mathfrak{B} \succcurlyeq \mathfrak{A}$ mit $\abs{B} = \kappa$.
	\end{corollary}

	\begin{proof}
		Betrachte $\mathcal{L}^c \coloneqq \mathcal{L}_A \;\dot{\cup}\; \{ c_i \mid i < \kappa \}$
		
		und die $\mathcal{L}^C$-Theorie $T^c \coloneqq \operatorname{Th}(\mathfrak{A}_A) \;\cup\; \{ \lnot c_i \doteq c_j \mid i \neq j \}$
		
		Zeige mit dem Kompaktheitssatz: $T^c$ ist konsistent.
		
		Sei $T_0 \subseteq_\text{endl} T^c$.
		
		Dann $T_0 \subseteq \operatorname{Th}(\mathfrak{A}) \cup \{ \lnot c_i \doteq c_j \mid i,j \in \text{ endlicher Menge} \}$.
		
		$\mathfrak{A}$ wird Modell von $T_0$, indem man die endlich vielen Konstanten in $T_0$ durch beliebige, paarweise verschiedene Elemente von $A$ interpretiert.
		
		Sei $\mathcal{L}' \models T^c$.
		
		Dann ist $\underbrace{\mathcal{L}' \upharpoonright_\mathcal{L}}_\text{Redukt auf $\mathcal{L}$} \succcurlyeq \mathfrak{A}$ und $\abs{B'} \geq \kappa$.
		
		Wähle Teilmenge $S \subseteq B$, die $A$ enthält und so, dass $\abs{S} = \kappa$. Wende Folgerung \ref{cor:elUnterstruktExistiert} auf $\mathfrak{B}'_A$ an.
		
		Dann erhält man $\mathfrak{B} \preccurlyeq \mathfrak{B}'_A$ in $\mathcal{L}_A$ mit $\abs{B} \geq \abs{S} = \kappa$ und $\abs{B} \leq \max \{ \abs{\mathcal{L}_A}, \abs{S}, \aleph_0 \} = \kappa$
		
		Und
		\begin{equation*}
			\begin{drcases}
				\mathfrak{A} \preccurlyeq \mathfrak{B}' \text{ in $\mathcal{L}_A$} \\
				\mathfrak{B} \preccurlyeq \mathfrak{B}' \text{ in $\mathcal{L}_A$} \\
				A \subseteq B'
			\end{drcases} \Rightarrow \mathfrak{A} \preccurlyeq \mathfrak{B}
		\end{equation*}
	\end{proof}

	\heading{Ultraprodukte}

	Seien $\mathfrak{A}_i$ $\mathcal{L}$-Strukturen $(i \in I)$ und sei
	\begin{equation*}
		\prod_{i \in I} A_i = \{ p: I \to \bigcup_{i \in I} A_i \mid p(i) \in A_i \}
	\end{equation*}
	Mit dem Auswahlaxiom gilt:
	\begin{equation*}
		A_i \neq \emptyset \text{ für alle $i \in I$} \Rightarrow \prod_{i \in I} A_i \neq \emptyset
	\end{equation*}
	
	Definiere $\mathcal{L}$-Struktur $\prod_{i \in I} \mathfrak{A}_i$ auf $\prod_{i \in I} A_i$.
	
	\begin{align*}
		f^\mathfrak{A}(p_1, \dots, p_n) = p \quad&\Leftrightarrow\quad \text{für alle $i \in I$ } p(i) =  f^{\mathfrak{A}_i}(p_1(i), \dots, p_n(i) \\
		(p_1, \dots, p_n) \in R^\mathfrak{A} \quad&\Leftrightarrow\quad \text{für alle $i \in I$ } (p_1(i), \dots, p_n(i)) \in R^{\mathfrak{A}_i}
	\end{align*}
	
	Betrachte Ultrafilter $\mathcal{U}$ in $\operatorname{Pot}(I)$ also
	\begin{itemize}
		\item $\mathcal{U} \subsetneq \operatorname{Pot}(I), \emptyset \notin \mathcal{U}$
		\item Wenn $X \in \mathcal{U}, X \subseteq Y$, dann $Y \in \mathcal{U}$
		\item Wenn $X,Y \in \mathcal{U}$, dann $X \cap Y \in \mathcal{U}$
		\item Wenn $X \subseteq I$, dann entweder $X \in \mathcal{U}$ oder $I \setminus X \in \mathcal{U}$.
	\end{itemize}

	Ultrafilter $\mathcal{U}$ definiert eine Art Maß auf $\operatorname{Pot}(I)$
	\begin{equation*}
		\mu_\mathcal{U} = \chi_\mathcal{U}: X \mapsto \begin{cases}
			1 & \text{wenn $X \in \mathcal{U}$} \\
			0 & \text{wenn $X \notin \mathcal{U}$}
		\end{cases}
	\end{equation*}
	$X$ mit $X \in \mathcal{U}$ heißt auch $\mathcal{U}$-groß.
	
	\begin{lemma}
		Ein Ultrafilter $\mathcal{U}$ definiert eine Äquivalenzrelation $\sim_\mathcal{U}$ auf $\prod_{i \in I} A_i$ durch
		\begin{equation*}
			p \sim_\mathcal{U} p' :\Leftrightarrow \{ i \in I \mid p(i) = p'(i) \} \in \mathcal{U}
		\end{equation*}
	\end{lemma}

	\begin{proof}
		\begin{itemize}
			\item Reflexiv: klar, da $I \in \mathcal{U}$
			\item Symmetrie: klar per Definition
			\item Transitivität: $p \sim_\mathcal{U} p' \sim_\mathcal{U} p''$
			\begin{equation*}
				\{ i \mid p(i) = p''(i) \} \supseteq \{ i \mid p(i) = p'(i) \} \cap \{ i \mid p'(i) = p''(i) \} \in \mathcal{U} \cap \mathcal{U} = \mathcal{U}.
			\end{equation*}
		\end{itemize}
	\end{proof}

	\begin{definition}
		Seien $\mathfrak{A}_i (i \in I)$ $\mathcal{L}$-Strukturen, $\mathcal{U}$ ein Ultrafilter auf $I$.
		
		Das Ultraprodukt der $\mathfrak{A}_i$ bezüglich $\mathcal{U}$ ist die $\mathcal{L}$-Struktur
		\begin{equation*}
			\prod_{i \in I} \mathfrak{A}_i / \sim_\mathcal{U}
		\end{equation*}
		mit Träger $\prod_{i \in I} A_i / \sim_\mathcal{U}$ und
		\begin{align*}
			(p_1/\sim_\mathcal{U}, \dots, p_m/\sim_\mathcal{U}) \in R^{\prod_{i \in I} \mathfrak{A}_i / \sim_\mathcal{U}} \quad&:\Leftrightarrow\quad \{ i \mid (p_1(i), \dots, p_n(i)) \in R^\mathfrak{A}_i \} \in \mathcal{U} \\f^{\prod_{i \in I} \mathfrak{A}_i / \sim_\mathcal{U}}(p_1/\sim_\mathcal{U}, \dots, p_m/\sim_\mathcal{U}) = p/\sim_\mathcal{U} \quad&:\Leftrightarrow\quad \{ i \mid f^{\mathfrak{A}_i}(p_1(i), \dots, p_n(i) = p(i) \} \in \mathcal{U}
		\end{align*}
	\end{definition}
	\begin{proof}
		\heading{Wohldefiniertheit}
		
		Seien $p_1 \sim_\mathcal{U} p_1', \dots, p_n \sim_\mathcal{U} p_n'$ zu zeigen ist
		\begin{equation*}
			X \coloneqq \{ i \mid (p_1(i), \dots, p_n(i)) \in R^{\mathfrak{A}_i} \} \in \mathcal{U} \Leftrightarrow \{ i \mid (p_1'(i), \dots, p_n'(i)) \in R^{\mathfrak{A}_i} \} \in \mathcal{U}
		\end{equation*}
		
		Sei $X_j = \{ i \mid p_j(i) = p_j'(i) \} \in \mathcal{U}$.
		
		Falls $X \in \mathcal{U}$ auf $X \cap X_1 \cap \dots \cap X_n \in \mathcal{U}$ gilt
		\begin{equation*}
			\begin{drcases}
				(p_1(i), \dots, p_n(i)) \in R^{\mathfrak{A}_i} \\
				p_1(i) = p_1'(i) \\
				\vdots \\
				p_n(i) = p_n'(i)
			\end{drcases} \Rightarrow (p_1'(i), \dots, p_n'(i)) \in R^{\mathfrak{A}_i}
		\end{equation*}
		
		Analog für Funktionszeichen.
		
		Warum existiert überhaupt solch ein $p_\mathcal{U}$?
		
		Man sieht, dass $f^{\prod_{i \in I} \mathfrak{A}_i} (p_1, \dots, p_n) / \mathcal{U}$ es tut.
		
		Falls $\mathfrak{A}_i = \mathfrak{A}$ für alle $i \in I$ dann heißt $\prod_{i \in I} \mathfrak{A}/\mathcal{U} = \mathfrak{A}^I/\mathcal{U}$ auch Ultrapotenz von $\mathfrak{A}$.
	\end{proof}

	\begin{theorem}[Satz von \L os]
		Sei $\varphi$ eine $\mathcal{L}$-Aussage dann gilt
		\begin{equation*}
			\prod_{i \in I} \mathfrak{A}_i / \mathcal{U} \models \varphi \quad\Leftrightarrow\quad \{ i \mid \mathfrak{A}_i \models \varphi \} \in \mathcal{U}
		\end{equation*}
		Insbesondere 
		\begin{itemize}
			\item falls $\mathfrak{A}_i \models T$ für alle $i$, dann $\prod_{i \in I} \mathfrak{A}_i / \mathcal{U} \models T$
			\item falls $\mathfrak{A}_i \equiv \mathfrak{A}_j$ für alle $i \in I$, dann $\prod \mathfrak{A}_i / \mathcal{U} \equiv \mathfrak{A}_i$
		\end{itemize}
	\end{theorem}

	\begin{corollary}
		\begin{equation*}
		 \delta: \mathfrak{A} \to \mathfrak{A}^I / \mathcal{U}, \quad a \mapsto (a,a, \dots, a,a)/ \mathcal{U}
		\end{equation*}
		ist elementare Einbettung, das heißt
		\begin{equation*}
			\mathfrak{A} \preccurlyeq \mathfrak{A}^I/\mathcal{U}
		\end{equation*}
	\end{corollary}

	\begin{proof} zum Satz von \L os (Skizze)
		
		Induktion über den Aufbau der Formeln
		
		\begin{itemize}
			\item $\varphi$ atomar: Entweder Induktion über den Aufbau der Terme oder betrachte termreduzierte Formeln. Dazu sei $f$ einstellig und $c$ Konstante eine atomare Formel ist auch $f f c \doteq c$, diese ist aber äquivalent zu $\exists x (f c \doteq x  \land f x \doteq c)$. Das heißt ohne Einschränkung kann man nur atomare Formeln der Formen $R \tau_1 \dots \tau_n$ oder $\tau_1 \doteq \tau_2$ oder $f \tau_1 \dots \tau_n \doteq \tau$ betrachten, wobei $\tau_i, \tau$ Konstanten oder Individuenvariablen sind.
			\item Satz von \L os für termreduzierte atomare Formeln ist im Wesentlichen die Definition der $\mathcal{L}$-Struktur auf $\prod A_i / \sim_\mathcal{U}$.
			\item Induktion:
			
			Für und
			\begin{align*}
				&\prod \mathfrak{A}_i / \mathcal{U} \models (\varphi \land \psi) \\
				\Leftrightarrow& \prod \mathfrak{A}_i / \mathcal{U} \models \phi \text{ und } \prod \mathfrak{A}_i / \mathcal{U} \models \psi \\
				\Leftrightarrow& I_\varphi = \{ i \mid \mathfrak{A}_i \models \varphi \} \in \mathcal{U} \text{ und } I_\psi = \{ i \mid \mathfrak{A}_i \models \psi\} \in \mathcal{U} \\
				\Leftrightarrow& \{ i \mid \mathfrak{A}_i \models \varphi \land \psi \} = I_\varphi \cap I_\psi \in \mathcal{U}
			\end{align*}
			
			Für nicht
			\begin{align*}
				&\prod_{i \in I} \mathfrak{A}_i / \mathcal{U} \models \lnot \varphi \\
				\Leftrightarrow& \prod_{i \in I} \mathfrak{A}_i / \mathcal{U} \not\models \varphi \\
				\Leftrightarrow& I_\varphi = \{ i \mid \mathfrak{A}_i \models \varphi \} \notin \mathcal{U} \\
				\overset{\text{Ultra}}{\Leftrightarrow}&  I \setminus I_\varphi \{ i \mid \mathfrak{A}_i \models \lnot \varphi \} \in \mathcal{U}
			\end{align*}
			
			Für Existenz
			\begin{align*}
				&\prod_{i \in I} \mathfrak{A}_i / \mathcal{U} \models \exists x \varphi \\
				\Leftrightarrow& \text{ex existiert $p$ mit } \prod \mathfrak{A}_i/\mathcal{U} \models \varphi(p / \mathcal{U}) \\
				\overset{\text{Ind.}}{\Leftrightarrow}& \text{es existiert $p$ mit } \{ i \mid \mathfrak{A}_i \models \varphi(p(i)) \} \in \mathcal{U} \\
				\Leftrightarrow& \{ i \mid \text{ex $p(i) \in A_i$ mit } \mathfrak{A}_i \models \varphi(p(i)) \} \in \mathcal{U} \\
				\Leftrightarrow& \{ i \mid \mathfrak{A}_i \models \exists x \varphi \} \in \mathcal{U}
			\end{align*}
		\end{itemize}
	\end{proof}

	\begin{remark}
		\begin{itemize}
			\item $\langle i \rangle = \{ X \subseteq I \mid i \in X \}$ Ultrafilter, der von $i$ erzeugte Haupt-Ultrafilter
			\begin{equation*}
				\prod_{i \in I} \mathfrak{A}_i/\langle i \rangle \cong \mathfrak{A}_i
			\end{equation*}
			\item Mit Lemma von Zorn (bzw. AC): Jeder eigentliche Filter kann zu einem Ultrafilter erweitert werden.
		\end{itemize}
	\end{remark}

	\begin{definition}
		Sei $I$ eine unendliche Menge, betrachte Filter der $\omega$-endlichen Mengen
		\begin{equation*}
			\mathcal{F} = \{ X \mid I \setminus X \text{ endlich} \}
		\end{equation*}
		$\mathcal{F}$ kann zu Ultrafilter $\mathcal{U}$ erweitert werden. Solche Ultrafilter heißen freie Ultrafilter. Dies sind die nicht-Haupt-Ultrafilter.
	\end{definition}

	\begin{remark}
		Wenn $\mathfrak{A}$ endlich ist, dann ist $\mathfrak{A}^I/\mathcal{U} \cong \mathfrak{A}$.
		
		Wenn $\mathfrak{A}$ unendlich ist und $\mathcal{U}$ frei ist, dann ist häufig $\mathfrak{A} \precnsim \prod \mathfrak{A}_i / \mathcal{U}$.
		
		Wenn $\abs{A_i} < \abs{A_{i+1}}$ endlich ist und $\mathcal{U}$ frei, dann ist
		\begin{equation*}
			\abs{\prod_{i \in I} \mathfrak{A}_i / \mathcal{U}} = 2^{\aleph_0}
		\end{equation*}
		
		Wenn $\abs{A_i} = \aleph_0$ für alle $i$ und $\mathcal{U}$ frei,
		\begin{equation*}
			\abs{\prod_{i \in I} \mathfrak{A}_i / \mathcal{U}} = 2^{\aleph_0}
		\end{equation*}
	\end{remark}

	\begin{theorem}
		Seien $\mathfrak{A}_i (i \in \setN)$ endliche $\mathcal{L}$-Strukturen. Für jedes $n \in \setN$ sei nur endlich oft $\abs{A_i} \leq n$. Sei $\mathcal{U}$ freier Ultrafilter auf $\setN$. Dann ist
		\begin{equation*}
			\abs{\prod_{i \in I} \mathfrak{A}_i / \mathcal{U}} = 2^{\aleph_0}
		\end{equation*}
	\end{theorem}

	\begin{proof}
		\begin{equation*}
			\abs{\prod_{i \in \setN} A_i} \leq \sup \{ \abs{A_i} \mid i \in \setN \}^{\aleph_0} = \aleph_0^{\aleph_0} = 2^{\aleph_0}
		\end{equation*}
		Damit
		\begin{equation*}
			\abs{\prod A_i / \mathcal{U}} \leq 2^{\aleph_0}
		\end{equation*}
		
		\heading{Für $\geq$:} Ohne Einschränkung sei $\abs{A_i} \leq \abs{A_{i+1}}$ und $\abs{A_i} = \{ 0, \dots, n_i \}$ mit $n_i = \abs{A_i} - 1$.
		
		Für $r,s \in \setR \cap [0,1)$ konstruiere $p_r \in \prod_{i \in I} A_i$ mit $r \neq s$, dann stimmen $p_r$ und $p_s$ nur auf endlich vielen Indizes überein.
		
		\begin{equation*}
			p_r(i) \coloneqq j \Leftrightarrow r \in \left[ \frac{j}{\abs{A_i}}, \frac{j+1}{\abs{A_i}} \right)
		\end{equation*}
		
		\begin{equation*}
			\Rightarrow p_r \nsim_\mathcal{U} p_s
		\end{equation*}
	\end{proof}

	\begin{proof} zum Kompaktheitssatz
		
		Sei $T$ eine endlich erfüllbare $\mathcal{L}$-Theorie. Zu zeigen ist $T$ ist konsistent.
		
		Sei $I = \operatorname{Pot}_{<\aleph_0}(T) = \{ T_0 \mid T_0 \subseteq_\text{endl} T \}$.
		
		Für $T_0 \subseteq_\text{endl} T$ d.h. $T_0 \in I$ sei $\langle T_0 \rangle = \{ T_1 \in I \mid T_0 \subseteq T_1 \}$.
		
		Sei weiter $\mathcal{F} = \{ \mathcal{X} \subseteq I \mid \text{ ex. $T_0 \in I$ mit } \langle T_0 \rangle \subseteq \mathcal{X} \}$.
		
		$\mathcal{F}$ ist Filter auf $I$:
		\begin{itemize}
			\item $\emptyset \in \mathcal{F}$
			\item Monotonie: per Definition
			\item $\mathcal{X}_1, \mathcal{X}_2 \in \mathcal{F}$, dann existiert $T_i \subseteq_\text{endl} T$ mit $\langle T_i \rangle \subseteq \mathcal{X}_i$. Dann gilt
			\begin{equation*}
				\langle T_1 \cup T_2 \rangle = \langle T_1 \rangle \cap \langle T_2 \rangle \subseteq \mathcal{X}_1 \cap \mathcal{X}_2
			\end{equation*}
		\end{itemize}
	
		Sei $\mathcal{U}$ ein Ultrafilter, der $\mathcal{F}$ erweitert. Wähle für jedes $T_0 \in I$ ein Modell $\mathfrak{M}_{T_0} \models T_0$ und setze $\mathfrak{M} \coloneq \prod_{T_0 \in I} \mathfrak{M}_{T_0} / \mathcal{U}$.
		
		Mit Satz von \L os: prüfe, dass $\varphi \in T \Rightarrow \mathfrak{M} \models \varphi$.
		
		\begin{equation*}
			\{ T_1 \in I \mid \mathfrak{M}_{T_1} \models \varphi \} \supseteq \{ T1 \in I \mid \varphi \in T_1 \} = \langle \{ \varphi \} \rangle \in \mathcal{F} \subseteq \mathcal{U}
		\end{equation*}
	\end{proof}

	\begin{definition}
		$(X, \mathcal{O})$ heißt topologischer Raum und $\mathcal{O}$ heißt Topologie auf $X$), falls
		\begin{itemize}
			\item $\mathcal{O} \subseteq \operatorname{Pot(X)}$
			\item $\mathcal{O}$ ist abgeschlossen bezüglich endlicher Schnitte und beliebiger Vereinigungen
			\item Insbesondere $\emptyset, X \in \mathcal{O}$
		\end{itemize}
	
		$U \in \mathcal{O}$ heißt offen bzw. offene Menge, $A \subseteq X$ mit $X \setminus A \in \mathcal{O}$ heißt abgeschlossen bzw. abgeschlossene Menge.
	\end{definition}

	\begin{definition}
		$Q \subseteq \operatorname{Pot}(X)$ heißt Basis einer Topologie $\mathcal{O}$, falls $Q$ abgeschlossen ist bezüglich endlicher Schnitte.
		
		Dann ist $\mathcal{O} = \{ \bigcup Q_i \mid Q_i \in Q \} \cup \{ \emptyset, X \}$ eine Topologie, und zwar die kleinste, in der alle Mengen aus $Q$ offen sind.
	\end{definition}

	\begin{definition}
		Eine Abbildung heißt stetig, falls Urbilder offener Mengen wieder offen sind.
	\end{definition}

	Sei $\mathfrak{B}$ eine Boolesche Algebra und $\mathcal{U}_\mathfrak{B}$ die Menge der Ultrafilter in $\mathfrak{B}$. Damit ist $\mathcal{U} \in \operatorname{Pot}(B)$ also $\mathcal{U}_\mathfrak{B} \in \operatorname{Pot}(\operatorname{Pot}(B))$.
	
	Für $a \in B$, definiere
	\begin{equation*}
		[[a]] \coloneqq \{ U \in \mathcal{U}_\mathfrak{B} \mid a \in U \} \subseteq \mathcal{U}_\mathfrak{B}
	\end{equation*}
	
	\begin{theorem}""
		\begin{enumerate}
			\item $[[\cdot]]: \mathfrak{B} \hookrightarrow \operatorname{Pot}(\mathcal{U}_\mathfrak{B})$ ist Einbettung Boolescher Algebren (Teil des Stoneschen Repräsentationssatzes)
			\item $\{ [[a]] \mid a \in B \}$ ist Basis einer Topologie auf $\mathcal{U}_\mathfrak{B}$.
		\end{enumerate}
	\end{theorem}

	\begin{definition}
		$\mathcal{U}_\mathfrak{B}$ heißt auch Stone-Raum $S(\mathfrak{B})$ von $\mathfrak{B}$.
	\end{definition}

	\begin{proof}""
		\begin{enumerate}
			\item
			\begin{itemize}
				\item $[[0]] = \emptyset$, da $0 \notin U$ per Definition
				\item $[[1]] = U_\mathfrak{B}$, da $1 \in U$ für jedes $U$
				\item $[[a \cap b]] = [[a]] \cap [[b]]$ folgt aus den Filtereigenschaften
				\item $[[a \cup b]] = [[a]] \cup [[b]]$ folgt aus de Morgan und dem nächsten Schritt
				\item $[[a^c]] = \{ U \mid a^c \in U \} \overset{\text{ultra}}{=} \{ U \mid a \notin U \} = [[a]]^c$
			\end{itemize}
			Das heißt $[[\cdot]]$ ist Homomorphismus der Booleschen Algebra.
		
			Fehlt noch injektivität: Seien $a \neq b$: Zu zeigen ist, es existiert ein Ultrafilter $U$ der $a$ und $b$ trennt, das heißt $a \in U \Leftrightarrow b \notin U$.
		
			Es gilt $a \nsubseteq b$ oder $b \nsubseteq a$ das heißt $a \cap b^c \neq \emptyset$ oder $a^c \cap b \neq \emptyset$.
		
			Es existiert also ultrafilter $U$ mit $a \cap b^c \in U$ oder $a^c \cap b \in U$.
			
			Falls z.B. $a \cap b^c \in U$, dann ist $a \in U, b^c \in U \Rightarrow b \notin U$.
			\item Wegen $[[a]] \cap [[b]] = [[a \cap b]]$
		\end{enumerate}
	\end{proof}

	\begin{remark}
		Die Basis-offenen Mengen $[[a]]$ sind auch abgeschlossen, da $[[a]]^c = [[a^c]]$.
		
		Mengen die offen und abgeschlossen sind heißen clopen.
		
		Topologische Räume mit einer Basis aus clopen Mengen sind total unzusammenhängend.
	\end{remark}

	\begin{definition}
		Ein topologischer Raum $(X, \mathcal{O})$ heißt kompakt, falls die endliche Überdeckungseigenschaft gilt:
		
		Falls $X = \bigcup_{i \in I} \{ U_i \mid U_i \text{ offen} \}$ dann existiert $I_0 \subseteq_\text{endl} I$ mit $X = \bigcup \{ U_i \mid i \in I_0 \}$
		
		Oder in äquivalenter Formulierung: $\bigcap \{ A_i \mid A_i \text{ abgeschlossen}, i \in I \} = \emptyset$ dann existiert $I_0 \subseteq_\text{endl} I$ mit $\bigcap \{ A_i \mid i \in I_0 \} = \emptyset$.
	\end{definition}

	\begin{theorem}
		Der Stone-Raum ist kompakt.
	\end{theorem}

	\begin{remark}
		Der Kompaktheitssatz ist äquivalent zur Kompaktheit von $S(\mathcal{F}_0(\mathcal{L}))$.
		
		Ohne Einschränkung sei $\bot \notin T$
		\begin{equation*}
			T \text{ inkonsistent} \qquad\Leftrightarrow\qquad \bigcap_{\varphi \in T} [[\varphi]] = \emptyset
		\end{equation*}
		und nach Kompaktheitssatz sagt es gibt endliches $T_0 \subseteq T$ so dass $T_0$ inkonsistent ist.
		
		Und mit der Kompaktheit von $S(\mathcal{F}_0(\mathcal{L}))$ existieren $\varphi_0, \dots, \varphi_n \in T$ mit $[\varphi_0[]] \cap \dots \cap [[\varphi_n]] = \emptyset$.
		
		Wir können im ersten Fall $T_0 = \{ \varphi_0, \dots, \varphi_n \}$ mit $\varphi_i$ aus dem zweiten Teil wählen.
	\end{remark}

	\subsubsection{Beispiele und Anwendungen}
	
	\begin{theorem}[Test von Vaught]
		$T$ sei eine konsistente $\mathcal{L}$-Theorie ohne endliche Modelle und es gebe $\kappa > \max \{ \aleph_0, \abs{\mathcal{L}} \}$, so dass $T$ bis auf Isomorphie höchstens genau ein Modell der Kardinalität $\kappa$ hat. ($T$ ist $\kappa$-Kategorisch)
		
		Dann ist $T$ vollständig.
	\end{theorem}

	\begin{proof}
		Seien $\mathfrak{A}, \mathfrak{B} \models T$ zu zeigen ist $\mathfrak{A} \equiv \mathfrak{B}$. $A$ und $B$ sind nach Voraussetzung unendlich.
		
		Sei $\kappa' > \max \{ \kappa, \abs{A}, \abs{B} \}$. Nach Löwenheim-Skolem-Tarski gibt es $\mathfrak{A}' \succcurlyeq \mathfrak{A}, \mathfrak{B}' \succcurlyeq \mathfrak{B}$ mit $\abs{A'} = \abs{B'} = \kappa'$ und wiederum nach Löwenheim-Skolem-Tarski existieren $\mathfrak{A}'' \preccurlyeq \mathfrak{A}', \mathfrak{B}'' \preccurlyeq \mathfrak{B}'$ mit $\abs{A''} = \abs{B''} = \kappa$. Damit gilt $\mathfrak{A} \equiv \mathfrak{A}' \equiv \mathfrak{A}'' \cong \mathfrak{B}'' \equiv \mathfrak{B}' \equiv \mathfrak{B}$ und $\cong \Rightarrow \equiv$.
	\end{proof}

	\begin{example}
		\begin{enumerate}
			\item $K$-Vektorräume, bis auf Isomorphie ist ein $K$-Vektorraum durch seine Dimension bestimmt. $\dim_K(V) \geq \abs{K} + \aleph_0 \Rightarrow \abs{V} = \dim_K(V)$.
			
			Übliche Axiomatisierung: $\mathcal{L}_\text{$K$-VR}= \{ +, -, 0, (\lambda_k)_{k \in K} \}$ und
			\begin{align*}
				T_\text{$K$-VR} &= \text{abelsche Gruppe} \\
				& \cup \{ \forall v \lambda_k v + \lambda_{k'} v = \lambda_{k+k'} v \mid k, k' \in K \} \\
				& \cup \{ \forall v \lambda_1 v = v \} \\
				& \cup \dots \\
				& \cup \{ \text{es gibt unendlich viele Elemente} \}
			\end{align*}
			
			Aus LA: $T_\text{$K$-VR}$ ist $\kappa$-kategorisch für alle $\kappa \geq \abs{K} + \aleph_0$.
			
			Axiomatisierung von Vektorräumen über variablen Körpern
			\begin{equation*}
				\mathcal{L} = \{ +_V, -_V, 0_V, +_K, \cdot_K, -_K, 0_k, 1_k, V, K \}
			\end{equation*}
			mit $V,K$ einstellige Relationszeichen und man drückt aus $V,K$ ist Partition des Universums.
			\begin{align*}
				\forall x ( V x \lor K x) \\
				\lnot \exists x (V x \land K x)
			\end{align*}
			
			und zusätzlich $+_V, -_V, 0_V$ ist abelsche Gruppe auf $V$ und z.B. $\forall x \forall y (K x \rightarrow (x +_V y = 0_K))$.
			
			$+_K, \cdot_k, \dots$ ist Körper auf $K$ und zusätzlich noch die Vektorraumaxiome.
			
			\item Offene dichte lineare Ordnungen $\mathcal{L} = \{ < \}$. Die Theorie dazu nennen wir $T_{DLO}$.
			\begin{itemize}
				\item offen: kein Maximum und kein Minimum
				\item dicht: $\forall x \forall y ( x < y \rightarrow \exists z (x < z \land z < y))$
			\end{itemize}
			Beispiele sind $(\setQ, <)$ und $(\setR, <)$
			
			\textit{Satz von Cantor:} $T_{DLO}$ ist $\aleph_0$-kategorisch
			
			\begin{proof}
				Seien $(A, <), (B, <)$ abzählbare, offene, dichte lineare Ordnungen.
				
				Seien $\{ a_i \mid i \in \omega \} = A, \{ b_i \mid b \in \omega \} = B$ Aufzählungen der Universen.
				
				Konstruiere induktiv ordnungserhaltende Bijektion $\beta: A \to B$ und setze $\beta(a_0) = b_0$.
				
				\heading{Ungerade Induktionsschritte:} Sei $\beta$ bereits auf $A_n = \{ a_{i_0}, \dots, a_{i_n} \}$ definiert, $\abs{A_n}$ ist ungerade. Idee: stelle sicher, dass $\beta$ surjektiv wird.
				
				Wähle $j$ minimal mit $b_j \notin \beta(A_n)$. Wähle $a_{i_{n+1}}$ so, dass $\beta(a_j) \coloneqq b_j$ eine ordnungserhaltende Fortsetzung des bisher konstruierten $\beta$ ist. $a_{i_{n+1}}$ existiert, da offen und dicht, setze $b_{i_{n+1}} = b_j$.
				
				\heading{Gerade Induktionsschritte:} Idee: Stelle sicher, dass $\beta$ totale Funktion ist.
				
				Sei $j$ der kleinste Index, so dass $\beta(a_j)$ noch nicht definiert ist. Wähle $b_{i_{n+1}}$ so, dass $a_{i_{n+1}} \coloneqq a_{i_{n+1}} \coloneqq a_j \mapsto b_{i_{n+1}}$ ordnungserhaltende Fortsetzung ist.
			\end{proof}
		
			Folgerung: $T_{DLO}$ ist vollständig.
			
			Aber: Für $k > \aleph_0$, gibt es $2^\kappa$ viele Modelle der Mächtigkeit $\kappa$, die paarweise $\ncong$.
		\end{enumerate}
	\end{example}

	\heading{Anwendungen}
	
	\begin{itemize}
		\item Vollständigkeit von Theorien
		\item Nichtstandardmodelle
		
		Nichtstandard-Modelle der Peano-Arithmetik
		\begin{equation*}
			(\setN, +, \cdot) \precneqq \setN^*
		\end{equation*}
		Nichtstandard-Analysis
		\begin{equation*}
			(\setR, +, \cdot, <) \precneqq \setR^*
		\end{equation*}
		Unendliches Modell der Theorie der endlichen Körper der Charakteristik $p$, pseudo-endlicher Körper
		\begin{equation*}
			\prod_{n \in \setN} \mathbb{F}_{p^n} / \mathcal{U}
		\end{equation*}
		Besonderer Automorphismus
		\begin{equation*}
			\prod_{p \text{ prim}} (\tilde{\mathbb{F}}_p, \operatorname{Frob:} x \mapsto x^p) / \mathcal{U} = (\setC, \alpha)
		\end{equation*}
		
		\item Transfer-Prinzipien
		
		$\mathcal{L}_{K_p} = \{ +, \cdot, -, 0, 1 \}$
		
		Eine $\mathcal{L}_{K_p}$-Aussage $\varphi$ gilt in allen Körpern der Charakteristik $0$ $\Leftrightarrow$ ex existiert $n_0 \in \setN$ so, dass $\varphi$ in allen Körpern der Charakteristik $p$ mit $p \geq n_0$ gilt.
		
		\item Nicht-Axiomatisierbarkeit
	\end{itemize}

	\section{Quantorenelimination}
	
	\begin{example}
		$\mathfrak{R} = (\setR, +, \cdot, -, 0, 1)$ hat keine Quantorenelimination: $\exists y \; y \cdot y = x$ ist in $\mathfrak{R}$ nicht äquivalent zu einer Formel ohne Quantoren.
		
		$\mathfrak{R}' = (\setR, +, \cdot, -, 0, 1, \leq)$ hat Quantorenelimination hier gilt dann:
		\begin{equation*}
			\exists y \; y \cdot y = x \qquad \sim \qquad 0 \leq x
		\end{equation*}
	\end{example}

	\subsection{Erhaltungssätze (Präservationssätze)}
	
	\begin{definition}
		$\varphi$ $\mathcal{L}$-Formel heißt
		\begin{itemize}
			\item quantorenfrei, wenn kein Quantor $\forall, \exists$ in $\varphi$ vorkommt
			\item universell, wenn $\varphi$ von der Form ist:
			\begin{equation*}
				\forall v_{i_1} \dots \forall v_{i_n} \psi \qquad n \in \setN, \psi \text{ quantorenfrei}
			\end{equation*}
			\item existenziell, wenn $\varphi$ von der Form ist:
			\begin{equation*}
				\exists v_{i_1} \dots \exists v_{i_n} \psi \qquad n \in \setN, \psi \text{ quantorenfrei}
			\end{equation*}
			\item $\forall\exists$-Formel
			\begin{equation*}
				\forall v_{i_1} \dots \forall v_{i_n} \exists v_{i_{n+1}} \dots v_{i_m}  \psi \qquad n \leq m \in \setN, \psi \text{ quantorenfrei}
			\end{equation*}
		\end{itemize}
	\end{definition}

	\begin{remark}
		$(\forall x \varphi \land \forall y \psi)$ ist nicht universell aber äquivalent zu einer universellen Formel.
	\end{remark}

	\begin{remark}
		Quantorenfreie, universelle, existentielle und $\forall\exists$-Formeln sind bis auf logische Äquivalenz abgeschlossen unter $\land, \lor$.
		
		Quantorenfreie Formeln sind abgeschlossen unter $\lnot, \rightarrow, \leftrightarrow$.
		
		$\lnot$ universell $\sim$ existentiell
		
		$\lnot$ existentiell $\sim$ universell
	\end{remark}

	\begin{lemma}
		Sei $\mathfrak{A} \subseteq \mathfrak{B}$ und $\varphi$ eine $\mathcal{L}_A$-Formel
		\begin{enumerate}
			\item Wenn $\varphi$ universell ist: $\mathfrak{B} \models \varphi \Rightarrow \mathfrak{A} \models \varphi$
			\item Wenn $\varphi$ existentiell ist: $\mathfrak{A} \models \varphi \Rightarrow \mathfrak{B} \models \varphi$
		\end{enumerate}
	\end{lemma}
	
	\begin{proof}
		\begin{enumerate}
			\item $\varphi = \varphi(\overline{a}) = \forall v_{i_1}, \dots \forall v_{i_n} \psi(\overline{v}, \overline{a})$ mit $\psi$ quantorenfrei
			
			$\mathfrak{B} \models \varphi(\overline{a})$
			
			d.h. für jedes $\overline{b} \in B$ gilt $\mathfrak{B} \models \psi(\overline{b}, \overline{a})$
			
			Insbesondere für jedes $\overline{b} \in A$ gilt $\mathfrak{B} \models \psi(\overline{b}, \overline{a})$ quantorenfrei
			
			$\mathfrak{A} \subseteq \mathfrak{B}:$ für jedes $\overline{b} \in A$ gilt $\mathfrak{A} \models \psi(\overline{b}, \overline{a})$
			
			das heißt $\mathfrak{A} \models \underbrace{\forall \overline{v} \; \psi(\overline{v}, \overline{a})}_{\varphi(\overline{a})}$
			
			\item analog
		\end{enumerate}
	\end{proof}

	\begin{lemma}[Zieglers Trennungslemma]
		Seien $T, T'$ Theorien und $\mathcal{H}$ eine Menge von $\mathcal{L}$-Aussagen mit:
		\begin{itemize}
			\item $\bot, \top \in \mathcal{H}$
			\item $\mathcal{H}$ abgeschlossen unter $\land, \lor$
		\end{itemize}
	
		Dann sind äquivalent:
		\begin{enumerate}
			\item Für $\mathfrak{A} \models T, \mathfrak{B} \models T'$ gibt es $\varphi \in \mathcal{H}$ mit $\mathfrak{A} \models \varphi, \mathfrak{B} \models \lnot \varphi$ ($\varphi$ trennt $\mathfrak{A}$ von $\mathfrak{B}$)
			\item Es gibt $\varphi \in \mathcal{H}$ mit $T \vdash \varphi, T' \vdash \lnot \varphi$  ($\varphi$ trennt $T$ von $T'$)
		\end{enumerate}
	\end{lemma}

	\begin{proof}
		\heading{2) $\Rightarrow$ 1):} klar
		
		\heading{1) $\Rightarrow$ 2)}: Sei $\mathfrak{A} \models T$ und sei $\mathcal{H}_\mathfrak{A} \coloneqq \{ \varphi \in \mathcal{H} \mid \mathfrak{A} \models \varphi \}$
		
		$\mathcal{H}_\mathfrak{A} \cup T'$ ist inkonsistent.
		
		$\mathfrak{B} \models \mathcal{H}_\mathfrak{A} \cup T'$ nach Voraussetzung existiert $\varphi \in \mathcal{H}$ mit $\mathfrak{A} \models \varphi$ (d.h. $\varphi \in \mathcal{H}_\mathfrak{A}$) und damit $\mathfrak{B} \models \lnot \varphi$ $\lightning$
		
		Kompaktheit: Es gibt $\varphi_1, \dots, \varphi_n \in \mathcal{H}_\mathfrak{A}$ mit $\{ \varphi_1, \dots, \varphi_n \} \cup T'$ ist inkonsistent. Und damit $T' \cup \{ \varphi_1 \land \dots \land \varphi_n \}$ ist inkonsistent.
		
		$\Leftrightarrow T' \vdash \lnot (\underbrace{\varphi_1 \land \dots \land \varphi_n}_{\varphi_\mathfrak{A} \in \mathcal{H}_\mathfrak{A}})$
		
		$T \cup \{ \lnot \varphi_\mathfrak{A} \mid \mathfrak{A} \models T \}$ ist inkonsistent.
		
		Kompaktheit: Es gibt $\mathfrak{A}_1, \dots, \mathfrak{A}_m \models T$ mit \begin{align*}
			&T \cup \{ \lnot \varphi_{\mathfrak{A}_1} , \dots, \lnot \varphi_{\mathfrak{A}_n} \} \text{ inkonsistent} \\
			\Leftrightarrow& T \cup \{ \lnot \varphi_{\mathfrak{A}_1} \land \dots \land \lnot \varphi_{\mathfrak{A}_n} \} \text{ inkonsistent} \\
			\Leftrightarrow& T \cup \{ \lnot (\varphi_{\mathfrak{A}_1} \lor \dots \lor \varphi_{\mathfrak{A}_n}) \} \text{ inkonsistent} \\
			\Leftrightarrow& T \vdash \underbrace{\varphi_{\mathfrak{A}_1} \lor \dots \lor \varphi_{\mathfrak{A}_n}}_{\in \mathcal{H}}
		\end{align*}
		
		Andererseits:
		\begin{equation*}
			T' \vdash \lnot \varphi_{\mathfrak{A}_1} \land \dots \land \lnot \varphi_{\mathfrak{A}_n} \qquad\sim\qquad \lnot(\varphi_{\mathfrak{A}_1} \lor \dots \lor \varphi_{\mathfrak{A}_n})
		\end{equation*}
	\end{proof}

	\heading{Notationen:} Sei $\Delta$ Menge von $\mathcal{L}$-Formeln, $\mathfrak{A}, \mathfrak{B}$ $\mathcal{L}$-Strukturen, $h: A \to B$
	\begin{equation*}
		h: \mathfrak{A} \to_\Delta \mathfrak{B} :\Leftrightarrow \text{$h$ erhält die Gültigkeit von $\Delta$-Formeln mit Parametern aus $A$}
	\end{equation*}
	Das heißt
	\begin{equation*}
		\mathfrak{A} \models \delta(\overline{a}) \Rightarrow \mathfrak{B} \models \delta(h(\overline{a}))
	\end{equation*}
	\begin{equation*}
		\Leftrightarrow \mathfrak{B}_h \models \operatorname{Th}_\Delta(\mathfrak{A}_A) = \{ \delta(\overline{a}) \mid \mathfrak{A} \models \delta(\overline{a}), \overline{a} \in A, \delta \in \Delta \}
	\end{equation*}
	
	\begin{itemize}
		\item $\Delta$ atomar: $h: \mathfrak{A} \to_\Delta \mathfrak{B}$ $\Leftrightarrow$ $h$ Homomorphismus
		\item $\Delta$ atomar, negiert atomar $h: \mathfrak{A} \to_\Delta \mathfrak{B}$ $\Leftrightarrow$ $h$ Einbettung
		\item $\Delta$ quantorenfrei $h: \mathfrak{A} \to_\Delta \mathfrak{B}$ $\Leftrightarrow$ $h$ Einbettung
		\item $\Delta$ alles $h: \mathfrak{A} \to_\Delta \mathfrak{B}$ $\Leftrightarrow$ $h$ elementar
	\end{itemize}

	\begin{lemma}
		Sei $T$ eine $\mathcal{L}$-Theorie, $\mathfrak{A}$ eine $\mathcal{L}$-Struktur. $\Delta$ abgeschlossen bezüglich $\exists, \land$ und Umbenennung von Variablen.
		
		$(v_0 \doteq c) \nsim v_1 \doteq c$
		
		Dann sind äquivalent
		\begin{enumerate}
			\item Jedes $\varphi \in \operatorname{Th}_\Delta(\mathfrak{A})$ ist konsistent mit $T$ (d.h. es existiert $\mathfrak{M}_\varphi \models T \cup \{ \varphi \}$)
			\item Es gibt $\mathfrak{B} \models T$ und $h: \mathfrak{A} \to_\Delta \mathfrak{B}$ (d.h. es existiert $\mathfrak{B} \models T \cup \operatorname{Th}_\Delta(\mathfrak{A}_A)$)
		\end{enumerate}
	\end{lemma}

	\begin{proof}
		\heading{2) $\Rightarrow$ 1)}: klar
		
		\heading{1) $\Rightarrow$ 2)}: Zeige mit Kompaktheit: $T \cup \operatorname{Th}_\Delta(\mathfrak{A}_A)$ ist konsistent.
		
		Angenommen nicht. Dann gibt es $\delta_i(a_i)$ mit $T \cup \{ \delta_1(\overline{a_1}, \dots, \delta_n(\overline{a_n})) \}$ inkonsistent
		
		Ohne Einschränkung mit $\overline{a} = \overline{a_1} \cap \dots \cap \overline{a_n}$
		\begin{equation*}
			T \cup \{ \underbrace{\delta_1(\overline{a}), \dots, \delta_n(\overline{a})}_{\text{ersetze durch $\delta(\overline{a}) = \delta_1(\overline{a}) \land \dots \land \delta_n(\overline{a})$}} \}
		\end{equation*}
		
		$\delta(\overline{a}) \in \Delta$
		
		$T \cup \{ \delta(\overline{a}) \}$ inkonsistent, das heißt $T \vdash \lnot \delta(\overline{a})$
		
		$T$ $\mathcal{L}$-Theorie, $\overline{a}$ Konstanten $\notin \mathcal{L}$
		
		Mit Logik folgt: $T \vdash \forall \overline{x} \lnot \delta(\overline{x}) \sim \lnot \underbrace{\exists \overline{x} \delta(\overline{x})}_{\in \Delta}$
		
		$\mathfrak{A} \models \delta(\overline{a}a)$ d.h. $\mathfrak{A} \models \exists \overline{x} \delta(\overline{x})$ aber $T \vdash \lnot \exists \overline{x} \delta(\overline{x})$ $\lightning$ zu 1)
	\end{proof}

	\begin{corollary}
		Betrachte $\mathcal{L}$-Struktur $\mathfrak{B}$ und $T = \operatorname{Th}(\mathfrak{B})$.
		
		\begin{align*}
			\text{Jedes $\varphi \in \operatorname{Th}_\Delta(\mathfrak{A})$}&\text{ ist konsistent mit $T$} &&\Leftrightarrow& \text{es gibt $\mathfrak{B}' \models T$ und}&\text{ $h: \mathfrak{A} \to_\Delta \mathfrak{B}'$} \\
			\Updownarrow& &&& & \\
			\operatorname{Th}_\Delta(\mathfrak{A}) &\subseteq \operatorname{Th}_\Delta(\mathfrak{B}) &&& \Updownarrow \\
			\Updownarrow& &&& & \\
			\mathfrak{A} \Rightarrow_\Delta& \mathfrak{B} &&& \text{es gibt $\mathfrak{B}' \equiv \mathfrak{B}$ }&\text{und $h: \mathfrak{A} \to_\Delta \mathfrak{B}'$}
		\end{align*}
	\end{corollary}

	\begin{theorem}
		Seien $T, T'$ $\mathcal{L}$-Theorien. Es sind äquivalent:
		\begin{enumerate}
			\item Es gibt universelle $\mathcal{L}$-Aussage $\varphi$, die $T$ von $T'$ trennt (d.h. $T \vdash \varphi, T' \vdash \lnot \varphi$)
			\item Wenn $\mathfrak{A} \models T, \mathfrak{B} \models T'$, dann ist $\mathfrak{B}$ keine Unterstruktur von $\mathfrak{A}$
		\end{enumerate}
	\end{theorem}

	\begin{proof}
		\heading{1) $\Rightarrow$ 2):} $\mathfrak{A} \models T$, also $\mathfrak{A} \models \varphi$, $\varphi$ universell.
		
		Wenn $\mathfrak{B} \subseteq \mathfrak{A}$, dann $\mathfrak{B} \models \varphi$ (Lemma) also $\mathfrak{B} \not\models T'$.
		
		\heading{$\lnot$1) $\Rightarrow$ $\lnot$2):} Trennungslemma: es gibt $\mathfrak{A} \models T, \mathfrak{B} \models T'$ und keine universelle Aussage trennt $\mathfrak{A}$ von $\mathfrak{B}$, d.h. $\mathfrak{A} \Rightarrow_\forall \mathfrak{B}$
		
		äquivalent: $\mathfrak{B} \Rightarrow_\exists \mathfrak{A}$
		
		Folgerung: Es gibt ein $\mathfrak{A}' \equiv \mathfrak{A}$ mit $h: \mathfrak{B} \to_\exists \mathfrak{A}'$
		
		Da qf $\subseteq \exists$ insbesondere $\mathfrak{B} \models T' \subseteq \mathfrak{A}' \models T$
	\end{proof}

	\begin{corollary}
		Sei $T$ $\mathcal{L}$-Theorie und $\varphi$ eine $\mathcal{L}$-Formel. Dann sind äquivalent:
		\begin{enumerate}
			\item Es gibt universelles $\psi$ mit $T \models \forall \overline{x} (\varphi(\overline{x}) \leftrightarrow \psi(\overline{x})$
			\item Falls $\mathfrak{A} \models T, \mathfrak{B} \models T, \mathfrak{A} \subseteq \mathfrak{B}$
			
			Für alle $\overline{a} \in A$ gilt: $\mathfrak{B} \models \varphi(\overline{a}) \Rightarrow \mathfrak{A} \models \varphi(\overline{a})$
		\end{enumerate}
	\end{corollary}

	\begin{proof}
		\heading{1) $\Rightarrow$ 2):}
		\begin{equation*}
			\mathfrak{B} \models \varphi(\overline{a}) \Leftrightarrow \mathfrak{B} \models \psi(\overline{a}) \Rightarrow \mathfrak{A} \models \psi(\overline{a}) \Leftrightarrow \mathfrak{A} \models \varphi(\overline{a})
		\end{equation*}
		
		\heading{2) $\Rightarrow$ 1):}
		Seien $\overline{c}$ neue Konstanten (für $\overline{a}$) und
		\begin{align*}
			\mathcal{L}_c = \mathcal{L} \cup \{ \overline{c} \} \\
			T^+ = T \cup \{ \varphi(\overline{c}) \} \\
			T^- =  T \cup \{ \lnot \varphi(\overline{c}) \}
		\end{align*}
		$\mathcal{L}_c$-Theorien
		
		2) sagt Unterstrukturen von Modellen von $T^+$ die Modelle von $T$ sind, sind Modelle von $T^+$ also nicht Modelle von $T^-$-
		
		Mit dem Satz folgt: Es gibt eine universelle $\mathcal{L}_c$ Aussage $\psi^{(c)}$ die $T^+$ von $T^-$ trennt, das heißt
		\begin{equation*}
			T^+ \vdash \psi(\overline{c}), T \vdash \lnot \psi(\overline{c})
		\end{equation*}
		Es folgt
		\begin{align*}
			T \cup \{ \varphi(\overline{c}) &\vdash \psi(\overline{c}) \} & T &\vdash \varphi(\overline{c}) \rightarrow \psi(\overline{c}) \\
			T \cup \{ \varphi(\overline{c}) \} &\vdash \psi(\overline{c}) & T &\vdash \lnot \varphi(\overline{c}) \rightarrow \lnot \psi(\overline{c})
		\end{align*}
		$\overline{c}$ kommt in $T$ vor, also:
		\begin{align*}
			T &\vdash \forall x (\varphi(x) \rightarrow \psi(x)) \\
			T &\vdash \forall x (\lnot \varphi(x) \rightarrow \lnot \psi(x))
		\end{align*}
		\begin{equation*}
			T \vdash \forall x (\varphi(x) \leftrightarrow \psi(x))
		\end{equation*}
	\end{proof}

	\begin{definition}
		$T_\forall \coloneqq = \{ \varphi \text{universelle $\mathcal{L}$-Aussage} \mid T \vdash \varphi \}$
		
		(Falls $T = T^\vdash$ dann $T_\forall = T \cap \forall$)
		
		Analog: $T_\exists, T_{\exists\forall}, \dots$
	\end{definition}

	\begin{corollary}
		$T$ ist universell, d.h. $T_\forall^\vdash = T^\vdash$. (äquivalent $T$ ist $\forall$-Axiomatisierbar, d.h. es bit $\varphi_i$ universell mit $\{ \varphi_i \mid i \in I \}^\vdash = T^\vdash$)
		
		$\Leftrightarrow$ wenn $\mathfrak{A} \subseteq \mathfrak{B}$ und $\mathfrak{B} \models T$, dann $\mathfrak{A} \models T$
	\end{corollary}
	\begin{proof}
		\heading{$\Rightarrow$} Wenn $\mathfrak{B} \models T$ und $\mathfrak{A} \subseteq \mathfrak{B}$ dann $\mathfrak{A} \models T_\forall$ also $\mathfrak{A} \models T_\forall^\vdash \supseteq T$
		
		\heading{$\Leftarrow$} Stets $T_\forall \subset T^\vdash$, also $T_\forall^\vdash \subseteq T^\vdash$.
		
		Zeige: $T^\vdash \subseteq T_\forall^\vdash$, d.h. $T_\forall \vdash T$
		
		Sei $\varphi \in T, \mathfrak{A} \subseteq \mathfrak{B} \models T$
		
		Nach Voraussetzung ist $\mathfrak{A} \models \varphi$, d.h. $\mathfrak{A} \not\models \{ \lnot \varphi \} \eqqcolon T'$.
		
		Satz $\Rightarrow$ ex. universelle $\psi$ mit $T \vdash \psi$ d.h. $\psi \in T_\forall$.
		
		$\{ \lnot \varphi \} = T' \vdash \lnot \psi (\Leftrightarrow \psi \vdash \varphi)$ insbesondere $T_\forall \vdash \varphi$.
	\end{proof}

	\heading{Dualisierung:}
	
	$\varphi$ existentiell modulo $T$ $\Leftrightarrow \forall \mathfrak{A}, \mathfrak{B} \models T, \mathfrak{A} \subseteq \mathfrak{B}, \mathfrak{A} \models \varphi(\overline{a}) \Rightarrow \mathfrak{B} \models \varphi(a)$
	
	$T$ ist existentiell, d.h. $T^\vdash = T_\exists^\vdash$ $\Leftrightarrow \mathfrak{A} \subseteq \mathfrak{B}, \mathfrak{A} \models T \Rightarrow \mathfrak{B} \models T$
	
	\begin{example}
		Die $\mathcal{L}_{HGr}$-Theorie der Gruppen ist nicht universell
		
		\begin{equation*}
			\underbrace{(\setN, +)}_\text{keine Gruppe} \subseteq \underbrace{(\setZ, +)}_\text{Gruppe}
		\end{equation*}
		
		Axiom z.B. $\exists x \forall y (y \circ x = y \land x \circ y = y)$
		
		Aber die $\{ \circ, {}^{-1}, e \} \eqqcolon \mathcal{L}_{Gr}$-Theorie der Gruppen ist universell
		
		Axiom z.B. $\forall y (y \circ e = y \land e \circ y = y)$
	\end{example}
	
	\heading{Einschub für die Allgemeinbildung}
	
	\textit{HSP-Theoriem von Birkhoff (aus der universellen Algebra)}
	
	Eine Klasse $\mathcal{K}$ von Algebren ($\mathcal{L}$-Strukturen ohne Relationen) ist eine Varietät, d.h. axiomatisiert durch Aussagen der Form $\forall \overline{x} \; \tau_1(\overline{x}) \doteq \tau_2(\overline{x})$
	
	$\Leftrightarrow$ $\mathcal{K}$ ist abgeschlossen unter:
	\begin{itemize}
		\item Homomorphen Bildern
		\item Unterstrukturen
		\item direkten Produkten
	\end{itemize}

	\textit{Elementare Strukturen}
	
	Eine Klasse von $\mathcal{L}$-Stukturen $\mathcal{K}$ ist elementar (d.h. axiomatisierbar: es gibt $\mathcal{L}$-Theorie $T$ mit $\mathcal{K} = \operatorname{Mod}(T)$)
	
	$\Leftrightarrow$ abgeschlossen unter $\equiv$ und Ultraprodukten
	
	$\Leftrightarrow$ abgeschlossen unter $\cong$, Ultraprodukten und elementaren Unterstrukturen
	
	$\Leftrightarrow$ abgeschlossen unter $\cong$, Ultraprodukten und Ultrawurzeln ($\mathfrak{A}^I/\mathcal{U} \in \mathcal{K} \Rightarrow \mathfrak{A} \in \mathcal{K}$)
	
	\textit{Satz ohne Beweis}
	
	Keisler mit GCH (generalized continuum hypothesis), Shela ohne
	
	$\mathfrak{A} \equiv \mathfrak{B}$ $\Leftrightarrow$ ex. Ultrafilter $\mathcal{U}, \mathcal{U}'$ mit $\mathfrak{A}^I/\mathcal{U} \cong \mathfrak{B}^I/\mathcal{U}'$
	
	\begin{lemma}
		$\forall \exists$-Aussagen werden unter Vereinigungen von Ketten präserviert, d.h.
		\begin{equation*}
			\mathfrak{A}_0 \subseteq \mathfrak{A}_2 \subseteq \dots \subseteq \bigcup_{i \in \omega} \mathfrak{A}_i = \mathfrak{A}_\omega
		\end{equation*}
		und $\mathfrak{A}_i \models \varphi$, $\varphi$ $\forall \exists$-Aussage, dann $\mathfrak{A}_\omega \models \varphi$ für alle $i \in \omega$.
	\end{lemma}

	\begin{proof}
		$\varphi = \forall \overline{x} \; \exists \overline{y} \; \psi(\overline{x}, \overline{y})$.
		
		Zu zeigen: für jedes $\overline{a} \in A_\omega$ gilt $\mathfrak{A}_\omega \models \exists \overline{y} \; \psi(\overline{a}, \overline{y})$
		
		Es gibt $n \in \omega$ mit $\overline{a} \in A_n$
		
		Nach Voraussetzung $\mathfrak{A}_n \models \varphi$, d.h. $\mathfrak{A}_n \models \exists \overline{y} \psi(\overline{a}, \overline{y})$ existentiell $\Rightarrow$ $\mathfrak{A}_\omega \models \exists \overline{y} \; \psi(\overline{a}, \overline{y})$
	\end{proof}

	\begin{theorem}
		$T, T'$ $\mathcal{L}$-Theorien Äquivalent sind:
		\begin{enumerate}
			\item Es gibt $\varphi$ $\forall \exists$-Aussage mit $T \vdash \varphi, T' \vdash \lnot \varphi$
			\item Falls $\mathfrak{A}_0 \subseteq \mathfrak{A}_1 \subseteq \dots$ Kette von Modellen von $T$, dann ist $\mathfrak{A}_\omega \not\models T'$
		\end{enumerate}
	\end{theorem}

	\begin{proof}
		\heading{1) $\Rightarrow 2)$}: Da $\mathfrak{A}_i \models T$ gilt insbesondere $\mathfrak{A} \models \varphi$. Da $\varphi$ $\forall\exists$, gilt nach Lemma $\mathfrak{A}_\omega \models \varphi$, somit $\mathfrak{A}_\omega \not\models T$.
		
		\heading{$\lnot 1) \Rightarrow \lnot 2)$}:
		
		Trennungslemma: Es gibt $\mathfrak{A}_0 \models T, \mathfrak{B}_0 \models T'$, nicht $\forall \exists$-trennbar d.h. $\mathfrak{A}_0 \Rightarrow_{\forall\exists} \mathfrak{B}_0$. Somit $\mathfrak{B}_0 \Rightarrow_{\exists\forall} \mathfrak{A}_0$.
		
		Einbettungslemma: Es gibt $h: \mathfrak{B}_0 \to_{\exists\forall} \mathfrak{A}_1 \equiv \mathfrak{A}_0$
		
		Ohne Einschränkung: $\mathfrak{B}_0 \subseteq \mathfrak{A}_1$, $h$ präserviert $\exists\forall$-Formeln aus $\mathcal{L}_{B_0}$
		
		$\Rightarrow \mathfrak{A}_1 \Rightarrow_{\forall\exists} \mathfrak{B}_0$ in $\mathcal{L}_{B_0}$
		
		Abschwächung: $\mathfrak{A}_1 \Rightarrow_\exists \mathfrak{B}_0$ in $\mathcal{L}_{B_0}$.
		
		Lemma: $h': \mathfrak{A}_1 \to_\exists \mathfrak{B}_1 \equiv \mathfrak{B}_0$ in $\mathcal{L}_{B_0}$
		
		Ohne Einschränkung: $\mathfrak{B}_0 \subseteq \mathfrak{A}_1 \subseteq \mathfrak{B}_1$
		
		Und damit: $\mathfrak{B}_0 \preccurlyeq \mathfrak{B}_1$
		
		\begin{equation*}
			\mathfrak{B}_0 \subseteq \mathfrak{A}_1 \subseteq \mathfrak{B}_1 \subseteq \mathfrak{A}_2 \subseteq \mathfrak{B}_2 \subseteq \dots
		\end{equation*}
		
		Ersetze $\mathfrak{B}_0$ durch $\mathfrak{B}_1$ und konstruiere analog $\mathfrak{A}_2, \mathfrak{B}_3$
		
		\begin{equation*}
			\mathfrak{A}_\omega = \bigcup_{i \in \omega} \mathfrak{A}_i = \bigcup_{i \in I} \mathfrak{B}_i = \mathfrak{B}_\omega \models T'
		\end{equation*}
	\end{proof}

	\begin{corollary}
		Äquivalent sind
		\begin{enumerate}
			\item $\varphi$ ist modulo $T$ äquivalent zu $\forall\exists$-Aussage $\psi$
			\item $\varphi$ wird unter Vereinigungen von Ketten präserviert
		\end{enumerate}
	\end{corollary}

	\begin{proof}
		\heading{$1) \Rightarrow 2)$}
		
		Wenn $\mathfrak{A}_0 \subseteq \mathfrak{A}_1 \subseteq \dots$ und $\mathfrak{A}_i \models \varphi$
		
		dann $\mathfrak{A}_i \models \psi$ (da $T \vdash (\varphi \leftrightarrow \psi)$)
		
		dann $\mathfrak{A}_\omega \models \psi$ (da $\psi \; \forall\exists$)
		
		somit $\mathfrak{A}_\omega \models \varphi$
		
		\heading{$2) \Rightarrow 1)$}
		\begin{align*}
			T^+ &= T \cup \{ \varphi \} \\
			T^- &= T \cup \{ \lnot \varphi \}
		\end{align*}
		
		2) sagt: Vereinigungen von Ketten von Modellen von $T^\vdash$ sind nicht Modelle von $T^-$.
		
		Satz: Es gibt $\forall\exists$-Aussage $\psi$ mit $T^+ \vdash \psi, T^- \vdash \lnot \psi$. Wie im universellen Fall $T \vdash (\varphi \leftrightarrow \psi)$
	\end{proof}

	\begin{corollary}
		$T$ ist induktiv d.h. unter Vereinigungen von Ketten abgeschlossen

		$\Leftrightarrow$ $T$ ist $\forall\exists$-axiomatisierbar, d.h. $T^\vdash = T_{\forall\exists}^\vdash$
	\end{corollary}

	\begin{proof}
		\heading{$\Leftarrow$}
		
		Wenn $\mathfrak{A}_0 \subseteq \mathfrak{A}_1 \subseteq \dots$ mit $\mathfrak{A}_i \models T$, dann $\mathfrak{A}_\omega \models T_{\forall\exists}$, also $\mathfrak{A}_\omega \models T$, da $T \subseteq T_{\forall\exists}^\vdash$
		
		\heading{$\Rightarrow$}
		
		Zeige $T_{\forall\exists} \vdash T$
		
		Sei $\varphi \in T$ und $\mathfrak{A}_0 \subseteq \mathfrak{A}_1 \subseteq \dots$ Kette von Modellen von $T$
		
		Nach Voraussetzung ist $\mathfrak{A}_\omega \models \varphi$, also $\mathfrak{A}_\omega \not\models \lnot \varphi$.
		
		Satz: Es gibt $\forall\exists$-Aussage $\psi$, die $T$ von $\{ \lnot \varphi \}$ trennt. $T \vdash \psi$, d.h. $\psi \in T_{\forall\exists}$
		
		$\lnot \varphi \vdash \psi$ d.h. $\psi \vdash \varphi$ insbesondere $T_{\forall\exists} \vdash \varphi$
	\end{proof}

	\begin{definition}
		Eine $\mathcal{L}$-Theorie $T$ heißt modellvollständig, wenn jede $\mathcal{L}$-Formel modulo $T$ universell ist. (äquivalent: jede $\mathcal{L}$-Formel ist modulo $T$ existentiell)
	\end{definition}

	\begin{theorem}[Test von Robinson]
		Äquivalent sind
		\begin{enumerate}
			\item $T$ ist modellvollständig
			\item Wenn $\mathfrak{M}, \mathfrak{N} \models T, \mathfrak{M} \subseteq \mathfrak{N}$ dann $\mathfrak{M} \preccurlyeq \mathfrak{N}$
			\item Wenn $\mathfrak{M}, \mathfrak{N} \models T, \mathfrak{M} \subseteq \mathfrak{N}$ und $\varphi$ $\mathcal{L}_M$-Existenzaussage (d.h. $\varphi = \exists \overline{x} \; \psi(\overline{x}, \overline{m})$, $\psi$ q.f.) dann $\mathfrak{N} \models \varphi \Rightarrow \mathfrak{M} \models \varphi$
		\end{enumerate}
	\end{theorem}
	
	\begin{definition}
		Eigenschaft 3) heißt auch: $\mathfrak{M}$ ist existenziell abgeschlossen in $\mathfrak{N}$.
		\begin{equation*}
			\mathfrak{M} \preccurlyeq_1 \mathfrak{N}
		\end{equation*}
	\end{definition}

	\begin{proof}
		\heading{2) $\Rightarrow 3)$:} trivial
		
		\heading{1) $\Rightarrow 2)$:} angenommen $\mathfrak{N} \models \varphi(\overline{m})$ $\mathcal{L}_M$-Aussage.
		
		Es gibt universelles $\psi$ mit $T \vdash \forall \overline{x} (\varphi(\overline{x}) \leftrightarrow \psi(\overline{x})$
		
		Dann $\mathfrak{N} \models \psi(\overline{m})$
		
		Da $\psi$ universell $\Rightarrow$ $\mathfrak{M} \models \psi(\overline{m})$ d.h. $\mathfrak{M} \models \varphi(\overline{m})$
		
		\heading{3) $\Rightarrow 1)$:} Sei $\varphi(\overline{x})$ eine existentielle $\mathcal{L}$-Formel. Satz über universelle Präservation: Es gibt $\psi(\overline{x})$ universell mit $T \vdash \forall \overline{x}(\varphi(\overline{x}) \leftrightarrow \psi(\overline{x}))$.
		
		Sei $\chi(\overline{x})$ beliebige Formel, zeige $\chi \sim $ univ.
		\begin{equation*}
			\chi(\overline{x}) \sim \forall \dots \forall \exists \dots \exists \forall \dots \forall \underbrace{\exists \dots \exists \rho'(\overline{x})}_{\sim \text{univ.}}
		\end{equation*}
		Zeige induktiv von hinten nach vorne durch die Formel das am Ende eine universelle Formel übrig bleibt. Dabei immer letzten Teil mit Existenzquantoren durch äquivalente universelle Formel ersetzen. Danach negieren um Allquantoren wieder zu Existenzquantoren zu machen und dann wie eben.
	\end{proof}

	\begin{lemma}
		Modellvollständige Theorien sind induktiv (d.h. $\forall\exists$-axiomatisierbar)
	\end{lemma}

	\begin{proof}
		Sei $\mathfrak{M}_0 \subseteq \mathfrak{M}_1 \subseteq \dots \subseteq \mathfrak{M}_\omega$ mit $\mathfrak{M}_i \models T$
		
		Aus $\mathfrak{M}_0 \preccurlyeq \mathfrak{M}_1 \preccurlyeq \dots$ folgt $\mathfrak{M}_\omega \models T$
	\end{proof}

	\begin{lemma}
		$T$ ist modellvollständig $\Leftrightarrow$ für jedes $\mathfrak{M}$ ist $T \cup \operatorname{Diag}(\mathfrak{M})$ eine vollständige $\mathcal{L}_M$-Theorie.
	\end{lemma}

	\begin{proof}
		\heading{$\Rightarrow$:} Seien $\mathfrak{N}_1, \mathfrak{N}_2 \models T \cup \operatorname{Diag}(\mathfrak{M})$. Zu zeigen ist $\mathfrak{N}_1 \equiv \mathfrak{N}_2$ in $\mathcal{L}_M$.
		
		$\mathfrak{N}_i \models \operatorname{Diag}(\mathfrak{M}) \Leftrightarrow$ es gibt eine Einbettung $\mathfrak{M} \hookrightarrow \mathfrak{N}_i$ d.h. ohne Einschränkung $\mathfrak{M} \subseteq \mathfrak{N}_i$.
		
		\begin{equation*}
			\mathfrak{N}_1 \models T \overset{\supseteq}{\succcurlyeq} \mathfrak{M} \models T \overset{\subseteq}{\preccurlyeq} \mathfrak{N}_2 \models T
		\end{equation*}
		
		$T$ modellvollständig $\Rightarrow$ $\subseteq = \preccurlyeq$
		
		Sei $\varphi$ $\mathcal{L}_M$-Aussage
		
		Dann $\mathfrak{N}_1 \models \varphi \Leftrightarrow \mathfrak{M} \models \varphi \Leftrightarrow \mathfrak{N}_2 \models \varphi$.
		
		\heading{$\Leftarrow$:} Seien $\mathfrak{M}, \mathfrak{N} \models T$, $\mathfrak{M} \subseteq \mathfrak{N}$.
		
		Dann $\mathfrak{M}, \mathfrak{N} \models \underbrace{T \cup \operatorname{Diag(\mathfrak{M})}}_\text{vollständig}$
		
		Sei $\varphi$ $\mathcal{L}_M$-Aussage:
		\begin{equation*}
			\mathfrak{M} \models \varphi \Leftrightarrow T \cup \operatorname{Diag}(\mathfrak{M}) \vdash \varphi \Leftrightarrow \mathfrak{N} \models \varphi
		\end{equation*}
	\end{proof}

	\begin{example}
		\begin{itemize}
			\item $\mathcal{L} = \emptyset$
			
			$T_\infty = \{ \exists^{\geq n} x \; x \doteq x \mid n \in \setN \}$ ist Modellvollständig
			
			$T = \emptyset$ ist nicht modellvollständig
			
			\item $\mathcal{L} = \{ < \}$ dichte offene lineare Ordnungen sind modellvollständig und haben Quantorenelimination
			
			\item $\mathcal{L} = \{ +, \cdot, - , 0, 1 \}$
			
			$ACF = $ Theorie der algebraisch abgeschlossenen Körper
			
			$ACF_p = $ Theorie der algebraisch abgeschlossenen Körper der Charakteristik $p$
			
			Beide Theorien sind modellvollständig und haben Quantorenelimination
			
			\item Formal reell abgeschlossene Körper $FRCF = \operatorname{Th}(\setR, +, \cdot, -, 0, 1)$ ist modellvollständig (ohne Quantorenelimination)
			
			z.B. $a,b,c \in \setR$ und $\exists x (ax^2 + bx + c \doteq 0)$ $\mathcal{L}_R$-Formel.
			
			\begin{equation*}
				\left( \setR \models \exists x (ax^2 + bx + c \doteq 0) \Leftrightarrow b^2 - 4ac \geq 0 \right)
			\end{equation*}
			
			\begin{equation*}
				\exists x (ax^2 + bx + c \doteq 0) \sim_{FRCF} \forall y (b^2 - 4ac = 0 \lor y^2 \neq -(b^2 - 4ac))
			\end{equation*}
			
			Reell abgeschlossene Körper $RCF = \operatorname{Th}(\setR, +, \cdot, -, 0, 1, \leq)$ hat Quantorenelimination
			
			\begin{equation*}
				x \leq y \sim \exists z \; z^2 = y - x \sim \forall z \; (y - x \doteq 0 \lor z^2 \neq x - y)
			\end{equation*}
		\end{itemize}
	\end{example}

	\heading{Anwendung:} Hilberts Nullstellensatz
	
	Sei $K \models ACF$ und seien $f_1, \dots, f_m \in K[X_1, \dots, X_n]$
	
	$\langle f_1, \dots, f_m \rangle = \{ f_1 g_1 + \dots + f_m g_m \mid g_i \in K[\overline{X}] \}$ das von den $f_i$ erzeugte Ideal.
	
	Wenn $\overline{a} \in K$ so dass $f_1(\overline{a}) = \dots = f_m(\overline{a})  = 0$ dann auch $h(\overline{a}) = 0$ für jedes $h \in \langle f_1, \dots, f_m \rangle$
	
	Wenn $1 \in \langle f_1, \dots, f_m \rangle$, dann existiert keine gemeinsame Nullstelle der $f_i$.
	
	Nullstellensatz: Falls $\langle f_1, \dots, f_m \rangle$ echtes Ideal, existiert $\overline{a} \in K$ mit $f_1(\overline{a}) = \dots = f_m(\overline{a}) = 0$
	
	\begin{proof}
		Sei $\mathfrak{M}$ ein maximales Ideal, das $\langle f_1, \dots, f_m \rangle$ enthält.
		
		$K \hookrightarrow \underbrace{K[\overline{X}]/\mathfrak{M}}_\text{Körper} \subseteq \underbrace{\tilde{K[\overline{X}]/\mathfrak{M}}}_\text{alg. Abschluss} = K_1$
		
		$\Rightarrow (x_1, \dots, x_n)$ ist gemeinsame Nullstelle von allen Polynomen in $\mathfrak{M}$.
		
		$K_1 \models \exists x_1 \dots \exists x_n (f(x_1, \dots, x_n) \doteq 0 \land \dots \land f_m(x_1, \dots, x_n) \doteq 0)$
		
		Da ACF Modellvollständig also $K \models \exists x_1 \dots \exists x_n (f(x_1, \dots, x_n) \doteq 0 \land \dots \land f_m(x_1, \dots, x_n) \doteq 0)$ da $K \preccurlyeq K_1$.
	\end{proof}

	\subsection{Quantorenelimination}
	
	\begin{definition}
		$T$ hat Quantorenelimination (QE), falls jede $\mathcal{L}$-Formel $\varphi(x_1, \dots, x_n)$ zu einer quantorenfreien Formel $\psi(x_1, \dots, x_n)$ äquivalent modulo $T$ ist.
		
		Wichtig: $\psi$ darf nicht mehr freie Variablen als $\varphi$ haben.
		
		Insbesondere: Jede $\mathcal{L}$-Aussage ist zu einer quantorenfreien $\mathcal{L}$-Aussage äquivalent
	\end{definition}

	\begin{remark}
		Falls $T$ QE hat und keine Konstanten in der Sprache sind, dann ist entweder $T$ inkonsistent oder $T$ ist vollständig. (Denn $\top, \bot$ sind die einzigen quantorenfreien Aussagen, $T$ konsistent, $\varphi$ $\mathcal{L}$-Aussage: $T \vdash \varphi \Leftrightarrow T \vdash (\varphi \leftrightarrow \top)$ und $T \not\vdash \varphi \Leftrightarrow (\varphi \leftrightarrow \bot)$)
	\end{remark}

	\begin{remark}
		Wenn $T$ Quantorenelimination hat ist $T$ modellvollständig, denn jede quantorenfreie-Formel ist universell.
	\end{remark}

	\begin{example}
		$\mathcal{L} = \{ p \}$ mit $p$ einstellige Relation
		
		$T = T_\infty \cup \{ \forall x p x \lor \forall x \lnot p x \}$
		
		$T$ hat keine QE
		\begin{equation*}
			T \vdash (\forall x px \leftrightarrow ?)
		\end{equation*}
		
		$T$ hat 2 Vervollständigungen:
		\begin{align*}
			T^+ &\vdash T \cup \{ \forall x p x \} \\
			T^- &\vdash T \cup \{ \forall x \lnot p x \}
		\end{align*}
		beide haben QE.
	\end{example}

	\begin{remark}[Morleyisierung]
		$T$ kann durch eine definitorische Erweiterung zu einer Theorie $T^*$ die Quantorenelimination hat gemacht werden.
		
		Sei $T$ $\mathcal{L}$-Theorie. Für jede $\mathcal{L}$-Formel $\varphi(x_1, \dots, x_n)$ wähle neues $n$-stelliges Relationszeichen $R_\varphi$ und setze
		\begin{equation*}
			\mathcal{L}^* \coloneqq \mathcal{L} \cup \{ R_\varphi \mid \varphi \text{ $\mathcal{L}$-Formel} \}
		\end{equation*}
		$T^*$ ist dann eine $\mathcal{L}^*$-Theorie
		\begin{equation*}
			T \cup \{ \forall x_1, \dots, \forall x_n (R_\varphi x_1 \dots x_n \leftrightarrow \varphi(x_1, \dots, x_n)) \}
		\end{equation*}
		$T^*$ hat Quantorenelimination.
		Jedes Modell $\mathfrak{M}$ von $T$ wird auf eindeutige Weise zu Modell von $T^*$.
	\end{remark}

	\begin{proof}
		Zweiter Teil: Damit $\mathfrak{M}$ zu Modell $\mathfrak{M}^*$ von $T^*$ wird, muss gelten $(m_1, \dots, m_n) \in R_\varphi^{\mathfrak{M}^*} \Leftrightarrow \mathfrak{M} \models \varphi(m_1, \dots, m_n)$.
		
		Erster Teil: Jede $\mathcal{L}$-Formel $\varphi$ ist modulo $T^*$ quantorenfrei, nämlich zu $R_\varphi$.
		
		Sei $\psi$ eine $\mathcal{L}^*$-Formel. Dann ist $\psi$ modulo äquivalent zu einer $\mathcal{L}$-Formel $\psi_*$, die aus $\psi$ entsteht, indem jedes $R_\varphi$ durch $\varphi$ ersetzt wird. Dann
		\begin{equation*}
			\psi \sim_{T^*} \psi_* \sim_{T^*} \underbrace{R_{\psi_*}}_\text{q.f. $\mathcal{L}^*$-Formel}
		\end{equation*}
	\end{proof}

	Manche Eigenschaften bleiben beim Übergang $T \leadsto T^*$ erhalten manche nicht:
	\begin{itemize}
		\item Erhalten werden
		\begin{itemize}
			\item vollständig
			\item $\kappa$-Kategorisch
		\end{itemize}
		\item Nicht erhalten werden
		\begin{itemize}
			\item Arten der Axiomatisierung
			\item nicht modellvollständig
			\item Substrukturen und Homomorphismen
		\end{itemize}
	\end{itemize}

	\begin{example}
		$\mathcal{L} = \emptyset, T_\infty = \{ \exists^{\geq n} x \; x \doteq x \mid n \in \setN \}$
		
		$\mathfrak{M} \models T_\infty$: $\mathfrak{M}$ ist unendliche Menge ohne Struktur
		
		$T_\infty$ hat Quantorenelimination.
		
		$X \subseteq M^n$ heißt definierbar, falls es eine $\mathcal{L}$-Formel $\varphi(x_1, \dots, x_n)$ gibt mit
		\begin{equation*}
			X = \{ (m_1, \dots, m_n) \mid \mathfrak{M} \models \varphi(m_1, \dots, m_n) \}
		\end{equation*}
		
		Zwei Formeln $\varphi, \psi$ sind modulo $T$ äquivalent $\Leftrightarrow$ in allen Modellen definieren sie die gleiche Menge.
		
		\heading{Atomare Formeln:}
		\begin{itemize}
			\item $x \doteq x$ definiert $M$
			\item $x \doteq y$ definiert die Diagonale in $M \times M$ $\Delta_M = \{ (m,m) \mid m \in M \}$
		\end{itemize}
	
		\heading{Negation:} $\lnot \varphi$ definiert das Komplement der von $\varphi$ definierten Menge
		
		\heading{Konjunktion:} definiert Schnitt, falls gleiche Variablen bzw. das Produkt falls disjunkte Variablen
		
		z.B. $x_1 \doteq x_2 \land x_3 \doteq x_4$ definiert $\underbrace{\Delta_M \times \Delta_M}_{\{ (x_1, x_2, x_3, x_4) \mid x_1 = x_2, x_3 = x_4 \}} \subseteq M^4$
		
		$x_1 \doteq x_2 \land x_3 \doteq x_3 \land x_4 \doteq x_4$ definiert $\Delta_M \times M^2$
		
		\heading{$\Exists$:} Projektion: Wenn $\varphi(x,y)$ eine Fläche definiert definiert $\exists y \varphi(x,y)$ die Projektion der Fläche auf die $x$-Achse.
		
		Hier kann man sehen:
		\begin{equation*}
			\left\{ \left\{ (m_1, \dots, m_n) \middle| \substack{m_i = m_j \; (i,j) \in P \\ m_k \neq m_l \; (k,l) \in N} \right\} \middle| P,N \subseteq \{1,dots, n \}^2  \right\}
		\end{equation*}
		ist die Familie der in $\mathfrak{M}$ definierbaren Mengen. Beobachtung: Sind alle q.f.-definierbar.
	\end{example}

	\begin{example}
		$K$ algebraisch abgeschlossener Körper
		
		$X \subseteq K^n$ ist definierbar $\Leftrightarrow$ $X$ ist Boolesche Kombination von Nullstellenmengen von Polynomen in $K[x_1, \dots, X_n]$.
	\end{example}

	\begin{example}
		Ordnungen
		\begin{itemize}
			\item offene dichte lineare Ordnungen haben QE in $\mathcal{L} = \{ < \}$
			\item dichte lineare Ordnungen mit Endpunkten hat keine QE in $\mathcal{L} = \{ < \}$
			
			$\forall y (x \doteq y \lor x < y)$ definiert das Minimum, und ist quantorenfrei nicht definierbar. Aber hat QE in $\mathcal{L}^+ = \{ <, c_\text{min}, c_\text{max} \}$
			\item $\operatorname{Th}(\setZ, <)$
			
			Die definierbaren Teilmengen von $\setZ$ sind $\emptyset, \setZ$. Aber in $\setZ^2$ gibt es nicht q.f. definierbare Teilmengen z.B. $\{ (x_1, x_2) \mid \text{$x_2$ ist direkter Nachfolger von $x_1$} \}$ und wird definiert durch $x_1 < x_2 \land \lnot \exists y (x_1 < x \land x < x_2)$
			
			\item $(\setQ, <)$ hat EQ d.h. insbesondere die definierbaren Teilmengen von $\setQ^2$ sind
			\begin{align*}
				\Delta_\setQ &= \{ (q,q) \mid q \in \setQ \} \\
				<_\setQ &= \{ (q,r) \mid q < r \} \\
				>_\setQ &= \{ (q,r) \mid q > r \} \\
				\emptyset \\
				\Delta_\setQ \cup <_\setQ \\
				\Delta_\setQ \cup >_\setQ \\
				<_\setQ \cup >_\setQ &= \setQ^2 \setminus \Delta_\setQ \\
				\setQ^2
			\end{align*}
			\item In $\mathcal{L} = \{ +, \cdot, -, 0, 1 \}$ haben QE: $ACF, ACF_0, ACF_p$
			
			$FRCF = \operatorname{Th}(\setR, +, \cdot, -, 0, 1)$ nicht
		\end{itemize}
	\end{example}

	\begin{definition}
		Eine primitive Existenzformel ist eine Formel von der Form
		\begin{equation*}
			\exists y \; \varphi(x_1, \dots, x_n, y)
		\end{equation*}
		Mit $\varphi$ ist Konjunktion von atomaren und negiert atomaren Formeln.
	\end{definition}

	\begin{lemma}
		$T$ hat QE $\Leftrightarrow$ jede primitive Existenzformel ist modulo $T$ quantorenfrei.
	\end{lemma}

	\begin{proof}
		\heading{$\Rightarrow$:} klar
		
		\heading{$\Leftarrow$:} Einfache Existenzformel
		\begin{equation}
			\exists y \psi(\overline{x}, y) \sim \exists y \bigvee_j \bigwedge_i (\lnot) \psi_{ij}(\overline{x}, y) \sim \bigvee \underbrace{\exists y \bigwedge_i (\lnot) \psi_{ij}(\overline{x}, y)}_\text{primitiv}
 		\end{equation}
 		nach Voraussetzung $\sim_T \bigvee_j q.f.$
 		
 		Einfache Allformel: $\varphi = \forall y \psi(\overline{x}, y)$ dann ist $\lnot \varphi \sim \exists
 		 y \lnot \psi(\overline{x}, y)$ einfache Existenzformel.
 		 
 		$\sim_T \chi$ q.f.
 		
 		$\Rightarrow \varphi \sim_T \lnot \chi$ quantorenfrei
 		
 		Beliebige Formel: Quantor für Quantor regeln Anwenden
	\end{proof}

	\begin{theorem}
		Sei $T$ $\mathcal{L}$-Theorie. Äquivalent sind:
		\begin{enumerate}
			\item $T$ hat QE
			\item Für alle Modelle $\mathfrak{M}, \mathfrak{N} \models T$ mit gemeinsamer Unterstruktur $\mathfrak{A}$ gilt $\mathfrak{M}_A \equiv \mathfrak{N}_A$, das heißt
			\begin{equation*}
				\mathfrak{M} \models \varphi(\overline{a}) \Leftrightarrow \mathfrak{N} \models \varphi(\overline{a})
			\end{equation*}
			für $\mathcal{L}$-Formeln $\varphi$ und $\overline{a} \in A$
			
			Mit anderen Worten $T \cup \operatorname{Diag}(\mathfrak{A})$ ist vollständig für Unterstrukturen $\mathfrak{A}$ von Modellen von $T$ ($T$ ist substrukturvollständig)
			\item Für alle $\mathfrak{M}, \mathfrak{N} \models T$ und endlich erzeugte Unterstrukturen $\mathfrak{A} = \langle a_1, \dots, a_n \rangle$ und jede primitive Existenzformel $\varphi(\overline{x}) = \exists y \psi(\overline{x}, y)$ und $\overline{a} \in A$ gilt:
			\begin{equation*}
				\mathfrak{M} \models \varphi(\overline{a}) \Rightarrow \mathfrak{N} \models \varphi(\overline{a})
			\end{equation*}
			Falls $\mathcal{L}$ keine Konstanten enthält, ist in 2) und 3) auch die leere Struktur $\emptyset$ als Unterstruktur zugelassen.
		\end{enumerate}
	\end{theorem}

	\begin{proof}
		\heading{$2) \Rightarrow 3)$:} klar
		
		\heading{$1) \Rightarrow 2)$:} Wegen EQ gilt: $T \models \forall \overline{x}(\varphi(\overline{x}) \leftrightarrow \chi(\overline{x})$ mit q.f. $\chi$.
		
		\begin{equation*}
			\mathfrak{M} \models \varphi(\overline{a}) \Leftrightarrow \mathfrak{M} \models \chi(\overline{a}) \Leftrightarrow \mathfrak{A} \models \chi(\overline{a}) \Leftrightarrow \mathfrak{N} \models \varphi(\overline{a})
		\end{equation*}
		
		\heading{$3) \Rightarrow 1)$:} Es reicht zu zeigen, dass primitive Existenzformeln modulo $T$ q.f. sind. Sei $\varphi(\overline{x})$ primitive Existenzformel und seien
		\begin{equation*}
			T^+ \coloneqq T \cup \{ \varphi(\overline{c}) \} \qquad T^- \coloneqq T \cup \{ \lnot \varphi(\overline{c}) \}
		\end{equation*} $\mathcal{L}_C$-Theorien wobei $\mathcal{L}_C = \mathcal{L} \cup \{ \overline{c} \}$.
		
		Wie bei der universellen Präservation reicht es zu zeigen, dass $T^+$ quantorenfrei von $T^-$ getrennt werden kann.
		
		Trennungslemma: Es reicht zu zeigen: Wenn $\mathfrak{M}, \mathfrak{N} \models T, \mathfrak{M} \models T^+$ und $\mathfrak{N}$ nicht quantorenfrei trennbar von $\mathfrak{M}$, dann $\mathfrak{N} \models T^+$.
		
		$\mathfrak{A}^\mathfrak{M} = \langle c_1^\mathfrak{M}, \dots, c_n^\mathfrak{M} \rangle \subseteq \mathfrak{M}$
		
		$\mathfrak{A}^\mathfrak{N} = \langle c_1^\mathfrak{N}, \dots, c_n^\mathfrak{N} \rangle \subseteq \mathfrak{N}$
		
		Zeige $\mathfrak{A}^\mathfrak{M} \cong \mathfrak{A}^\mathfrak{N}$ (Dann fertig, weil ohne Einschränkung dann $\mathfrak{A}^\mathfrak{M}$ gemeinsame Unterstruktur $\mathfrak{M} \models \varphi(\overline{a})$ also $\mathfrak{N} \models \varphi(\overline{a})$)
		
		\begin{equation*}
			\mathfrak{A}^\mathfrak{M} = \{ \tau^\mathfrak{M} \mid \text{$\tau$ geschlossene $\mathcal{L}_{\overline{C}}$-Terme} \} = \{ \tau^\mathfrak{M}(\overline{a}) \mid \text{$\tau$ $\mathcal{L}$-Terme mit $\leq n$ freien Variablen} \}
		\end{equation*}
		
		\begin{equation*}
			\mathfrak{A}^\mathfrak{N} = \{ \tau^\mathfrak{N} \mid \text{$\tau$ geschlossene $\mathcal{L}_{\overline{C}}$-Terme} \} = \{ \tau^\mathfrak{N}(\overline{b}) \mid \text{$\tau$ $\mathcal{L}$-Terme mit $\leq n$ freien Variablen} \}
		\end{equation*}
		
		Idee: Definiere
		\begin{align*}
			\alpha: \mathfrak{A}^\mathfrak{M} &\to \mathfrak{A}^\mathfrak{N} \\
			\tau^\mathfrak{M}(\overline{a}) &\mapsto \tau^\mathfrak{N}(\overline{b})
		\end{align*}
		
		Zeige: Wohldefiniert, injektiv, surjektiv, starker Homomorphismus
		\begin{itemize}
			\item Wohldefiniert?
			\begin{equation*}
				\tau_1^\mathfrak{M}(\overline{a}) = \tau_2^\mathfrak{M}(\overline{a}) \Leftrightarrow \mathfrak{M} \models (\tau_1(\overline{x}) = \tau_2(\overline{x}))\frac{\overline{a}}{\overline{x}} \Rightarrow \mathfrak{N} \models (\tau_1(\overline{x}) = \tau_2(\overline{x}))\frac{\overline{b}}{\overline{x}} \Leftrightarrow \tau_1^\mathfrak{N}(\overline{B}) ? \tau_2^\mathfrak{N}(\overline{b})
			\end{equation*}
		\end{itemize}
	\end{proof}

	\begin{corollary}
		Angenommen es gilt: Für alle $\mathfrak{M}_1, \mathfrak{M}_2 \models T$ mit $\langle a_1, \dots, a_n \rangle \subseteq \mathfrak{M}_1, \mathfrak{M}_2$ und $b \in M_1$ existiert ein $b' \in \mathfrak{M}_2^* \succcurlyeq \mathfrak{M}_2$ mit
		\begin{equation*}
			\mathfrak{M}_1 \supseteq \langle a_1, \dots, a_n, b \rangle \cong \langle a_1, \dots, a_n, b' \rangle \subseteq \mathfrak{M}_2^*
		\end{equation*}
		Dann hat $T$ Quantorenelimination.
	\end{corollary}

	\begin{proof}
		Sei $\exists x \varphi(x,\overline{y})$ primitive Existenzformel und $\mathfrak{M} \models \varphi(b, \overline{a})$. Also $\mathfrak{M}_1 \supseteq \langle a_1, \dots, a_n, b \rangle \models \varphi(b, \overline{a})$.
		
		$\Rightarrow \mathfrak{M}_2^* \supseteq \langle a_1, \dots, a_n, b' \rangle \models \varphi(b', \overline{a})$
		
		$\Rightarrow \mathfrak{M}_2^* \models \varphi(b', \overline{a})$ d.h. $\mathfrak{M}_2^* \models \exists x \varphi(x, \overline{a})$.
		
		Und da $\mathfrak{M}_2 \preccurlyeq \mathfrak{M}_2^*$ gilt $\mathfrak{M}_2 \models \exists x \varphi(x, \overline{a})$.
	\end{proof}

	\begin{example}
		\begin{itemize}
			\item $DLO$ hat QE
			\item $RCF = \operatorname{Th}(\setR, +, \cdot, -, 0, 1, <)$ hat QE (braucht etwas Algebra, siehe Buch von Prestel)
			\item $ACF, ACF_0, ACF_p$ hat QE:
			
			Seien $\mathfrak{K}_1, \mathfrak{K}_2 \models ACF$ und $\mathfrak{A} = \langle a_1, \dots, a_n \rangle \subseteq \mathfrak{K}_1, \mathfrak{K}_2$.
			
			$\mathfrak{A}$ ist Unterstruktur, d.h. Unterring.
			
			Aus gemeinsamer Unterstruktur folgt das $\operatorname{char}(\mathfrak{K}_1) = \operatorname{char}(\mathfrak{K}_2)$.
			
			Sei $b \in \mathfrak{K}_1$ und $\mathfrak{Q}_1 = \operatorname{Quot}(\mathfrak{A})$ sowie $\mathfrak{Q}_2 = \operatorname{Quot}(\mathfrak{A})$ und $\tilde{\mathfrak{Q}_1}, \tilde{\mathfrak{Q}_2}$ der jeweilige Algebraische Abschluss von $\mathfrak{Q}_i$ in $\mathfrak{K}_i$.
			
			Da $\mathfrak{K}_1$ algebraisch abgeschlossen, ist auch $\tilde{\mathfrak{Q}_1}$ algebraisch abgeschlossen.
			
			\textit{Algebra:} Der Algebraische Abschluss ist bis auf Isomorphie eindeutig bestimmt also $\alpha: \tilde{\mathfrak{Q}_1} \cong \tilde{\mathfrak{Q}_2}$.
			
			\heading{1. Fall:} $b$ ist algebraisch über $\mathfrak{A}$. Dann finde $b' \in \mathfrak{K}_2$, nämlich $\alpha(b)$.
			
			\heading{2. Fall:} $b$ ist transzendent über $\mathfrak{A}$.
			
			\textit{Algebra:} Mit $\tilde{Q_1}(b)$ kleinster Unterkörper von $\mathfrak{K}_1$ der $\tilde{Q_1}$ und $b$ enthält und $\tilde{Q}(X)$ dem rationalen Funktionenkörper gilt $\tilde{Q_1}(b) \cong \tilde{Q_1}(X)$.
			
			Finde $b' \in \mathfrak{K}_2^* \succcurlyeq \mathfrak{K}_2$ transzendent über $\tilde{Q_2}$ (bzw. $\mathfrak{A}$). Dann fertig da
			\begin{equation*}
				\tilde{Q_2}(b') \cong \tilde{Q_2}(X) \cong \tilde{Q_1}(X) \cong \tilde{Q_1}(b)
			\end{equation*}
			$b'$ existiert mit Kompaktheit:
			\begin{equation*}
				T_c \coloneqq \operatorname{Th}({\mathfrak{K}_2}_{\mathfrak{K}_2}) \cup \{ \lnot P(c) \doteq 0 \mid P \in \tilde{Q_2}[X], P \neq 0 \}
			\end{equation*}
			ist eine $\mathcal{L}_{\mathfrak{K}_2 \cup \{ c \}}$-Theorie mit $c$ neuer Konstante.
			
			Zeige $T_c$ ist endlich erfüllbar und zwar in $\mathfrak{K}_2$: Für jedes Polynom $P$ ist $\{ x \mid P(x) = 0 \}$ endlich aber $\mathfrak{K}_2$ ist unendlich.
		\end{itemize}
	\end{example}

	\begin{definition}
		Sei $T$ $\mathcal{L}$-Theorie
		\begin{enumerate}
			\item Eine $\mathcal{L}$-Struktur $\mathfrak{A}$ heißt Primstruktur von $T$ wenn sie sich in jedes Modell von $T$ einbetten lässt.
			\item Ein Modell von $T$ $\mathfrak{A}$ heißt algebraisches Primmodell von $T$, wenn sich $\mathfrak{A}$ in jedes Modell von $T$ einbetten lässt.
			\item Ein Modell von $T$ $\mathfrak{A}$ heißt elementares Primmodell von $T$, wenn sich $\mathfrak{A}$ in jedes Modell von $T$ elementar einbetten lässt.
		\end{enumerate}
	\end{definition}

	\begin{theorem}""
		\begin{enumerate}
			\item Falls $T$ eine Primstruktur hat und QE hat, dann ist $T$ vollständig
			\item Falls $T$ ein algebraisches Primmodell hat und modellvollständig ist, dann ist $T$ vollständig.
			\item Falls $T$ ein elementares Primmodell hat, dann ist $T$ vollständig.
		\end{enumerate}
	\end{theorem}

	\begin{proof}
		\heading{3):} $\mathfrak{A} \preccurlyeq \mathfrak{M}_1, \mathfrak{M}_2$ und $\mathfrak{M_1}, \mathfrak{M_2} \models T$.
		
		$\Rightarrow \mathfrak{M_1} \equiv \mathfrak{A} \equiv \mathfrak{M_2}$
		
		\heading{2):} $\mathfrak{A} \subseteq \mathfrak{M}_1, \mathfrak{M}_2$ und $\mathfrak{M_1}, \mathfrak{M_2} \models T$.
		
		Da $T$ modellvollständig folgt $\mathfrak{A} \preccurlyeq \mathfrak{M}_1, \mathfrak{M}_2$ und damit wie bei 3)
	\end{proof}

	\begin{example}""
		
		\begin{tabular}{|c|c|c|c|}
			\hline 
			& existiert nicht & eindeutig & nicht eindeutig \\ 
			\hline 
			Primstrukturen & \shortstack{Theorie der Körper \\ \\ $\aleph_0$ unabh. Prädikate}  & \shortstack{Theorie der Körper \\ der Charakteristik \\ $p \neq 0$ \\ \\ Theorie der Ord. \\ \\ $\operatorname{Th}(\setN, +, \cdot, 0, 1)$} & \shortstack{Theorie der Körper \\ der Charakteristik $0$ \\ \\ DLO} \\ 
			\hline 
			alg. Primmodelle & \shortstack{Theorie der Körper \\ \\ $\aleph_0$ unabh. Prädikate} & DLO & \shortstack{Äquivalenzrelationen \\ mit unendlich vielen \\ unendlichen Klassen} \\ 
			\hline 
			elem. Primmodelle & \shortstack{Theorie der Körper \\ \\ $\aleph_0$ unabh. Prädikate} & DLO & ? \\ 
			\hline 
		\end{tabular}
	
		\heading{$\aleph_0$ unabhängige Prädikate}
		
		$\mathcal{L} = \{ P_i \mid i \in \omega \}$ $P_i$ einstellige Relationszeichen
		
		\begin{equation*}
			T \coloneqq \left\{ \exists x \left( \bigwedge_{j=0}^n P_{i_j} x \land \bigwedge_l=0^m \lnot P_{k_l} x \right) \middle| n,m \in \setN, \{i_0, \dots, i_n \} \cap \{ k_0, \dots, k_m \} = \emptyset, i_j, k_l \in \setN  \right\}
		\end{equation*}
		
		$m \in M$ definiert Funktion $\chi_m: \setN \to \{ 0,1 \}$ durch $\chi_m(i) = 1 \Leftrightarrow m \in P_i^M$. Löwenheim Skolem: es gibt abzählbare Modelle $\mathfrak{M} \subseteq \mathfrak{M}'$ mit $\mathfrak{M}, \mathfrak{M}' \models T$, dann muss für jedes $m \in M$ die Funktion $\chi_m$ in $\mathfrak{M}'$ realisiert sein.
		
		Wenn $m \in$ Primstruktur, müsste also $\chi_m$ in jedem Modell von $T$ realisiert sein.	
	\end{example}

	\begin{definition}
		Sei $T$ $\mathcal{L}$-Theorie. Eine $\mathcal{L}$-Theorie $T^*$ heißt Modellbegleiter von $T$ falls
		\begin{enumerate}
			\item $T_\forall = T_\forall^*$, d.h. jedes Modell von $T$ lässt sich in ein Modell von $T^*$ einbetten und umgekehrt.
			\item $T^*$ ist modellvollständig
		\end{enumerate}
	\end{definition}

	\begin{theorem}[Robinson]
		Bis auf Äquivalenz von Theorien ist der Modellbegleiter eindeutig bestimmt, sofern er existiert.
	\end{theorem}

	\begin{proof}
		Seien $T^*, T^+$ Modellbegleiter von $T$. Sei $\mathcal{A}_0 \models T^*$.
		
		$\mathfrak{A}_0 \models T \subseteq \mathfrak{M}_0 \models T \mathfrak{B}_0 \models T^+ \subseteq \mathfrak{N}_0 \models T \subseteq \mathfrak{A}_1 \models T^* \subseteq \mathfrak{M}_1 \models T \subseteq \mathfrak{B}_1 \models T^+ \subseteq \dots$
		
		$\Rightarrow \mathbb{A}_0 \preccurlyeq \mathfrak{A}_1$ und $\mathfrak{B}_0 \preccurlyeq \mathfrak{B}_1$, da $T^*, T^+$ modellvollständig sind.
		
		$\mathfrak{A}_0 \preccurlyeq \bigcup_{i \in \omega} \mathfrak{A}_i = \bigcup_{i \in \omega} \mathfrak{B}_i \models T^+$, somit $\mathfrak{A}_0 \models T^+$ und Somit $\operatorname{Mod}(T^*) = \operatorname{Mod(T^+)}$.
	\end{proof}

	\begin{example}
		\begin{itemize}
			\item $ACF$ ist Modellbegleiter der Theorie der Körper
			\item $FRCF = \operatorname{Th}(\setR, +, \cdot, - , 0, 1)$ ist Modellbegleiter der formal reellen Körper (= anordenbare Körper)
			\item $RCF = \operatorname{Th}(\setR, +, \cdot, -, 0, 1, <)$ ist Modellbegleiter der Theorie der angeordneten Körper
			\item $DLO$ ist Modellbegleiter der Theorie der Ordnungen
			\item Theorie des Zufallsgraphen ist Modellbegleiter der Theorie der Graphen
			\item Theorie der Gruppen in $\mathcal{L} = \{ \circ, {}^{-1}, e \}$ hat keinen Modellbegleiter
			\item $DCF_0$ differentiell abgeschlossene Körper ist der Modellbegleiter der Theorie der differentiellen Körper
		\end{itemize}
	\end{example}

	\begin{remark}
		Falls $T$ modellvollständig, dann ist $T$ Modellbegleiter von $T_\forall$ (bzw. von jeder Theorie $T'$ mit $T_\forall \subseteq T' \subseteq T$).
		
		z.B. $ACF$ ist Modellbegleiter der Theorie der Integritätsbereiche (=Unterstrukturen der Körper in der Ringsprache)
	\end{remark}

	\begin{definition}
		Eine $\mathcal{L}$-Struktur $\mathfrak{A}$ heißt $T$-existenziell abgeschlossen, falls
		\begin{itemize}
			\item $\mathfrak{A} \models T_\forall$ (d.h. es gibt $\mathfrak{M} \models T$ mit $\mathfrak{A} \subseteq \mathfrak{M}$)
			\item $\mathfrak{A}$ ist existentiell abgeschlossen in jedem Modell von $T$, d.h. falls $\mathfrak{A} \subseteq \mathfrak{M} \models T$, dann $\mathfrak{A} \preccurlyeq_1  \mathfrak{M}$: Also falls $\mathfrak{M} \models \varphi(\overline{a})$, $\varphi(\overline{a})$ existentielle $\mathcal{L}_A$-Formel, dann $\mathfrak{A} \models \varphi(\overline{a})$
		\end{itemize}
	\end{definition}

	\begin{theorem}
		Sei $T_\forall^* = T_\forall$. Dann $T^*$ ist Modellbegleiter von $T$ $\Leftrightarrow$ alle Modelle von $T^*$ sind $T$-existentiell abgeschlossen.
	\end{theorem}

	\begin{proof}
		\heading{$\Rightarrow$:} Sei $\mathfrak{A} \subseteq \mathfrak{M} \models T$. Dann $\mathfrak{A} \subseteq \mathfrak{M} \hookrightarrow \mathfrak{B} \models T^*$.
		
		$\mathfrak{A} \preccurlyeq \mathfrak{B}$ da $T^*$ modellvollständig.
		
		Sei nun $\mathfrak{M} \models \varphi(\overline{a})$. Dann $\mathfrak{B} \models \varphi(\overline{a})$ also auch $\mathfrak{A} \models \varphi(\overline{a})$.
		
		\heading{$\Leftrightarrow$:} Sei $\mathfrak{A} \models T^*, \mathfrak{B} \models T^*, \mathfrak{A} \subseteq \mathfrak{B}$. Zeige $\mathfrak{A} \preccurlyeq \mathfrak{B}$.
		
		Dann $\mathfrak{A} \subseteq \mathfrak{B} \hookrightarrow \mathfrak{M} \models T$ mit $T^* \supseteq T_\forall$ und $\mathfrak{A} \preccurlyeq_1 \mathfrak{M}$ nach Voraussetzung.
		
		Sei $\mathfrak{B} \models \varphi(\overline{a})$, $\varphi$ existentiell. Also $\mathfrak{M} \models \varphi(\overline{a})$ und somit auch $\mathfrak{A} \models \varphi(\overline{a})$, da $\mathfrak{A} \preccurlyeq_1 \mathfrak{M}$. Robinsons Test!
	\end{proof}

 	\begin{theorem}
 		Falls es eine $\mathcal{L}$-Theorie $T^*$ gibt, so dass $\operatorname{Mod}(T^*) = \{ \mathfrak{M} \models T_\forall \mid \mathfrak{M} \text{ $T$-ex. abgeschlossen} \}$. Dann ist $T^*$ Modellbegleiter von $T$.
 	\end{theorem}
 
 	\begin{proof}[Beweisidee]
 		\begin{itemize}
 			\item $T_\forall \subseteq T^*$ nach Voraussetzung
 			\item Umkehrung: zeige, dass sich jedes Modell von $T_\forall$ zu einer $T$-existentiell abgeschlossenen Struktur erweitern lässt. (dann $T_\forall = T_\forall^*$)
 			\item Modellvollständigkeit folgt mit dem vorherigen Satz
 		\end{itemize}
 	\end{proof}
 
 	\begin{definition}
 		Theorie $T$ hat Amalgamierungseigenschaft (AP), falls jedes Diagramm
 		
 		\begin{center}
 			\begin{tikzpicture}
	 			\node(R2) at (-2,0){$\mathfrak{M_1} \models T$};
	 			\node(R3) at (2, 0){$\mathfrak{M_2} \models T$};
	 			\node(R4) at (0, -2){$\mathfrak{M_0} \models T$};
	 			
	 			\draw[right hook->] (R4) -- (R2);
	 			\draw[right hook->] (R4) -- (R3);
 			\end{tikzpicture}
 		\end{center}
 	
		sich zu einem kommutativen Diagramm
		
		\begin{center}
			\begin{tikzpicture}
				\node(R1) at (0,2){$\mathfrak{M_3} \models T$};
				\node(R2) at (-2,0){$\mathfrak{M_1} \models T$};
				\node(R3) at (2, 0){$\mathfrak{M_2} \models T$};
				\node(R4) at (0, -2){$\mathfrak{M_0} \models T$};
				
				\draw[right hook->] (R2) -- (R1) node[midway,above]{$g_1$};
				\draw[right hook->] (R4) -- (R2) node[midway,left]{$f_1$};
				\draw[right hook->] (R4) -- (R3) node[midway,below]{$f_2$};
				\draw[right hook->] (R3) -- (R1) node[midway,right]{$g_2$};
			\end{tikzpicture}
		\end{center}
	
		vervollständigen lässt. Kommutativ heißt $g_1 \circ f_1 = g_2 \circ f_2$.
 	\end{definition}
 
 	\begin{theorem}
 		Falls $T$ universell ist und AP hat, und falls $T^*$ Modellbegleiter von $T$ dann hat $T^*$ QE. (In diesem Fall heißt $T^*$ auch Modellvervollständigung von $T$).
 	\end{theorem}
 
 	\begin{proof}""
 		\begin{center}
 			\begin{tikzpicture}
	 			\node(R1) at (0,4){$\mathfrak{M} \models T^*$};
	 			\node(R2) at (0,2){$\mathfrak{M}' \models T$};
	 			\node(R3) at (-2, 0){$\mathfrak{M_1}' \models T$};
	 			\node(R4) at (-2, -2){$\mathfrak{M_1} \models T^*$};
	 			\node(R5) at (0,-4){$\mathfrak{A} \models T_\forall^* = T_\forall = T$};
	 			\node(R6) at (2, -2){$\mathfrak{M_2} \models T^*$};
	 			\node(R7) at (2, 0){$\mathfrak{M_2}' \models T$};
	 			
	 			\draw[right hook->] (R2) -- (R1);
	 			\draw[right hook->] (R3) -- (R2);
	 			\draw[right hook->] (R4) -- (R3);
	 			\draw[right hook->] (R6) -- (R7);
	 			\draw[right hook->] (R7) -- (R2);
	 			\draw[right hook->] (R5) -- (R4) node[midway,right]{$\supseteq$};
	 			\draw[right hook->] (R5) -- (R6) node[midway,right]{$\subseteq$};
	 			\draw[] (R1) to [out=180,in=160] (R4);
	 			\draw[] (R1) to [out=0,in=20] (R6);
	 			
	 			\node(S1) at (-4, 1.5){$\preccurlyeq$};
	 			\node(S1) at (4, 1.5){$\succcurlyeq$};
 			\end{tikzpicture}
 		\end{center}
 
 	
 		$\mathfrak{M}'$ existiert wegen Amalgamierungseigenschaft. Die Einbettungen von Modellen von $T^*$ in Modelle von $T$ funktioniert wegen der Modellbegleitereigenschaft.
 		
 		Also $\mathfrak{M}_i \equiv \mathfrak{M}$ in $\mathcal{L}_{M_i}$ insbesondere $\mathfrak{M}_i \equiv \mathfrak{M}$ in $\mathcal{L}_A$. Kriterium für QE!
 	\end{proof}
 
 	\begin{remark}
 		\begin{equation*}
	 		T^* \text{ hat QE} \Leftrightarrow T^* \cup \operatorname{Diag}(\mathfrak{A}) \text{ ist vollständig für alle $\mathfrak{A} \models T_\forall^*$}
 		\end{equation*}
 		\begin{align*}
 			&T^* \text{ ist Modellvervollständigung von } T \\ \Leftrightarrow &T^* \cup \operatorname{Diag}(\mathfrak{A}) \text{ ist vollständig für alle $\mathfrak{A} \models T, \mathfrak{A} \models T^*$} \text{ und $T_\forall^* = T_\forall$}
 		\end{align*}
 		\begin{equation*}
	 		T^* \text{ ist Modellbegleiter von $T$} \Leftrightarrow T^* \cup \operatorname{Diag}(\mathfrak{A}) \text{ ist vollständig für alle $\mathfrak{A} \models T^*$} \text{ und $T_\forall^* = T_\forall$}
 		\end{equation*}
 	\end{remark}
 
 	\begin{remark}
 		Wenn $T$ modellvollständig ist hat $T$ die Amalgamierungseigenschaft.
 	\end{remark}
 
 	\begin{proof}
 		\begin{equation*}
	 		\mathfrak{M_1} \models T \supseteq \mathfrak{M_0} \models T \subseteq \mathfrak{M_2} \models T
 		\end{equation*}
 		
 		Zu zeigen $T \cup \operatorname{Diag}(\mathfrak{M_1}) \cup \operatorname{Diag}(\mathfrak{M_2})$ ist konsistent. Angenommen nicht dann ist $T \cup \{ \varphi(\overline{m}_1), \psi(\overline{m}_2) \}$ ist inkonsistent. d.h. $T \vdash (\varphi(\overline{m}_1) \rightarrow \lnot \psi(\overline{m}_2))$.
 		
 		Interpolationssatz: $T \vdash (\varphi(\overline{m}_1) \rightarrow \chi(\overline{m}_0)) \land (\chi(\overline{m}_0) \rightarrow \lnot \psi(\overline{m}_2))$ mit $\chi(\overline{m}_0)$ $\mathcal{L}_{\mathfrak{M_0}}$-Formel.
 		
 		Das heißt
 		\begin{align*}
	 		\mathfrak{M_1} &\models \chi(\overline{m}_0) \\
	 		\mathfrak{M_2} \models \lnot \chi(\overline{m}_0)
 		\end{align*}
 		$\lightning$ zu $T \cup \operatorname{Diag}(\mathfrak{M_0})$ ist vollständig.
 	\end{proof}
 
 	\section{Abzählbare Modelle}
 	
 	\subsection{Typen}
 	
 	\begin{definition}
 		$\mathfrak{M}$ sei $\mathcal{L}$-Struktur, $A \subseteq M$.
 		
 		\begin{itemize}
 			\item Eine Menge $\Sigma = \Sigma(x_1, \dots, x_n)$ von $\mathcal{L}_A$-Formeln $\varphi(x_1, \dots, x_m)$ heißt endlich erfüllbar in $\mathfrak{M}$, falls für alle $k \in \setN$ und $\varphi_1, \dots, \varphi_k \in \Sigma$
 			\begin{equation*}
	 			\mathfrak{M} \models \exists x_1 \dots \exists x_n(\varphi_1(\overline{x}) \land \dots \land \varphi_n(\overline{x})
	 		\end{equation*}
 			äquivalent: es gibt $m_1, \dots, m_n \in M$ mit $\mathfrak{M} \models \varphi_1(\overline{m}) \land \dots \land \varphi_k(\overline{m})$
 			\item $\mathfrak{M}$ realisert $\Sigma$ (oder $\mathfrak{M}$ erfüllt $\Sigma$), falls es $m_1, \dots, m_n \in M$ gibt mit $\mathfrak{M} \models \varphi(m_1, \dots, m_n)$ für alle $\varphi \in \Sigma$.
 			\item Sonst $\mathfrak{M}$ übergeht $\Sigma$ ($\mathfrak{M}$ lässt $\Sigma$ aus)
 		\end{itemize}
 	\end{definition}
 
 	\begin{lemma}
 		$\Sigma$ endlich erfüllbar in $\mathfrak{A}$ $\Leftrightarrow$ es gibt $\mathfrak{B} \succcurlyeq \mathfrak{A}$, $\mathfrak{B}$ realisiert $\Sigma$
 	\end{lemma}
 
 	\begin{proof}
 		\heading{$\Leftarrow$} klar
 		
 		\heading{$\Rightarrow$} zeige: $\operatorname{Th}(\mathfrak{A}_A) \cup \Sigma(\overline{c})$ ist endlich erfüllbar.
 	\end{proof}
 
 	\begin{definition}
 		\begin{itemize}
 			\item Eine Menge $p = p(x_1, \dots, x_n)$ von $\mathcal{L}_A$-Formeln heißt (vollständiger) $n$-Type über $A$ in $\mathfrak{M}$, falls $p$ maximale in $\mathfrak{M}$ endlich erfüllbare Menge von $\mathcal{L}_A$-Formeln $\varphi(x_1, \dots, x_n)$ ist.
 			
 			Äquivalent: Für jede $\mathcal{L}_A$-Formel $\varphi(x_1, \dots, x_n)$ ist $\varphi \in p$ oder $\lnot \varphi \in p$.
 			\item Wenn $m_1, \dots, m_n \in M$, dann ist
 			\begin{equation*}
	 			\operatorname{tp}^\mathfrak{M}(m_1, \dots, m_n/A) = \{ \varphi(x_1, \dots, x_n) \mid \varphi \text{ $\mathcal{L}_A$-Formel} \; \mathfrak{M} \models \varphi(m_1, \dots, m_n) \}
 			\end{equation*}
 			ist der (vollständige) Typ von $(m_1, \dots, m_n)$ über $A$ in $\mathfrak{M}$.
 			\item Die Menge der $n$-Typen über $A$ in $\mathfrak{M}$ wird mit $S_n^\mathfrak{M}(A)$ bezeichnet.
 			
 			Wenn $\mathfrak{M}$ aus dem Kontext ersichtlich ist, schreibt man $\operatorname{tp}(\overline{m}/A), S_n(A), \dots$
 			
 			Häufig schreibt man $S^\mathfrak{M}(A)$ für $S_1^\mathfrak{M}(A)$
 		\end{itemize}
 	\end{definition}
 
 	\begin{remark}
 		\begin{itemize}
 			\item $\operatorname{tp}^\mathfrak{M}(\overline{m}/A)$ ist ein vollständiger Typ
 			\item Jeder $n$-Typ über $A$ in $\mathfrak{M}$ ist in einer elementaren Erweiterung von $\mathfrak{M}$ realisiert.
 		\end{itemize}
 	\end{remark}
 
 	\begin{remark}
 		Wir betrachten $\mathcal{L}_A$-Formeln modulo Äquivalenz in $\mathfrak{M}$,
 		\begin{equation*}
			\varphi(x_1, \dots, x_n) \sim_\mathfrak{M} \psi(x_1, \dots, x_n) :\Leftrightarrow \mathfrak{M} \models \forall \overline{x}(\varphi(\overline{x}) \leftrightarrow \psi(\overline{x}))
 		\end{equation*}
 		
 		Allgemein hängt es von $\mathfrak{M}$ ab, d.h. wenn $A \subseteq M, A \subseteq N$, dann kann sich $\sim_\mathfrak{M}$ von $\sim_\mathfrak{N}$ unterscheiden!
 		
 		Aber: Falls $\mathfrak{M}, \mathfrak{N} \models T$ und $T$ hat QE, und $\mathfrak{A} \subseteq \mathfrak{M}, \mathfrak{A} \subset \mathfrak{N}$, dann stimmt Äquivalenz für $\mathcal{L}_A$-Formeln in $\mathfrak{M}$ und $\mathfrak{N}$ überein!
 		
 		Boolesche Algebra $\mathcal{F}_n(\mathcal{L}_A)$: Formeln mit feien Variablen unter $v_0, \dots, v_{n-1}$ und Parametern (=neue Konstanten) für Elemente aus $A$.
 		
 		$\sim_\mathfrak{M}$ ist Kongruenzrelation auf $\mathcal{F}_n(\mathcal{L}_A)$, das heißt $\mathcal{F}_n(\mathcal{L}_A)/\sim_\mathfrak{M}$ ist Boolesche Algebra.
 		
 		$\mathcal{F}_n(\mathcal{L}_A) \to \mathcal{F}_n(\mathcal{L}_A)/\sim_\mathfrak{M}, \varphi \mapsto \text{Äquivalenzklasse von $\varphi$ bezüglich $\sim_\mathfrak{M}$}$ ist Homomorphismus boolescher Algebren.
 		
 		\textit{Unterbemerkung:} $h: \mathfrak{B} \to \mathfrak{B}'$ sei surjektiver Homomorphismus Boolescher Algebren.
 		
 		$b \sim_h c :\Leftrightarrow h(b) = h(c)$, dann ist $\sim_h$ Kongruenzrelation und $\mathfrak{B}' \cong \mathfrak{B}/\sim_h$.
 		
 		$h$ bzw. $\sim_h$ ist bestimmt durch
 		\begin{equation*}
	 		\{ b \in B \mid h(b) = 0 \} = \ker(h)
 		\end{equation*}
 		bzw. durch
 		\begin{equation*}
	 		\{ b \in B \mid h(b) = 1 = \ker^*(h) \}
 		\end{equation*}
 		
 		$S_n^\mathfrak{M}(A)$ ist der Stone-Raum von $\mathcal{F}_n(\mathcal{L}_A)/\sim_\mathfrak{M}$ (vollständiger $n$-Typ = Ultrafilter dieser Algebra).
 		
 		Insbesondere ist $S_n^\mathfrak{M}(A)$ ein kompakter (total unzusammenhängender) topologischer Raum.
 	\end{remark}
 
 	\begin{remark}
 		Wenn $p,q \in S_n^\mathfrak{M}(A)$, $p \neq q$ dann gibt es $\mathcal{L}_A$-Formel $\varphi(x_1, \dots, x_n)$ mit $\varphi \in p, \varphi \notin q$ und damit $\lnot \varphi \in q$.
 		
 		Also $p \in \langle \varphi \rangle = \{ r \in S_n^\mathfrak{M}(A) \mid \varphi \in r \}$
 		
 		$q \in \langle \lnot \varphi \rangle$
 		
 		$\langle \varphi \rangle, \langle \lnot \varphi \rangle$ disjunkte clopen Mengen, die $p$ und $q$ trennen (insb. Hausdorffsch).
 	\end{remark}
 
 	\begin{example}
 		\begin{itemize}
 			\item $n = 0$, $S_0^\mathfrak{M}(\emptyset)= \{ \operatorname{Th}(\mathfrak{M}) \}$ einelementig
 			\item $\mathfrak{Q} = (\setQ, <), n = 1$
 			
 			$A = \emptyset:$ $\{ \{ x \doteq x \}^\vdash \}$
 			
 			$A = \setZ:$ 
 			\begin{itemize}
 				\item $\{ x < n \mid n \in \setZ \}$ unendlich klein
 				\item $\{ n < x \mid n \in \setZ \}$ unendlich groß
 				\item $\{ x \doteq n \}, \{ n < x, x < n + 1 \}$ für jedes $n \in \setZ$
 			\end{itemize}
 			2-Punkt-Kompaktifizierung von $\frac12 \setZ$.
 			
 			$A = \setQ$, $S_1^\mathfrak{Q}(\setQ)$
 			\begin{itemize}
 				\item realisierte Typen $p_q = \{ x \doteq q \}^\vdash$ für jedes $q \in \setQ$
 				\item $p_{-\infty} = \{ x < q \mid q \in \setQ \}^\vdash$
 				\item $p_{+\infty} = \{ q < x \mid q \in \setQ \}^\vdash$
 				\item $p_r = \{ q < x, x < q' \mid q < r, r < q' \}$ für $r \in \setR \setminus \setQ$
 				\item $p_q^+ = \{ q < x, x < q' \mid q < q' \}$ für jedes $q \in \setQ$
 				\item $p_q^- = \{ q' < x, x < q \mid q' < q \}$ für jedes $q \in \setQ$
 			\end{itemize}
 		\end{itemize}
 	\end{example}
 
 	\begin{example}
 		\begin{itemize}
 			\item $\setC \models ACF$, sei $n = 1$ und $A$ ein Unterkörper
 	
 			Wie sieht $S_1^\setC(A)$ aus?
 			
 			Realisierte Typen: $p_a = \{ x \doteq a \}^\vdash$ für $a \in A$.
 			
 			Nicht realisierte algebraische Typen: $p_P = \{ P(x) \doteq 0 \}$ mit $P \in A[X]$ und $\deg P > 1$ und $P$ normiert und irreduzibel
 			
 			\textit{Bem:} Falls $\alpha \in \operatorname{Aut}_A(\mathfrak{M}) = \{ \alpha \in \operatorname{Aut}(\mathfrak{M}) \mid a_{\mid A} = id \}$ dann ist
 			\begin{equation*}
	 			\operatorname{tp}^\mathfrak{M}(m_1, \dots, m_n / A) = \operatorname{tp}^\mathfrak{M}(\alpha(m_1), \dots, \alpha(m_n) / A)
 			\end{equation*}
 			
 			Transzendenter Typ: $p_\infty = \{ P(x) \neq 0 \mid P \in A[X], \text{$P$ nicht konstant} \}$ (in $\setC$ realisiert $\Leftrightarrow$ $\tilde{A} \neq \setC$)
 			
 			\item Sei $E$ eine Äquivalenzrelation mit unendlich vielen unendlichen Klassen und ohne endlich Klassen. Setze $n = 2, A = \emptyset$
 			
 			$p_= = \{ x \doteq y \}^\vdash$
 			
 			$p_E = \{ x \neq y, E x y \}^\vdash$
 			
 			$p_{\lnot E} = \{ \lnot E x y \}^\vdash$
 		\end{itemize}
 	\end{example}
 
 	\begin{theorem}
 		Gegeben $\mathfrak{M}$, $A \subseteq \mathfrak{M}$, $n \in \setN$. Dann existiert $\mathfrak{M}' \succcurlyeq \mathfrak{M}$, welches alle Typen aus $S_n^\mathfrak{M}(A)$ realisiert.
 	\end{theorem}
 
 	\begin{proof}
 		Sei $\{ p_i \mid i \in \kappa \}$ Aufzählung von $S_n^\mathfrak{M}(A)$. Nehme neue Konstanten $c_{i_1}, \dots, c_{i_n}$ für alle $i < \kappa$ betrachte $\operatorname{Th}(\mathfrak{M}_M) \cup \{ p_i(c_{i_1}, \dots, c_{i_n}) \mid i < \kappa \}$.
 		
 		Zeige: ist endlich erfüllbar. Es reicht zu zeigen:
 		\begin{equation*}
	 		\varphi_{j_1}(c_{j_{1_1}}, \dots, c_{j_{1_n}}) \land \dots \land \varphi_{j_k}(c_{j_{k_1}}, \dots, c_{j_{k_n}})
 		\end{equation*}
 		ist durch geeignete Interpretation der $c_j$ in $\mathfrak{M}$ erfüllbar.
 		
 		Das ist der Fall, da die Typen in $\mathfrak{M}$ endlich erfüllbar.
 	\end{proof}
 
 	\begin{definition}
 		Ein $n$-Typ $p \in S_n^\mathfrak{M}(A)$ heißt
 		\begin{itemize}
 			\item realisiert, falls eine Formel
 			\begin{equation*}
	 			x_1 \doteq a_1 \land \dots \land x_n \doteq a_n \in p
 			\end{equation*}
 			\item isoliert, falls es eine Formel $\varphi$ gibt so dass $p$ der einzige Typ aus $S_n^\mathfrak{M}(A)$ ist, der $\varphi$ enthält.
 			\begin{equation*}
	 			\left( \langle \varphi \rangle = \{ q \in S_n^\mathfrak{M}(A) \mid \varphi \in q \} \overset{!}{=} \{ p \} \right)
 			\end{equation*}
 			Äquivalent: $p$ ist isolierter Punkt im Stone-Raum, d.h. $\{ p \}$ ist offen.
 		\end{itemize}
 	\end{definition}
 
 	\begin{remark}
 		\begin{itemize}
 			\item Die realisierten Typen sind durch die Formel $x_1 = a_1 \land \dots \land x_n \doteq a_n$ isoliert.
 			\item Ein isolierter Type ist in $\mathfrak{M}$ realisiert, denn $\mathfrak{M} \models \exists \overline{x} \varphi(\overline{x})$ das heißt es gibt $\overline{m} \in M$ mit $\mathfrak{M} \models \varphi(\overline{m})$. Also $\varphi \in \operatorname{tp}^\mathfrak{M}(\overline{m} / A)$ somit $p = \operatorname{tp}^\mathfrak{M}(\overline{m} / A)$
 		\end{itemize}
 	\end{remark}
 
 	\begin{remark}
 		Ein unendlicher kompakter Raum besitzt nicht isolierte Punkte.
 		
 		Angenommen $T = \{ p_i \mid i \in \kappa \}$ und alle $p_i$ sind isoliert.
 		\begin{equation*}
	 		T = \bigcup_{i \in \kappa} \{ p_i \}
 		\end{equation*}
 		also eine offene Überdeckung. Da der Raum kompakt ist existiert eine endliche Teilüberdeckung. $\Rightarrow \kappa$ endlich
 	\end{remark}
	
	\subsection{Der Omitting-Types-Satz}
	
	\begin{example}
		$\mathcal{L} = \{ c_i \mid i \in \omega \}$
		
		$T = \{ \lnot c_i \doteq c_j \mid i \neq j \}$
		
		vollständige Theorie (mit QE)
		
		$\mathfrak{M} \models T$
		\begin{equation*}
			M = \underbrace{\{ c_i^\mathfrak{M} \mid i \in \omega \}}_\text{abzählbar unendl.} \cup \text{evtl. weitere Elemente}
		\end{equation*}
		
		$S_1^\mathfrak{M}(\emptyset)$
		\begin{align*}
			p_i &= \{ x \doteq c_i \}^\vdash \\
			p_\infty &= \{ x \neq c_i \mid i \in \omega \}^\vdash
		\end{align*}
		
		Die $p_i$ sind in allen Modellen realisiert. Der Typ $p_\infty$ dagegen kann übergangen werden nämlich im Primmodell $\mathfrak{M_0} = \{ c_i^\mathfrak{M_0} \mid i \in \omega \}$.		
	\end{example}

	\begin{definition}
		Sei $T$ eine $\mathcal{L}$-Theorie.
		
		Sei $S_n^T$ die Menge der $n$-Typen in $T$, also maximale mit $T$ konsistente Menge von $\mathcal{L}$-Formeln $\varphi(v_0, \dots, v_{n-1})$.
		
		Sei $S_\omega^T$ die Menge der $\omega$-Typen in $T$, also maximale mit $T$ konsistente Mengen von $\mathcal{L}$-Formeln $\varphi(v_0, \dots, v_{n-1})$ mit $n$ variabel.
		
		$p$ $n$-Typ in $T$
		
		$\Leftrightarrow$ $T \cup p$ ist konsistent und für alle $\varphi(v_0, \dots, v_{n-1})$ ist $\varphi \in p$ oder $\lnot \varphi \in p$
		
		$\Leftrightarrow$ Es gibt Modell $\mathfrak{M} \models T$ und $m_0, \dots, m_{n-1} \in M$ mit $\operatorname{tp}^\mathfrak{M}(m_0, \dots, m_{n-1} / \emptyset) = p$
		
		$\Leftrightarrow$ $p$ entspricht Ultrafilter in $\mathcal{F}_n(\mathcal{L}) / \sim_T$
		
		$p$ $\omega$-Typ in $T$
		
		$\Leftrightarrow$ $T \cup p$ ist konsistent und für alle $n$ und für alle $\varphi(v_0, \dots, v_{n-1})$ ist $\varphi \in p$ oder $\lnot \varphi \in p$
		
		$\Leftrightarrow$ Es gibt Modell $\mathfrak{M} \models T$ und $m_i \in M, i \in \omega$ mit $\operatorname{tp}^\mathfrak{M}(m_0, \dots / \emptyset) = p$
		
		$\Leftrightarrow$ $p$ entspricht Ultrafilter in $\mathcal{F}_\infty(\mathcal{L}) / \sim_T$
	\end{definition}

	\begin{remark}
		\begin{equation*}
			S_n^T = \bigcup \{ S_n^{T'} \mid \text{$T'$ ist Vervollständigung von $T$} \}
		\end{equation*}
	\end{remark}

	\begin{definition}
		Sei $T$ eine $\mathcal{L}$-Theorie und $\Sigma(\overline{x})$ eine Menge von $\mathcal{L}$-Formeln $\varphi(\overline{x})$.
		
		\begin{itemize}
			\item $\varphi(\overline{x})$ realisiert lokal $\Sigma(\overline{x})$, falls
			\begin{itemize}
				\item $T \cup \{ \varphi(\overline{x}) \}$ ist konsistent
				\item $T \vdash \forall \overline{x} (\varphi(\overline{x}) \to \sigma(\overline{x}))$ für jedes $\sigma \in \Sigma$.
			\end{itemize}
			\item $T$ lässt $\Sigma(\overline{x})$ aus, falls es ein $\mathfrak{M} \models T$ gibt so dass für alle $\overline{m} \in M$
			\begin{equation*}
				\Sigma \nsubseteq \operatorname{tp}^\mathfrak{M}(\overline{m}/\emptyset)
			\end{equation*}
		\end{itemize}
	\end{definition}

	\begin{remark}
		Falls $\varphi$ $\Sigma$ lokal realisiert und
		\begin{enumerate}
			\item $\mathfrak{M} \models \varphi(\overline{x})$, dann realisiert $\mathfrak{M}$ $\Sigma$.
			\item $T \vdash \exists \overline{x} \varphi(\overline{x})$, dann lässt $T$ $\Sigma$ nicht aus.
		\end{enumerate}
	\end{remark}

	\begin{theorem}[Omitting Types]
		Sei $T$ eine $\mathcal{L}$-Theorie, und $\abs{\mathcal{L}} \leq \aleph_0$.
		
		Falls $\Sigma(\overline{x})$ nicht lokal realisiert ist, gibt es $\mathfrak{M} \models T$ welches $\Sigma$ auslässt.
	\end{theorem}

	\heading{Einschub: Etwas Topologie}
	
	\begin{definition}
		Sei $X$ ein Topologischer Raum.
		
		$N  \subseteq X$ heißt nirgends dicht, falls es keine offene Menge $\mathcal{O} \neq \emptyset$ mit $\mathcal{O} \subseteq \overline{N}$ gibt. (jedes offene $\mathcal{O} \neq \emptyset$ schneidet $X \setminus N$)
		
		$M \subseteq X$ heißt mager, falls $M \subseteq \bigcup_{i \in \omega} N_i$ mit $N_i$ nirgends dicht.
	\end{definition}

	\begin{remark}
		Die mageren Mengen bilden ein $\sigma$-Ideal, das heißt
		\begin{itemize}
			\item $A \subseteq M$ mager $\Rightarrow$ $A$ mager
			\item $M_i$ mager $\Rightarrow$ $\bigcup_{i \in \omega} M_i$ mager
		\end{itemize}
	\end{remark}

	\begin{theorem}[Satz von Baire]
		Jeder lokal-kompakte Hausdorff-Raum ist nicht mager.
		
		Hausdorff: Für alle $x \neq y$ existieren disjunkte offene Mengen $x \in \mathcal{O}_x, y \in \mathcal{O}_y$
		
		Lokal-Kompakt: Für alle $x$ gibt es offene Menge $x \in \mathcal{O}_x$ so dass $\overline{\mathcal{O}_x}$ kompakt ist.
	\end{theorem}

	\begin{proof}
		$S(\mathcal{F}_{n/\infty}(\mathcal{L})/\sim_T)$: Abgeschlossene Mengen in solch einem Stone-Raum?
		
		\begin{align*}
			N \text{ abgeschlossen} \quad\Leftrightarrow\quad N &= \bigcap_{i \in I} \langle \varphi_i \rangle \\
			&= \{ p \mid \varphi_i \in p, i \in I \} \\
			&= \{ p \mid \Sigma \subseteq p \}
		\end{align*}
		mit $\Sigma = \{ p \mid \Sigma \subseteq p \}$.
		
		Abgeschlossene Mengen entsprechen Theorien/Formelmengen in $\mathcal{F}_{n/\infty}(\mathcal{L}) / \sim_T$.
		
		Wann ist $N$ nirgends dicht?
		
		$N$ nicht nirgends dicht
		
		$\Leftrightarrow$ Es gibt $\varphi$ mit $\langle \varphi \rangle \subseteq N$ $\varphi \nsim_T \bot$ d.h. $\varphi$ konsistent mit $T$
		
		$\Leftrightarrow$ (falls $N$ durch $\Sigma$ gegeben) jedes $p$ mit $\varphi \in p$ erfüllt $\Sigma \subseteq p$, das heißt $\varphi$ realisiert lokal $\Sigma$.
		
		Kontraposition: nirgends dicht entspricht nicht lokal realisiert
		
		\textit{Zeige also:} Nirgends dichte Formelmengen können übergangen werden.
		
		\textit{1. Schritt:} Wähle $\mathfrak{N} \models T$, das für alle $n$ alle $n$-Typen in $T$ realisiert. (Existenz ähnlich wie bei dem Satz über die elemenatare Erweiterung, die alle $p \in S_n^\mathfrak{M}(A)$ realisiert).
		
		Sei $A \subseteq N$ abzählbar $A = \{ a_i \mid i \in \omega \}$. Was muss $\operatorname{tp}((a_i)_{i \in \omega})$ erfüllen, damit $A$ Träger eine elementaren Unterstruktur ist?
		
		Gilt Tarskis Test? Betrachte Formel $\varphi(\overline{x}, y)$ (Erinnerung: Falls $\mathfrak{N} \models \exists y \varphi(\overline{a}, y)$ dann gibt es solch ein $y$ in $A$)
		
		Setze
		\begin{equation*}
			\mathcal{O}_\varphi \coloneqq \bigcup_{i \in \omega} \langle (\exists y \varphi(\overline{x}, y) \to \varphi(\overline{x}, x_i)) \rangle
		\end{equation*}
		
		$\omega$-Typ $p(x_0, x_1, \dots) \in \mathcal{O}_\varphi$ $\Leftrightarrow$ Realisierung von $p$ erfüllt Tarskis Test für $\varphi$
		
		$p(x_0, x_1, \dots) \in \bigcap_\varphi \mathcal{O}_\varphi$ $\Leftrightarrow$ Realisierung ist Träger einer elementaren Unterstruktur
		
		\textit{2. Schritt:} Behauptung: $S_\omega^T \setminus \mathcal{O}_\varphi$ ist nirgends dicht. Also falls $\mathcal{O} \neq \emptyset$ offen, ist $\mathcal{O} \cap \mathcal{O}_\varphi \neq \emptyset$. Ohne Einschränkung $\mathcal{O} = \langle \psi(\overline{x}) \rangle$.
		
		Wähle $i$ so, dass $x_i$ weder in $\varphi(\overline{x}, y)$ noch in $\psi(\overline{x})$ vorkommt. Dann ist $T \cup \{ \exists y \varphi(\overline{x},y) \rightarrow \varphi(x, x_i), \psi(\overline{x}) \}$ konsistent.
		
		Folglich ist $S_\omega^T \setminus \bigcap_\varphi \mathcal{O}_\varphi = \bigcup_\varphi S_\omega^T \setminus \mathcal{O}_\varphi$ mager. Da $\abs{\mathcal{L}} \leq \aleph_0$, gibt es hächstens abzählbar viele Formeln der Form $\exists y \varphi(\overline{x}, y)$.
		
		Mit Baire $S_\omega^T \setminus \bigcap_\varphi \mathcal{O}_\varphi \neq \emptyset$.
		
		\textit{3. Schritt:} Entferne zusätzlich alles, was $\Sigma$ realisiert. Problem ist $\Sigma$ lebt in $S_n^T$ nicht in $S_\omega^T$.
		\begin{enumerate}
			\item Ein Tupel in $A$, welches $\Sigma$ realisiert, könnte beliebige Koordinaten haben.
			\item Verschiedene Komponenten könnten gleich sein.
		\end{enumerate}
	
		Betrachte also $\alpha: n \to \omega$.
		
		Idee: $\Sigma(v_0, \dots, v_{n-1}) \rightsquigarrow \Sigma(v_{\alpha(0)}, \dots, v_{\alpha(n-1)})$. $\alpha$ induziert Abbildung
		\begin{align*}
			\mathcal{F}_n(\mathcal{L})/ \sim_T &\longrightarrow \mathcal{F}_\infty(\mathcal{L}) / \sim_T \\
			\varphi(v_0, \dots, v_{n-1}) &\longmapsto \varphi(v_{\alpha(0)}, \dots, v_{\alpha(n-1)})
		\end{align*}
		welche wiederum eine Abbildung:
		\begin{align*}
			\tilde{\alpha}:&& S_\omega^T &\longrightarrow S_n^T \\
			&&p(x_0, x_1, \dots) &\longmapsto \text{Einschränkung auf die Variablen $x_{\alpha(0)}, \dots, x_{\alpha(n-1)}$ }
		\end{align*}
		induziert.
		
		Allgemein: $B \to B'$ Homomorphismus Boolescher Algebren induziert stetige Abbildung $S(B') \to S(B)$.
		
		Stetigkeit von $\tilde{\alpha}$: $\tilde{\alpha}^{-1}(\langle \varphi(v_0, \dots, v_{n-1})\rangle) = \langle \varphi(v_{\alpha(0)}, \dots, v_{\alpha(n-1)}) \rangle$.
		
		Hier: $\tilde{\alpha}$ ist sogar offen!
		
		\textit{Lemma aus der Topologie:} $\alpha: X \to Y$ offene Abbildung dann gilt: $N \subseteq Y$ nirgends dicht $\Rightarrow$ $\alpha^{-1}(M)$ nirgends dicht.
		
		\begin{equation*}
			\left( S_\omega^T \setminus \bigcap_\varphi \mathcal{O}_\varphi \right) \cup \bigcup_{\alpha: n \to \omega} \alpha^{-1}[\Sigma(\overline{x})]
		\end{equation*}
		ist mager, also $\neq S_\omega^T$.
		
		Sei $p \notin \left( S_\omega^T \setminus \bigcap_\varphi \mathcal{O}_\varphi \right) \cup \bigcup_{\alpha: n \to \omega} \alpha^{-1}[\Sigma(\overline{x})]$ und $(a_i)_{i \in \omega} \models p$.
		
		Dann ist $A = \{ a_i \mid i \in \omega \}$ Träger einer elementaren Unterstruktur von $N$ die $\Sigma$ auslässt.
		
		Zeige noch $\tilde{\alpha}$ ist offene Abbildung, das heißt für offenes $\mathcal{O}$ ist $\tilde{\alpha}(\mathcal{O})$ offen.
		
		$\mathcal{O} = \bigcup_{i \in I} \langle \varphi_i \rangle$ Basis offen, dann ist
		\begin{equation*}
			\tilde{\alpha}(\bigcup \dots) = \bigcup \tilde{\alpha}(\dots)
		\end{equation*}
		
		Ohne Einschränkung $\mathcal{O} = \langle \varphi \rangle$. Und ohne Einschränkung
		\begin{equation*}
			\varphi = \varphi(\underbrace{x_1, \dots, x_k}_{\in \operatorname{Bild}(\alpha)}, \underbrace{y_1, \dots, y_l}_{\notin \operatorname{Bild}(\alpha)}) \qquad\qquad x_i,y_j \in \{ v_0, \dots \}
		\end{equation*}
		
		\begin{align*}
			\tilde{\alpha}(\langle \varphi(\overline{x}, \overline{y} \rangle ) &= \tilde{\alpha}(\{ p \in S_\omega^T \mid \varphi \in p \}) \\
			&= \tilde{\alpha}(\{ \operatorname{tp}((c_i)_{i \in \omega / \emptyset}) \mid c_i \in N, \mathfrak{N} \models \varphi(\overline{c}) \}) \\
			\intertext{$\mathfrak{N}$ soll auch alle $\omega$-Typen $/\emptyset$ realisieren} \\
			&= \{ \operatorname{tp}(c_{\alpha(0)}, \dots, c_{\alpha(n)}) \mid c_i \in N, \mathfrak{N} \models \varphi(\overline{c}) \} \\
			\intertext{in Variablen $v_0, \dots, v_{n-1}$} \\
			&= \left\{ q \in S_n^T \mid \exists \overline{y} \varphi\left( \frac{v_0, \dots, v_{n-1}}{\overline{x}} , \overline{y} \right) \in q, v_i = v_j \in q \text{für alle $0 \leq i < j < n$ mit $\alpha(i) = \alpha(j)$} \right\} \\
			&= \langle \exists \overline{y} \varphi(\frac{\overline{v}}{\overline{x}}, \overline{y}) \cap \bigcap_{\alpha(i) = \alpha(j)} \langle v_i \doteq v_j \rangle \\
			&= \langle \exists \overline{y} \varphi(\frac{\overline{v}}{\overline{x}}, \overline{y}) \land \bigwedge_{\alpha(i) = \alpha(j)} v_i \doteq v_j
		\end{align*}
		ist offen.
	
	\end{proof}

	\begin{corollary}
		Sei $X_n \subseteq S_n^T$ mager. Dann gibt es ein Modell $\mathfrak{M} \models T$, das alle $X_n$ übergeht.
	\end{corollary}

	\begin{corollary}
		\begin{itemize}
			\item Insbesondere: Wenn $T$ eine vollständige $\mathcal{L}$-Theorie ist und $\abs{\mathcal{L}} \leq \aleph_0$ und $p \in S_n^T$: $p$ kann übergangen werden $\Leftrightarrow$ $p$ ist nicht isoliert
			\item Sei $\mathcal{L} \leq \aleph_0$ und $X_n$ magere Mengen von $n$-Typen. Dann können alle $X_n$ simultan übergangen werden.
			
			Insbesondere alle abzähbare Mengen von nicht isolierten Typen sind mager!
		\end{itemize}
	\end{corollary}

	\begin{remark}
		Sei $X$ ein topologischer Raum, $x_0 \in X$ dann gilt
		\begin{equation*}
			\{ x_0 \} \text{ mager} \Leftrightarrow \{ x_0 \} \text{ nirgends dicht} \Leftrightarrow \text{$x_0$ nicht isoliert}
		\end{equation*}
	\end{remark}

	\begin{example}
		$\mathcal{L} = \{ c \} \cup \{ c_i \mid i \in \aleph_1 \}$
		
		$T = \{ c_i \neq c_j \mid i < j < \aleph_1 \}$
		
		$\Sigma(x) = \{ x \neq c \} \cup \{ x \neq c_i \mid i \in \aleph_0 \}$
		
		$\Sigma$ ist stets realisiert, da jedes $\mathfrak{M} \models T$ überabzählbar ist.
		
		$\Sigma$ ist nicht lokal realisiert ($T$ hat QE, also wäre eine lokal realisierende Formel von der Form $x = x_\alpha$ für ein $\alpha < \aleph_1$ oder $x = c$ oder die Verneinungen davon)
	\end{example}

	\begin{definition}
		Seien $\mathfrak{M}, \mathfrak{N}$ $\mathcal{L}$-Strukturen $A \subseteq M, B \subseteq N$.wenn
		
		$h: A \to B$ heißt elementare Abbildung falls $h$ die Gültigkeit von $\mathcal{L}_A$-Formeln erhält, d.h. $\mathfrak{M} \models \varphi(\overline{h}) \Leftrightarrow \mathfrak{N} \models \varphi(h(\overline{h}))$.
		
		$h: A \to B$ heißt partieller Isomorphismus, falls $h$ elementar und bijektiv ist.
	\end{definition}

	\begin{remark}
		\begin{itemize}
			\item Elementare Abbildungen sind stets injektiv!
			\item $h$ elementar $\Leftrightarrow$ $\forall \overline{a}: \operatorname{tp}^\mathfrak{M}(\overline{a} / \emptyset) = \operatorname{tp}^\mathfrak{N}(h(\overline{a}) / \emptyset)$
			\item Spezialfall: $A = B = \emptyset$ (es existiert genau eine Abbildung $\emptyset \to \emptyset$ die leere Abbildung)
			
			$h: \emptyset \to \emptyset$ ist elementar $\Leftrightarrow$ $\mathfrak{M} \equiv \mathfrak{N}$
			\item Falls $h: A \to B$ partieller Isomorphismus.
			
			Induziert $\mathcal{F}_\infty(\mathcal{L}_A) / \sim_\mathfrak{M} \to \mathcal{F}_\infty(\mathcal{L}_B) / \sim_\mathfrak{N}$
			
			Induziert $\tilde{h}: S_n^\mathfrak{N}(B) \to S_n^\mathfrak{M}(A)$ einen Homöomorphismus
			
			\begin{equation*}
				\{ \varphi(\overline{x}, \overline{a}) \mid \varphi \in p \} = p \in S_n(A) \overset{\tilde{h}^{-1}}{\longrightarrow} \{ \varphi(\overline{x}, h(\overline{a})) \mid \varphi \in p \} = q \in S_n(B)
			\end{equation*}
		\end{itemize}
	\end{remark}

	\begin{remark}(ohne Beweis)
		
		Sei $h: A \to B$ partieller Isomorphismus. Dann gibt es $\mathfrak{M}' \succcurlyeq \mathfrak{M}, \mathfrak{N}' \succcurlyeq \mathfrak{N}$ und $h': \mathfrak{M}' \overset{\cong}{\to} \mathfrak{N}'$ mit $h'_{\mid A} = h$.
	\end{remark}

	\begin{remark}
		$\operatorname{tp}(a_1, a_2 / B) = \{ \varphi(x_1, x_2, \overline{b}) \mid \mathfrak{M} \models \varphi(a_1, a_2, \overline{b}) \}$ ist bestimmt durch $\operatorname{tp}^\mathfrak{M}(a_2 /B) = \{ \varphi(x, \overline{b}) \mid \mathfrak{M} \models \varphi(a_2, \overline{b}) \}$ und $\operatorname{tp}^\mathfrak{M}(a_1 / B, a_2) = \{ \mathfrak{M} \models \varphi(a_1, a_2, \overline{b}) \}$
	\end{remark}

	Sei nun $T$ eine vollständige abzählbare (d.h. $\abs{\mathcal{L}} \leq \aleph_0$) Theorie.
	
	\begin{definition}
		$\mathfrak{M} \models T$ heißt atomar, falls für alle $n$ und alle $m_1, \dots, m_n \in M$ $\operatorname{tp}^\mathfrak{M}(m_1, \dots, m_n / \emptyset )$ isoliert ist.
	\end{definition}

	\begin{remark}
		Wenn $p \in S_n^\mathfrak{M}(\emptyset)$ isoliert ist und zwar durch $\varphi(\overline{x})$ das heißt $\emptyset \neq \langle \varphi(\overline{x}) \rangle = \{ p \}$, dann gilt für jede Formel $\psi(\overline{x})$ mit $\mathfrak{M} \models \exists
		 \overline{x} \psi(\overline{x})$
		 \begin{align*}
			 \text{entweder:}& \operatorname{Th}(\mathfrak{M}) \vdash \forall \overline{x} ( \varphi(\overline{x}) \to \psi(\overline{x}) \\
			 \text{oder:} & \operatorname{Th}(\mathfrak{M}) \vdash \forall \overline{x} ( \varphi(\overline{x}) \to \lnot \psi(\overline{x})
		 \end{align*}
		 Das heißt $\Phi \coloneqq \{ \overline{m} \in M \mid \mathfrak{M} \models \varphi(\overline{m}) \}$ kann nicht definierbar aufgeteilt werden, also definierbare Teilmengen von $M^n$ enthält entweder $\Phi$ oder ist disjunkt zu $\Phi$.
		 
		 In $\mathcal{F}_n(\mathcal{L}) / \sim_\mathfrak{M}$ ist $\varphi$ ein Atom, das heißt ein minimales Elemente oberhalb von $\bot$.
		 
		 Wenn $\mathfrak{M}$ atomar ist, sind alle $\mathcal{F}_n(\mathcal{L})/ \sim_\mathfrak{M}$ ist atomare Boolesche Algebra.
	\end{remark}

	\begin{theorem}
		$\mathfrak{M}$ ist genau dann elementares Primmodell von $T$, wenn $\mathfrak{M}$ atomar und höchstens abzählbar ist.
	\end{theorem}

	\begin{proof}
		\heading{$\Rightarrow$:} Angenommen $p = \operatorname{tp}(m_1, \dots, m_n / \emptyset)$ ist nicht isoliert.
		
		Ommiting Types: Es existiert $\mathfrak{N} \models T$, welches $p$ übergeht. $\mathfrak{M} \hookrightarrow \mathfrak{N}$ da $\mathfrak{M}$ elementares Primmodell. Dann realisiert $\mathfrak{M}$ höchstens so viele Typen wie $\mathfrak{N}$. $\lightning$
		
		Da $\abs{\mathcal{L}} \leq \aleph_0$ gibt es überhaupt abzählbare Modelle, dann muss ein Primmodell auch abzählbar sein.
		
		\heading{$\Leftarrow$:} Sei $M = \{ m_i \mid i \in \omega \}$, $\mathfrak{M}$ atomar. Sei $\mathfrak{N} \models T$, konstruiere indukti Einbettung $h: \mathfrak{M} \hookrightarrow \mathfrak{N}$ das heißt wir konstruieren elementare Abbildungen $h_n: \{ m_0, \dots, m_{n-1} \} \to N$ mit $h_n \subseteq h_{n+1}$
		
		$h_0: \emptyset \to \emptyset$ ist elementar, da $\mathfrak{M} \equiv \mathfrak{N}$.
		
		$n \to n+1:$
		
		$h_n: \{ m_0, \dots, m_{n-1} \} \to N$ sei konstruiert
		
		Behauptung: $\operatorname{tp}(m_n / m_0, \dots, m_{n-1})$ ist isoliert
		
		$\operatorname{tp}(m_n / m_0, \dots, m_{n-1}, m_n)$ ist isoliert durch $\varphi(x_0, \dots, x_n)$, also ist $\operatorname{tp}(m_n / m_0, \dots, m_{n-1})$ isoliert (in $S_1^\mathfrak{M}(m_0, \dots, m_{n-1})$) durch $\varphi(m_0, \dots, m_{n-1}, x)$
		
		$h_n: \{ m_0, \dots, m_{n-1} \} \to \{ h(m_0), \dots, h(m_{n-1}) \}$ ist partieller Isomorphismus, also ist
		\begin{equation*}
			\tilde{h}_n^{-1}(\operatorname{tp}(m_n / m_0, \dots, m_{n-1})) \in S_1^\mathfrak{N}(h_n(m_0), \dots, h_n(m_{n-1}))
		\end{equation*}
		isoliert.
		
		Da $\mathfrak{M} \models \exists x \varphi(m_0, \dots, m_{n-1}, x)$ gilt $\mathfrak{N} \models \exists x \varphi(h(m_0), \dots, h(m_{n-1}), x)$
		
		Wähle $h_{n+1}(m_n)$ so dass es in $\mathfrak{N}$ $\varphi(h(m_0), \dots, h(m_{n-1}, x)$ erfüllt.
		
		$h = \bigcup_{n \in \omega} h_n: M \to N$ ist elementar d.h. $h: \mathfrak{M} \to \mathfrak{N}$ ist elementare Einbettung.	
	\end{proof}

	\begin{corollary}
		Wenn $\mathfrak{M}$ und $\mathfrak{N}$ elementare Primstrukturen sind, dann $\mathfrak{M} \cong \mathfrak{N}$. ($\abs{\mathcal{L}} \leq \aleph_0$ ist wichtig hier!)
	\end{corollary}

	\begin{example}
		\begin{itemize}
			\item $ACF_0$ hat $\tilde{\setQ}$ als elementares Primmodell. Das realisiert nur algebraische Typen, transzendenter Typ wird übergangen.
			\item $(\setQ, <)$ ist $\aleph_0$-Kategorisch also ist $(\setQ, <)$ elementares Primodell
			\item $\omega$ unabhängige Prädikate. Es gibt keien isolierten Typen! $(S_1^T)$ ist perfekte Menge
		\end{itemize}
	\end{example}

	\begin{theorem}
		$T$ (abzählbar und vollständig) besitzt genau dann ein elementares Primmodell, wenn für alle $n$ die isolierten Typen dicht in $S_n^T$ liegen.
	\end{theorem}

	\begin{proof}
		\heading{$\Rightarrow$} $\mathfrak{M}$ sei elementares Primmodell $\emptyset \neq \langle \varphi(\overline{x}) \rangle$ nicht leere offene Menge in $S_n^T$.
		
		Da nicht leer ist $\varphi(\overline{x})$ konsistent mit $T$, das heißt $\mathfrak{M} \models \exists \overline{x} \varphi(\overline{x})$.
		
		Falls $\overline{m} \in M$ so, dass $\mathfrak{M} \models \varphi(\overline{m})$ dann ist $\operatorname{tp}^m(\overline{m} / \emptyset) \in \langle \varphi(\overline{x}) \rangle$ isoliert, da $\mathfrak{M}$ atomar.
		
		\heading{$\Leftarrow$:} $I_n = \{ p \in S_n^T \mid p \text{ isoliert} \}$
		
		$I_n$ ist offen. $p \in I_n$ wir isoliert durch $\varphi_p$ also $I_n = \bigcup_{p \in I_n} \langle \varphi_p \rangle$
		
		$S_n^T \setminus I_n$ ist abgeschlossen. Für nirgends dicht zeige falls $\mathcal{O}$ offen und $\mathcal{O} \subseteq S_n^T \setminus I_n$, dann $\mathcal{O} = \emptyset$.
		
		$I_n$ dicht $\Rightarrow$ $I_n \cap \mathcal{O}\neq \emptyset$ für jedes offene $\mathcal{O} \neq \emptyset$.
		
		Omitting Types: Es gibt Modell, das alle $S_n^T \setminus I_n$ simultan übergeht, also nur isolierte Typen realisiert (also atomar). Ohne Einschränkung ist das Modell abzählbar also elementares Primmodell.
	\end{proof}

	\subsection{\texorpdfstring{$\aleph_0$}{Aleph 0}-kategorische Theorien}
	
	Löwenheim-Skolem-Tarski: Für jede unendliche Kardinalzahl $\kappa$ existiert ein Modell der Mächtigkeit $\kappa$.
	
	\begin{definition}
		\begin{enumerate}
			\item $\mathfrak{M} \models T$ heißt $\kappa$-universell, falls jedes $\mathfrak{N} \models T$ mit $\abs{N} \leq \kappa$ sich elementar in $\mathfrak{M}$ einbetten lässt.
			\item $\mathfrak{M} \models T$ heißt $\kappa$-saturiert, falls für alle $A \subseteq M$ mit $\abs{A} < \kappa$ alle $p \in S_1^\mathfrak{M}(A)$ in $\mathfrak{M}$ realisiert sind. $\mathfrak{M}$ heißt saturiert, falls $\mathfrak{M}$ $\abs{M}$-saturiert ist.
		\end{enumerate}
	\end{definition}

	\begin{remark}
		\begin{enumerate}
			\item $\mathfrak{M}$ kann nie $\abs{M}^+$-saturiert sein. $M = \{ m_i \mid i < \kappa \}$.
		
			Existiert $p \in S_1(M)$ mit $\{ x \neq \mid i \in \kappa \} \subseteq p$ kann in $\mathfrak{M}$ nicht realisiert sein!
			
			\item Falls $\mathfrak{M}$ $\kappa$-saturiert ist, dann realisiert $\mathfrak{M}$ auch alle $p \in S_n^\mathfrak{M}(A)$ mit $\abs{A} < \kappa$.
			
			$p = p(x_1, \dots, x_n) \in S_n(A)$
			
			$p \upharpoonright x_1 = \{ \varphi(x_1) \mid \varphi \in p(x_1, \dots, x_n) \} \in S_1(A)$.
			
			Realisiere $p \upharpoonright x_1$ in $\mathfrak{M}$ durch $m_1$.
			
			$p(m_1, x_2, \dots, x_n) \in S_{n-1}(A \cup \{ m_1 \})$
			
			$p(m_1, \overline{x}) \upharpoonright x_2 \in S_1(A \cup \{ m_1 \})$, realisiert in $\mathfrak{M}$ durch $m_2$.
			
			Etc. $(m_1, \dots, m_n)$ realisiert $p$.
		\end{enumerate}
	\end{remark}

	\begin{theorem}
		Sei $T$ vollständig und abzählbar sowie $\mathfrak{M} \models T$ abzählbar und saturiert
		\begin{enumerate}
			\item $\mathfrak{M}$ ist $\aleph_0$-universell
			\item Falls $\mathfrak{N} \models T$ abzählbar und saturiert ist, dann $\mathfrak{N} \cong \mathfrak{M}$.
		\end{enumerate}
	\end{theorem}

	\begin{proof}
		\heading{1)} Sei $\mathfrak{N} \models T$ abzählbar, $N = \{ n_i \mid i \in \omega \}$
		
		Konstruiere induktiv elementare Abbildungen $h_i: \{ n_0, \dots, n_{i-1} \} \to M$
		
		$h_0 = \emptyset: \emptyset \to M$ elementar, da $T$ vollständig.
		
		$i \to i+1$ Idee: realisiere $\operatorname{tp}(n_i/ n_0, \dots, n_{i-1})$ im Bild das heißt $\tilde{h}^{-1}_i(\operatorname{tp}^\mathfrak{N}(n_i / n_0, \dots, n_{i-1})) \in S_1^\mathfrak{M}(h_i(n_0), \dots, h_i(n_{i-1}))$.
		
		Da $\mathfrak{M}$ $\aleph_0$-saturiert, gibt es $m_i \coloneqq h_{i+1}(n_i)$ mit
		\begin{equation*}
			\operatorname{tp}(m_i / m_0, \dots, m_i-1) = \tilde{h}_i^{-1}(\operatorname{tp}^\mathfrak{N}(n_i / n_0, \dots, n_i-1))
		\end{equation*}
		und $h = \bigcup_{i \in \omega} h_i: N \to M$ ist elementar
		
		\heading{2)} Back and Forth
	\end{proof}
	
	\begin{example}""
		
		\begin{enumerate}
			\item $\mathcal{L} = \emptyset$, $T$ unendliche Menge
			
			$T$ ist $\aleph_0$-kategorisch, jede abzählbare Menge ist elementares Primmodell, $\aleph_0$-saturiert und damit $\aleph_0$-universell. Aber es gibt viele Einbettungen (= Injektionen) zwischen zwei abzählbaren Modellen, die nicht bijektiv sind.
			
			\item $T = ACF_0$ elementares Primmodell: $\tilde{\setQ}$ jede Einbettung $\tilde{\setQ} \hookrightarrow \tilde{\setQ}$ ist bijektiv.
			
			abzählbar, $\aleph_0$-saturiert: $\overline{\setQ(X_0, X_1, \dots )}$ hat viele nicht surjektive Selbsteinbettungen.
			
			\item $\mathcal{L} = \{ P_i \mid i \in \omega \}$
			
			$T$ die $P_i$ sind unabhängige Prädikate
			
			$\abs{S_1^T} = 2^{\aleph_0}$
			
			Kein abzählbares Modell kann alle Typen aus $S_1^T$ realisieren, also auch $\aleph_0$-saturiert sein.
		\end{enumerate}
	\end{example}

	\begin{theorem}
		Sei $T$ vollständig und abzählbar:
		
		$T$ besitzt ein abzählbares saturiertes Modell $\Leftrightarrow$ für alle $n$ ist $\abs{S_n^T} \leq \aleph_0$.
	\end{theorem}

	\begin{theorem}[Engeler, Ryll-Nardzewski, Svenonius]
		Sei $T$ vollständig und abzählbar. Dann sind äquivalent:
		\begin{enumerate}
			\item $T$ ist $\aleph_0$-Kategorisch
			\item alle abzählbaren Modelle von $T$ sind atomar
			\item alle abzählbaren Modelle von $T$ sind elementare Primmodelle
			\item alle Modelle von $T$ sind $\aleph_0$-saturiert
			\item für alle $n$ ist $\mathcal{F}_n(\mathcal{L}/ \sim_T)$ endlich d.h. es gibt nur endlich viele Formeln $\varphi(x_1, \dots, x_n)$ bis auf $\sim_T$.
			\item für alle $n$ ist $S_n^T$ endlich
			\item für alle $n$ sind alle Typen in $S_n^T$ isoliert
			\item für alle Modelle $\mathfrak{M} \models T$ und endliche $A \subseteq M$ sind alle Typen in $S_1^\mathfrak{M}(A)$ isoliert.
		\end{enumerate}
	\end{theorem}

	\begin{proof}
		\textit{Fall 1:} $T$ hat eindeutig bestimmtes endliches Modell. Dann gibt es nur endlich viele Typen, da jeder Typ irgendwo realisiert sein muss. Vieles ist trivialerweise klar, anderes folgt wie im 2. Fall.
		
		\textit{Fall 2:} Es gibt abzählbare Modelle (Löwenheim!)
		
		\heading{$1) \Rightarrow 2),3)$} Löwenheim: Jedes Modell von $T$ hat abzählbare elementare Unterstruktur, die das bis auf Isomorphie, eindeutige abzählbare Modell ist. Dieses ist elementare Primstruktur, somit atomar.
		
		\heading{$2) \Leftrightarrow 3)$} Satz von letzter Woche: Charakterisierung elementarer Primmodelle
		
		\heading{$3) \Rightarrow 1)$} Elementare Primmodelle sind eindeutig bis auf $\cong$.
		
		\heading{$4) \Rightarrow 1)$} Satz von eben: abzählbare $\aleph_0$-saturierte Modelle sind eindeutig bis auf $\cong$.
		
		\heading{$5) \Rightarrow 6)$} $S_n^T$ ist Menge der Ultrafilter von $\mathcal{F}_n(\mathcal{L})/\sim_T$, und da $\mathcal{F}_n(\mathcal{L})/\sim_T$ endlich ist auch $S_n^T$ endlich.
		
		\heading{$6) \Rightarrow 5)$} Falls $\varphi(x_1, \dots, x_n) \nsim_T \psi(x_1, \dots, x_n)$, dann gibt es $p \in S_n^T$ mit $\varphi \in p, \psi \notin p$ oder umgekehrt: Eine von $\varphi \land \lnot \psi, \lnot \varphi \land \psi$ ist konsistent. Vervollständige $\{ \varphi, \lnot \psi \}$ bzw. $\{ \lnot \varphi, \psi \}$ zu einem Typ.
		
		\heading{$6) \Rightarrow 7)$} $S_n^T = \{ p_1, \dots, p_k \}$. Dann existieren Formeln $\varphi_{ij}$ für $i \neq j$ mit $\varphi_{ij} \in p_i \setminus p_j$.
		
		$p_i$ wird isoliert durch $\bigwedge_{j \neq i} \varphi_{ij}$.
		
		\heading{$7) \Rightarrow 6)$} Alle Typen $p \in S_n^T$ sind isoliert durch $\varphi_p$-
		\begin{equation*}
			S_n^T = \bigcup_{p \in S} \langle \varphi_p \rangle
		\end{equation*}
		ist eine offene Überdeckung mit Singletons. Wegen Kompaktheit existiert eine endliche Teilüberdeckung. D.h. $S_n^T$ ist endlich.
		
		\heading{$8) \Rightarrow 4)$} Isolierte Typen sind stets realisiert.
		
		\heading{$\lnot 7) \Rightarrow \lnot 1)$} Angenommen $p \in S_n^T$ ist nicht isoliert. Ommiting Types: es gibt ein abzählbares Modell, das $p$ übergeht.
		
		Es gibt aber auch $\mathfrak{M} \models T$ das $p$ realisiert durch $\overline{m}$. Löwenheim: Es gibt abzählbare elementare Unterstruktur, die $\overline{m}$ enthält, also auch $p$ realisiert.
		
		\heading{$7) \Rightarrow 8)$} Sei $p \in S_1^\mathfrak{M}(a_1, \dots, a_k)$. $p$ wird in $\mathfrak{M}^* \succcurlyeq \mathfrak{M}$ realisiert durch $b \in M^*$. $\operatorname{tp}(b/a_1, \dots, a_k) = p$
		\begin{equation*}
			\operatorname{tp}(b, a_1, \dots, a_k) \in S_{n+1}^\mathfrak{M}(\emptyset) = S_{n+1}^T
		\end{equation*}
		ist also isoliert durch $\varphi(x_0, x_1, \dots, x_k)$.
		
		\textit{Behauptung:} $\varphi(x_0, a_1, \dots, a_k)$ isoliert $p \in S_1^\mathfrak{M}(a_1, \dots, a_k)$.
		
		Sei $\mathfrak{M}^{**} \succcurlyeq \mathfrak{M}, b' \in M^{**}$ mit $\mathfrak{M}^{**} \models \varphi(b', a_1, \dots, a_k)$. Ohne Einschränkung $\mathfrak{M}^{**} \succcurlyeq \mathfrak{M}^*$ (denn $T$ hat AP)
		
		Da $\varphi$ $\operatorname{tp}(b, a_1, \dots, a_k)$ isoliert, gilt
		\begin{equation*}
			\operatorname{tp}^{\mathfrak{M}^{**}}(b, a_1, \dots, a_k) = \operatorname{tp}^{\mathfrak{M}^{**}}(b', a_1, \dots, a_k)
		\end{equation*}
		Daraus folgt
		\begin{equation*}
			\operatorname{tp}^{\mathfrak{M}^{**}}(b / a_1, \dots, a_k) = \operatorname{tp}^{\mathfrak{M}^{**}}(b'/ a_1, \dots, a_k) = p
		\end{equation*}
	\end{proof}

	\begin{example}
		$\aleph_0$-saturierte Modelle
		
		$\mathcal{L} = \{ E \}$
		
		$T = $ $E$ ist Äquivalenzrelation, für jedes $n \geq 1$ gibt es genau eine Klasse mit $n$ Elementen.
		
		Falls nur endlich viele Klassen mit unendlich vielen Elementen entstehen und $A$ eine endliche Menge ist die aus jeder unendlichen Klasse mindestens ein Element enthält.
		
		$p = (\{ \lnot E x a \mid a \in A \} \cup \{ \exists^{\geq n} y E x y \mid n \in \setN \})^\vdash$
		
		ist dann nicht realisiert. Für unendliche viele unendliche Äquivalenzklassen allerdings schon.
	\end{example}

	\begin{corollary}
		Sei $\mathcal{L}' \subseteq \mathcal{L}$, $\mathfrak{A}$ eine $\mathcal{L}$-Struktur und $a_1, \dots, a_k \in A$
		\begin{enumerate}
			\item $\operatorname{Th}(\mathfrak{A})$ $\aleph_0$-kategorisch $\Rightarrow$ $\operatorname{Th}(\mathfrak{A} \upharpoonright \mathcal{L}')$ $\aleph_0$-kategorisch
			\item $\operatorname{Th}(\mathcal{A})$ $\aleph_0$-kategorisch $\Leftrightarrow$ $\operatorname{Th}/\mathfrak{A}, a_1, \dots, a_k$ $\aleph_0$-kategorisch
		\end{enumerate}
	\end{corollary}

	\begin{proof}
		\heading{1)} Seien $\varphi(x_1, \dots, x_n), \psi(x_1, \dots, x_n)$ $\mathcal{L}'$-Formeln und
		\begin{equation*}
			\operatorname{Th}(\mathfrak{A}) \vdash \forall \overline{x} (\varphi(\overline{x} \leftrightarrow \overline{x}))
		\end{equation*}
		das heißt $\varphi \sim_{\operatorname{Th}(\mathfrak{A})} \psi$
		mit dem Interpolationssatz folgt
		\begin{equation*}
			\operatorname{Th}(\mathfrak{A}) \upharpoonright \mathcal{L}' = \operatorname{Th}(\mathfrak{A} \upharpoonright \mathcal{L}') \vdash \forall \overline{x}(\varphi(\overline{x}) \leftrightarrow \psi(\overline{x}))
		\end{equation*}
		Ryll-Nardzewski
		
		$\operatorname{Th}(\mathfrak{A})$ $\aleph_0$-kategorisch $\Leftrightarrow$ $\forall n$ nur endlich viele Formeln $\varphi(x_1, \dots, x_n)$ modulo $\operatorname{Th}(\mathfrak{A})$
		
		$\Rightarrow$ $\forall n$ nur endlich viele Formeln $\varphi(x_1, \dots, x_n)$ modulo $\operatorname{Th}(\mathfrak{A} \upharpoonright \mathcal{L}')$
		
		$\Leftrightarrow \operatorname{Th}(\mathfrak{A} \upharpoonright \mathcal{L'})$ $\aleph_0$-kategorisch
		
		\heading{2)}
		
		\heading{$\Leftarrow$} Spezialfall von 1)
		
		\heading{$\Rightarrow$}
		
		$\operatorname{Th}(\mathfrak{A})$ $\aleph_0$-Kategorisch $\Leftrightarrow$ jedes Modell von $\operatorname{Th}(\mathfrak{A})$ ist $\aleph_0$-saturiert
		
		$\Rightarrow$ Jedes Modell von $\operatorname{Th}(\mathfrak{A}, a_1, \dots, a_n)$ ist $\aleph_0$-saturiert
		
		$\Leftrightarrow$ $\operatorname{Th}(\mathfrak{A}, a_1, \dots, a_k)$ $\aleph_0$-kategorisch
	\end{proof}

	\begin{remark}
		Wenn $\mathfrak{M}^+ \models \operatorname{Th}(\mathfrak{A}, a_1, \dots, a_k)$ als $\mathcal{L}_{\{ c_1, \dots, c_k \}}$-Theorie, dann ist $\mathfrak{M}^+ \upharpoonright \mathcal{L} \models \operatorname{Th}(\mathfrak{A})$
		
		Aber nicht jedes Modell von $\operatorname{Th}(\mathfrak{A})$ lässt sich notwendigerweise zu einem Modell von $\operatorname{Th}(\mathfrak{A}, a_1, \dots, a_k)$ expandieren.
	\end{remark}

	\begin{definition}
		Eine $\mathcal{L}$-Theorie $T$ heißt schmal, falls $T$ vollständig und abzählbar und für jedes $n \in \setN$ ist $\abs{S_n^T} \leq \aleph_0$.
	\end{definition}

	\begin{theorem}
		Sei $T$ vollständig und abzählbar. Dann sind äquivalent:
		\begin{enumerate}
			\item $T$ hat ein abzählbares $\aleph_0$-saturiertes Modell
			\item $T$ ist schmal
			\item Für alle $\mathfrak{M} \models T$, alle $n$ und alle $a_1, \dots, a_n \in M$ ist $\abs{S_1^\mathfrak{M}(a_1, \dots, a_n)} \leq \aleph_0$
		\end{enumerate}
	\end{theorem}

	\begin{proof}
		\heading{$1) \Rightarrow 2)$} Sei $\mathfrak{M}_\omega$ abzählbares saturiertes Modell von $T$. Dann realisiert $\mathfrak{M}_\omega$ alle $n$-Typen über endlichen Parametermengen, also insbesondere über $\emptyset$. ($S_n^\mathfrak{M}(\emptyset) = S_n^T$, da $T$ vollständig)
		
		\begin{equation*}
			\abs{S_n^T} = \abs{S_n^\mathfrak{M}(\emptyset)} \leq \abs{M^n} = \aleph_0
		\end{equation*}
		
		\heading{$2) \Rightarrow 3)$}
		\begin{align*}
			S_n^\mathfrak{M}(a_1, \dots, a_n) &\to S_{n+1}^\mathfrak{M}(\emptyset) \\p = \{ \varphi(x, \overline{a}) \mid \varphi \in p \} &\mapsto \tilde{p} = \{ \varphi(x, \overline{y}) \mid \varphi \in p \}
		\end{align*}
		die Abbildung ist injektiv. Wenn $p \neq p'$, dann gibt es $\varphi(x, \overline{a}) \in p \setminus p'$ also $\varphi(x, \overline{y}) \in \tilde{p} \setminus \tilde{p}'$
		
		\heading{$3) \Rightarrow 1)$} Kettenkonstruktion
		
		Starte mit $\mathfrak{M}_0 \models T$ abzählbar. Es gibt abzählbar viele endliche Teilmengen von $M_0$.
		
		Nach Vorraussetzung: über jedem endlichen $A \subseteq M_0$ gibt es nur abzählbar viele $1$-Typen.
		
		Also gibt es insgesamt nur abzählbar viele $1$-Typen über endlichen Teilmengen von $\mathfrak{M}_0$.
		
		Realisiere alle diese Typen in $\mathfrak{M}_1^* \succcurlyeq \mathfrak{M}_0$. Sei $C_1 \subseteq M_1^*$ abzählbar und so, dass für jeden Typ $p \in S_1^{\mathfrak{M}_0}(A)$ eine Realisierung in $C_1$ liegt.
		
		Löwenheim-Skolem-Tarski: Es gibt $\mathfrak{M_1} \preccurlyeq \mathfrak{M}_1^*$ mit $C \cup M_0 \subseteq M_1$ und $M_1$ abzählbar. $\mathfrak{M}_1$ ist abzählbar und realisiert alle $1$-Typen über endlichen Parametermengen von $M_0$. Wir erhalten weiter
		\begin{equation*}
			\mathfrak{M}_0 \preccurlyeq \mathfrak{M}_1 \preccurlyeq \mathfrak{M}_2 \preccurlyeq \dots
		\end{equation*}
		Setze $\mathfrak{M}_\omega = \bigcup_{n \in \omega} \mathfrak{M}_n$ diese ist abzählbar und $\aleph_0$-saturiert.
	\end{proof}

	\begin{theorem}[Vaught]
		Es gibt keine abzählbare vollständige Theorie, die genau $2$ abzählbare Modelle bis auf Isomorphie hat.
	\end{theorem}

	\begin{proof}
		\heading{1. Fall:} $T$ nicht schmal, das heißt es gibt $n_0 \in \setN$ mit $\abs{S_{n_0}^T} > \aleph_0$. Jeder $n_0$-Type ist aber in einem abzählbaren Modell realisiert, das heißt es gibt überabzählbar viele abzählbare Modelle.
		
		\heading{2. Fall:} $T$ schmal, nicht $\aleph_0$-kategorisch: Das heißt es gibt $n$ und einen nicht-isolierten Typ $p \in S_n^T$.
		
		Omitting Types: Es gibt abzählbares Modell, das $p$ übergeht. Außerdem gibt es abzählbares Modell, das $p$ realisiert.
		
		Sei $\overline{c} \in M_1$ ein Tupel, das $p$ realisiert. $\operatorname{Th}(\mathfrak{M}_1, \overline{c})$ ist nicht $\aleph_0$-kategorisch, hat also $2$ nicht isomorphe abzählbare Modelle $(\mathfrak{M}_1, \overline{c}), (\mathfrak{M}_2, \overline{c})$. Also ist $\mathfrak{M}_2 \models T$, das $p$ realisiert, und $\mathfrak{M}_2 \ncong \mathfrak{M}_1$.
	\end{proof}

	\begin{example}
		$\mathcal{L} = \{ <, P_1, \dots, P_n, c_0, c_1, \dots \}$
		
		Wobei $<$ eine dichte lineare Ordnung beschreibt. $P_i$ bilden dichte Partition
		\begin{equation*}
			\forall x_1 \forall x_2 (x_1 < x_2 \rightarrow \exists y_1, \dots, y_n \bigwedge_{i=1}^n (x_y < x_i < x_2) \land P_i y_i)
		\end{equation*}
		
		Dies hat dann $n+2$ abzählbare Modelle.
	\end{example}

	\begin{example}
		Sei $T$ eine vollständige abzählbare Theorie
		\begin{equation*}
			I(T, \aleph_0) = \text{Anzahl der abzählbaren Modelle bis auf} \cong
		\end{equation*}
	\end{example}

	Welche Werte sind möglich? $1,3,4,5, \dots, \aleph_0, 2^{\aleph_0}$
	
	Nicht möglich: $2$
	
	\subsubsection{Der Satz von Fra\"\i s\'{e}}
	
	Sowohl $(\setQ, <)$ als auch der Zufallsgraph haben Quantorenelimination und sind $\aleph_0$-kategorisch.
	
	Endliche Substrukturen von $(\setQ, <)$ sind alle endlichen Ordnungen, vom Zufallsgraph sind es alle endlichen Graphen.
	
	\begin{definition}
		Eine $\mathcal{L}$-Struktur heißt schwach homogen, falls gilt: Wenn $A, B, A'$ endlich erzeugte Unterstrukturen von $\mathfrak{M}$ sind, $A \subseteq A'$ und $\beta: A \to B$ ein Isomorphismus, dann existiert $B' \supseteq B$ und ein Isomorphismus $\beta': A' \to B'$ der $\beta$ fortsetzt.
		
		$\mathfrak{M}$ heißt stark homogen, falls gilt: Wenn $A, B$ endliche erzeugte Unterstrukturen und $\beta: A \to B$ ein Isomorphismus, dann setzt sich $\beta$ zu einem Automorphismus von $\mathfrak{M}$ fort.
	\end{definition}

	\begin{theorem}[Fra\"\i s\'{e}]
		Sei $\mathcal{L}$ endliche, relationale Sprache. Sei $\mathcal{K}$ eine Klasse endlicher $\mathcal{L}$-Strukturen mit
		\begin{itemize}
			\item $\emptyset \in \mathcal{K}$ (ausnahmsweise wird $\emptyset$ als $\mathcal{L}$-Struktur betrachtet!)
			\item $\mathcal{K}$ ist abgeschlossen bezüglich Isomorphie und Substruktur
			\item $\mathcal{K}$ hat die Amalgamierungseigenschaft
		\end{itemize}
		Dann existiert eine (bis auf $\cong$) eindeutige abzählbare $\mathcal{L}$-Stuktur $\mathfrak{M}$ mit:
		\begin{itemize}
			\item $\mathfrak{M}$ ist schwach homogen
			\item $\mathcal{K} = \operatorname{Skelett}(\mathfrak{M}) \coloneqq \{ \mathfrak{A} \text{ endl. $\mathcal{L}$-Struktur} \mid \text{ es gibt } \mathfrak{A}' \subseteq \mathfrak{M}: \mathfrak{A} \cong \mathfrak{A}' \}$
		\end{itemize}
		Außerdem gilt
		\begin{itemize}
			\item $\mathfrak{M}$ ist stark homogen
			\item $\mathfrak{M}$ ist $\aleph_0$-kategorisch
			\item $\mathfrak{M}$ hat Quantorenelimination
		\end{itemize}
	\end{theorem}

	\begin{proof}
		\begin{enumerate}
			\item Konstruktion
			
			Sei $\{ h_i: A_i \to B_i \mid i \in \omega \}$ eine Aufzählung aller Isomorphietypen von Einbettungen zwischen $\mathcal{K}$-Stukturen, wobei jeder Isomorphietyp unendlich oft vorkomme.
			
			\begin{center}
				\begin{tikzpicture}
					\node(R1) at (0,0){$h: A$};
					\node(ri) at (3,0){$B$};
					\node(R2) at (0, -2){$h: A'$};
					\node(s) at (3, -2){$B$};
					
					\draw[->] (R1) -- (ri);
					\draw[->] (R1) -- (R2) node[midway,left]{$\alpha$};
					\draw[->] (R2) -- (s);
					\draw[->] (ri) -- (s) node[midway,right]{$\beta$};
				\end{tikzpicture}
			\end{center}
		
			$h, h'$ sind isomorph, falls es Isomorphismen $\alpha: A \to A', \beta: B \to B'$ gibt mit $\beta \circ h = h' \circ \alpha$.
			
			Für jede endliche Menge $A$ gibt es nur endlich viele mögliche $\mathcal{L}$-Stukturen auf $A$ da $\mathcal{L}$ endlich. Insgesamt gibt es abzählbar viele Isomorphietypen.
			
			Konstruiere induktiv $\mathfrak{M}_n \in \mathcal{K}$:
			
			\heading{$n=0$} $\mathfrak{M}_0 = \emptyset$
			
			\heading{$n \to n + 1$} Wähle minimales $i \geq n$ mit $A_i \hookrightarrow \mathfrak{M}_n$. Seien $A_i^0, \dots, A_i^{k_i}$ die Bilder der möglichen Einbettungen von $A_i$ in $\mathfrak{M}_n$. Amalgamire
			
			$\mathfrak{M}_n$ mit $B_i$ über $A_i^0$ zu $\mathfrak{M}_n^1 \in \mathcal{K}$
			
			$\mathfrak{M}_n^1$ mit $B_i$ über $A_i^1$ zu $\mathfrak{M}_n^2 \in \mathcal{K}$
			
			$\vdots$
			
			$\mathfrak{M}_n^k$ mit $B_i$ über $A_i^k$ zu $\mathfrak{M}_{n+1} \in \mathcal{K}$
			
			Setze $\mathfrak{M} = \bigcup_{n \in \omega} \mathfrak{M}_n$.
			
			\item Zeige $\mathcal{K}$ ist Skelett von $\mathfrak{M}$.
			
			\heading{$\operatorname{Skelett}(\mathfrak{M}) \subseteq \mathcal{K}$:} Sei $A \subseteq \mathfrak{M}$ endlich. Dann gibt es $n$ mit $A \subseteq \mathfrak{M}_n \in \mathcal{K}$.
			
			Also $A \in \mathcal{K}$, da $\mathcal{K}$ abgeschlossen bezüglich $\subseteq$.
			
			$A' \in \operatorname{Skelett}(\mathfrak{M}) \Rightarrow A' \cong A \subseteq \mathfrak{M}$
			
			$\Rightarrow A' \in \mathcal{K}$
			
			\heading{$K \subseteq \operatorname{Skelett}(\mathfrak{M})$:} Sei $B \in \mathcal{K}$. Dann gibt es $B' \cong B$ so dass $\emptyset \hookrightarrow B'$ in der Aufzählung der Isomorphietypen vorkommt.
			
			Sei $i_0$ der erste Index des Vorkommens. Da $\emptyset \hookrightarrow \mathfrak{M}_{i_0}$ wird im Konstruktionsschritt $i_0 \to i_0 + 1$ die Einbettung $\emptyset \hookrightarrow B'$ zu $\mathfrak{M}_{i_0}$ amalgamiert. Also $B'$ bis auf Isomorphie $\subseteq \mathfrak{M}_{i_0 + 1} \subseteq \mathfrak{M}$ das heißt $B \in \operatorname{Skelett(\mathfrak{M})}$.
			
			\item $\mathfrak{M}$ ist schwach homogen
			
			Es gibt $n$ mit $B \subseteq \mathfrak{M}_n$.
			
			Der Isomorphietyp der Erweiterung $A \subseteq A'$ taucht in der liste unendlich oft auf. Sei $i_0$ der kleinste Index $\geq n$ eines Vorkommens dieses Isomorphietyps. Im Konstruktionsschritt $i_0 \to i_0 + 1$ wird ein zu $A'$ isomorphes $B'$ über $B$ hinzuamalgamiert.
			
			\item Seien $\mathfrak{M}, \mathfrak{N}$ schwach homogen mit Skelett $\mathcal{K}$, $A$ gemeinsame Unterstruktur. Zeige mit back-and-forth $\mathfrak{M} \cong_A \mathfrak{N}$. Daraus folgt: mit $A = \emptyset$ die Eindeutigkeit und die $\aleph_0$-Kategorizität.
			
		
		\end{enumerate}
	\end{proof}

	\begin{example}""
		\begin{enumerate}
			\item endliche Ordnungen $\rightsquigarrow$ $\mathfrak{M} = (\setQ, <)$
			\item endliche Graphen $\rightsquigarrow$ $\mathfrak{M} = $ Zufallsgraph
			\item endliche Turniere = gerichteter Graph, zwischen je $2$ Punkten gibt es genau eine gerichtete Kante
			\item endliche partielle Ordnungen
			\item endliche geordnete Graphen $\rightsquigarrow$ $\mathfrak{M} = (\setQ, <, E)$ ein wie $\setQ$ geordneter Zufallsgraph
		\end{enumerate}
	\end{example}

\end{document}